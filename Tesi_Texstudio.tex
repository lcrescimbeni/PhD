\documentclass[12pt,a4paper]{report}
\usepackage[utf8]{inputenc}
\usepackage{float}
\usepackage{amsmath}
\usepackage{amsfonts}
\usepackage{textcomp}
\usepackage{graphicx}
\usepackage{subcaption}
\usepackage{hyperref}
\linespread{1}
\usepackage[english]{babel}
\usepackage{amssymb}
\usepackage[dvipsnames]{xcolor}
\usepackage[
%backend=biber, 
natbib=true,
style=numeric,
sorting=none
]{biblatex}
\usepackage{graphicx}
\usepackage[left=2cm,right=2cm,top=1cm,bottom=1.5cm]{geometry}
\newcommand{\rem}[1]{[\emph{#1}]}
\newcommand{\exn}{\phantom{xxx}}
\addbibresource{Bib.bib}

\begin{document}
	\begin{titlepage} %crea l'enviroment
		\begin{figure}[t] %inserisce le figure
			\begin{minipage}[t]{0.5\textwidth}\raggedleft
				
				\centering\includegraphics[width=0.8\textwidth]{Plot/Misc/GSI_logo.png}
			\end{minipage}
			\begin{minipage}[t]{0.5\textwidth}\raggedright
				
				\centering\includegraphics[width=0.8\textwidth]{Plot/Misc/Frankfurt_logo.png}
			\end{minipage}
		\end{figure}
		\vspace{200mm}
		
		\begin{Large}
			\begin{center}
				\vspace{40mm}
				{\LARGE{DISSERTATION}}\\
				\vspace{20mm} 
				\text{zur Erlangung des Doktorgrades}\\
				\text{der Naturwissenschaften} \\
				\vspace{20mm}
				\text{vorgelegt beim Fachbereich Physik}\\
				\text{ der Johann Wolfgang Goethe-Universität}\\
				\text{in Frankfurt am Main}\\
				\vspace{10mm}
				{\huge{\bf Cryogenic Current Comparator for FAIR.}}\\
			\end{center}
		\end{Large}
		
		
		\vspace{36mm}
		%minipage divide la pagina in due sezioni settabili
		\begin{minipage}[t]{0.47\textwidth}
			{\large{\bf Relator:\\ Prof. Holger Podlech}}
			
		\end{minipage}
		\hfill
		\begin{minipage}[t]{0.47\textwidth}\raggedleft
			{\large{\bf Candidate: \\ Lorenzo Crescimbeni\\ }}
		\end{minipage}
		
		\vspace{10mm}
		
		\hrulefill
		
		\vspace{5mm}
		
		\centering{\large{\bf Frankfurt am Main, 2024 }}
		
	\end{titlepage}
	\chapter*{Abstract}
	
	
	
	\tableofcontents
	
	\chapter{Introduction}
	Particle physics studies has been a fundamental part of physics in the last century, and so are the machines that allows to perform them, the particle accelerators. Thanks to them physicists where able to develop theories, like the standard model, that were able to unveil some of the mysteries of our universe. In addiction, particle accelerator are now used all over the world for health-care and industry, becoming a fundamental tool also in the life of not scientists. The work to improve particle accelerators is now more important than ever, with accelerator physicist all over the world trying to push the boundaries of existing technology to develop new facilities or improve the existing ones to solve the remaining question regarding fundamentals physics, like dark matter or violation of the standard model, or to develop new technologies, spacing from sources of clean energies, techniques for reducing pollution, new health treatments, and an infinite number of other researches in every technological and scientific field. \\
	As other facilities around the world, GSI is also taking part in this effort, with the FAIR project. FAIR (Facility for Antiproton and Ion Research) is the under construction upgrade to the existing GSI facility, that will strongly increase the available beam parameters of the research center, with higher particle numbers and species available, together with increased energies and intensities, thus allowing to perform sensitive work for new discoveries in a broad spectrum of disciplines. Requirements of the new facility ask also for improved beam diagnostic, that needs to be able to cover intensity ranges that are not covered by existing diagnostics.\\
	In particular, the FAIR facility needs a diagnostic device that is able to provide a non destructive beam current measurement of low intensity beams, and for this reason the Cryogenic Current Comparator (CCC) has been chosen.\\
	The CCC is a complex beam diagnostic device that uses superconductivity technologies to provide a non destructive, absolutely calibrated, measure of beam intensity (beam current) in the order of nA. This thesis will cover the development and improvement of a CCC prototype for the FAIR facility, covering its construction, functionality, the improvements that has been performed and the test in GSI transfer lines, to show the prototype functionality as a detector for FAIR. \\
	The first chapter will provide an introduction to the GSI and FAIR facilities, for a better understanding of the experimental environment, together with a short overview of the previous development history of the CCC at GSI. To better suit FAIR diagnostics needing the CCC has to be used also as a detector for spill analysis, meaning that a core part of this thesis will be the analysis of slow extracted beam using the CCC, thus in this chapter is possible to find a brief introduction to slow extraction, that will work as a basis for further discussion of the CCC test results. Furthermore, a short summary of existing beam diagnostics used at GSI will be presented, for a better understanding of why the CCC is included in the essential beam diagnostic for FAIR.\\
	\section{GSI and FAIR facilities}
	The work presented in this thesis has been mainly performed in the "GSI Helmholtzzentrum für Schwerionenforschung" (Helmholtz center for heavy ion research), an accelerator located in Darmstadt, Germany, and all the work have been focused on a detector capable of satisfying the request of FAIR, the planned expansion of the existing GSI facility. A characteristic of GSI is the capability of accelerating ions from the elementary particle (protons) up to the heaviest stable element (Uranium), so with a broad range of charge state and masses. GSI facility is mainly known for the discovery of several super-heavy elements (i.e. \cite{Elements1}, \cite{Elements2}) and tumor therapy with ion beams \cite{TumorTherapy}.\\
	Over the years, the research fields that have consolidated as the main ones at GSI are biophysics, material and medical science, together with atomic, nuclear, plasma and particle physics. In Fig. \ref{GSI} it's possible to see a schematic view of the existing facility (in blue) and the upcoming FAIR \cite{FAIR}. \\
	\begin{figure} [H]
		\centering
		\includegraphics[width=1\linewidth]{Plot/Chapter_0/GSI_facility.png}
		\caption{\small{Schematic view of GSI/FAIR accelerator facilities where the actual existing facility is in blue and all the upcoming ones are highlighted}}
		\label{GSI}
	\end{figure}
	The GSI facility consist of the UNIversal Linear ACcelerator UNILAC, that accelerates the ions produced in the ion sources up to an energies of 30 MeV/u. These beams are provided to experiments in the first experimental hall or injected in the heavy ion synchrotron SIS 18, that is capable to accelerate all kinds of injected ions from UNILAC to an energy peak dependent on the ion species (up to 1 GeV/u for U$^{73+}$ ions to 4.5 GeV/u for protons). The accelerated ions are then extracted and go through the FRagment Separator (FRS) where short-lived radioactive secondary beam are produced via collision with a thin beryllium target. These beams can be stored in the Experimental Storage Ring (ESR), where thanks to stochastic and electron cooling are able to reach excellent beam properties (with an energy between 4 and 500 MeV/u), or re-injected into SIS 18 for further acceleration. The beam can then be extracted (from SIS18 or ESR) and sent to experiments in the following experimental hall, or injected into CRYRING@ESR, the last storage ring of the chain where cooled beams can be stored for long amount of time for further manipulations and in-ring experiments.\\
	The experimental caves located after ESR are locations in which the beam condition can be really similar to the ones of beams that will be injected into the upcoming FAIR facility, and can then be used as optimal testing point for devices that will be installed in the transfer lines of FAIR. In fact, the existing GSI facility will be used as injection for FAIR, where the beams extracted from SIS18 (after possible manipulation through the FRS and ESR) are then injected in a transfer line (Fig. \ref{GSI} in red) leading to SIS100, a larger storage ring (maximum magnetic rigidity $\rho = 100 \text{ T} \cdot \text{m} $) where the particle beam will be further accelerated up to 29 GeV/u, depending on ion species. The particles extracted from SIS100 will be then sent to several experiment and storage rings, allowing to study a wide spectrum of particle species and beam intensities that has never been available before in GSI. This, together with all the possible operational modes of the storage rings and different extraction modes and parameters, demands for a really high dynamic range of intensity measurement in all the FAIR facility, independently of ion species and beam parameters.
	\section{A brief introduction to slow extraction}
	One of the core part of the accelerator operation is the extraction of particles from a synchrotron, that can then be sent to other accelerator machines for further manipulation or to the experiments. One of the method adopted to extract the beam is called slow extraction, and analysis of slow extracted particle beams, also called spills, is a part of this work, thus it is possible to find in this section a short introduction to the topic, further information can be found in \cite{Slow_extraction} \cite{Cern_slow_extraction}.\\
	Inside a synchrotron the particles follows a beam trajectory that is defined by magnetic elements with different order of magnitude, dipole, quadrupole and sextupole, while the rf cavities provide accelerating fields to manage the beam energy. Dipole are used to steer the particle beams to keep it along the defined trajectory of the machine, quadrupole are used to focus and de-focus the beam and sextupoles are used to compensate the beam chromaticity. This magnetic elements are active together to keep the beam stable on the reference orbit and are referred to as beam optics. It is the trajectory of a reference particle with momentum $|\vec{p}_0|$, the nominal momentum of the particle in the ring. It's also possible to define the nominal revolution frequency $f_0 = \frac{2\pi R}{|\vec{v}_0|}$ where $2\pi R$ is the circumference of the synchrotron ring (SIS18 = 216 m) and $|\vec{v}_0| $ the nominal velocity. \\
	On the transverse direction particles are subjected to oscillation called the betatron oscillation. The amount of betatron oscillation performed in a revolution of the machine is called betatron tune. It's essential, for the beam stability, that this tune is not an integer or a fraction of an integer, if this condition is not verified we are then in a situation in which the particle are always in the same position every n round (with n an integer) generating a resonance. If in these resonance point it's present a small force, that can also originate from imperfection, like magnet misalignment or field errors, the particle will receive the same impulse, in the same direction, every n round. All these small impulse will add up generating a resonant growth of the betatron amplitude until it is too high and the particle is lost (because it hits the beam pipe or components of the accelerator machine).\\
	On the longitudinal direction we need to distinguish two modes for the particle beam, the coasting beam and the bunched beam. If no longitudinal force are applied, the particle will revolve inside the synchrotron with a constant angular velocity with a stable momentum, and the particle moves forward or backward relative to the synchronous particle depending on their energy in respect to it (the one on the reference orbit).\\
	It's possible to divide the particle beam in bunches by turning on a sinusoidal voltage $U_{rf}$ provided by one or more rf cavity. The particles circulating with the nominal energy reach the cavities synchronous to the desired phase $\Phi_0$ of the electric field, while the particles with a difference in energy reach the rf cavity before or after the nominal particle, thus finding different phases of the electric field, oscillating around the nominal particle. If the energy of a particle is too different from the nominal energy it will not oscillate around the stable particle and will get lost during the acceleration process, because it is not stable. The particles moves than in bunches, and while they are following the nominal trajectory, the particles that compose the bunches oscillate. To describe the relative position of a particle inside the bunch it's possible to use the phase space eclipse. The area of the phase space eclipse it's called Emittance and it's constant if the energy of the particles doesn't change. The longitudinal phase space of bunched beams is shown in Fig.\ref{Phase space}, where the separatrix is the boundary that separate the stable from the unstable regions in the phase space. If a particle crosses the separatrix it is not stable anymore and is lost. \\
	\begin{figure} [H]
		\centering
		\includegraphics[width=0.8\linewidth]{Plot/Chapter_0/Phase_space.png}
		\caption{\small{Longitudinal phase space of bunched beams is shown. h is the harmonic of the accelerator (in SIS18 the bunching is provided by two identical rf cavities that generates 4 rf bucket, thus h=4). Separatrix mark the area of the phase in which the particles are stable, if the energy difference of a particle increase enough to overcome the separatrix is lost.}}
		\label{Phase space}
	\end{figure}
	Oscillations in the longitudinal direction are called synchrotron oscillation and they have a frequency that can be calculated with:
	\begin{equation}
		f_s=f_{rev}\sqrt{|\frac{\eta \cdot (Q \cdot U_{rf}) \cdot h \cdot \cos{\phi_0} }{2 \pi \cdot \beta^2 \cdot E }|}
	\end{equation}
	where $f_{rev}$ is the revolution frequency of the nominal particle in the synchrotron, $\eta = \frac{1}{\gamma^2} - \alpha_p$ is called sleeping factor, with $\gamma$ the Lorentz Factor and $\alpha_p$ the momentum compaction factor. $Q \cdot U_{rf}$ is the energy gain by each passage through an rf cavity (Q is the elementary charge and $U_{rf}$ is the accelerating voltage), $\phi_0$ is the reference phase of the nominal particle, $\beta$ the relativistic velocity and E the nominal energy of the particle beam.\\
	It's necessary now to describe how particles are extracted from the synchrotron. To perform slow extraction the third integer resonance of the accelerator it's used, achieved when the tune is $Q=\frac{n}{3}$ with n an integer. The third order resonance can be excited using a sextupole, achieving a phase space where the stable areas is marked by three separatrix, as shown in Fig.\ref{Third_resonance}. The calculations and the third integer resonance complete theory are beyond the scope of this thesis and can be found in \cite{Cern_slow_extraction}, for this thesis it's important to know that the stable area is $\frac{\epsilon}{S}$ where $\epsilon$ is a quantity that is correlated to machine tune and it's 0 if the tune is a third order integer and S is correlated to the sextupole strength. \\
	\begin{figure} [H]
		\centering
		\includegraphics[width=0.8\linewidth]{Plot/Chapter_0/Third_integer_resonance.png}
		\caption{\small{Phase space in the third integer resonance excited by a sextupole. The three separatrix mark the stable triangle, the area of this triangle depends on the sextupole strength and on the machine tune.}}
		\label{Third_resonance}
	\end{figure}
	It's then possible to shrink the stable area enough to make particles cross the separatrix and leave the ring. To perform the slow extraction from a synchrotron using the third integer resonance there are three main methods:
	\begin{itemize}
		\item \textbf{Varying the sextupole strength}: It's possible to increase the sextupole strength to decrease the stable area available to the particles. This method is generally not used because the sextupole strength is also correlated to the size of the extracted beam. For medical use and for some experiments it's essential to have the lowest possible beam size.
		\item \textbf{Increasing the particle amplitude}: It's possible to increase the amplitude of the transverse oscillation of the particle to push them out of the stable area. This can be performed with several different methods \cite{Cern_slow_extraction}. In this way it's possible to have particles that are able to cross the separatrix without changing the quadrupoles and sextupole intensity, so while keeping the separatrix fixed, usually thanks to energy changes of the beam to increase its emittance.
		\item \textbf{Moving the tune near resonance value}: This method is the one that is used in GSI. It's possible to change the machine tune by increasing the intensity of quadrupole strength, the more the tune got close to the third integer value, the more the stable area shrinks. After a certain value the stable area is small enough to allow the particles to cross the separatrix and getting extracted.
	\end{itemize}
	A standard scheme for slow extraction can be described as follows: Using quadrupole and sextupole magnets, the resonance condition for the betatron oscillations is fulfilled at a specific orbit, that is generally called resonance orbit. As the beam orbit is shifted closely to the resonance orbit, the amplitude of betatron oscillations increase rapidly. The amplitude will increment until the particle crosses the separatrix in the phase space, where the amplitude of betatron oscillation it's big enough to fall into the septum of the kicker magnet, that provide an impulse to the particle that kick it out of the orbit, thus being extracted from the ring.
	In GSI, in particular in SIS18, the particles can be extracted in spills from 100 ms to several seconds using the third order integer extraction.\\
	To better suit the experiment request the spill should be smooth, uniform, with an even distribution of particle per s for all the spill length. This is particularly important for medical treatment and for experiment that needs to calculate branching ratio of events. Therefore, methods have been tested to further improve spill quality more than what is possible by only changing machine parameters. The effects of those spill smoothing can be examined by measuring the spills using the CCC, and this will be performed in further part of this thesis.
	
	\section{Beam intensity measurement in GSI/FAIR}
	To ensure correct operation of particle accelerators it's essential to measure in a reliable and precise way all beam parameters, making beam diagnostic an essential aspect of accelerator machines. Each of these parameters is measured by one or more diagnostic device, each with specific properties and limitations. In this section we will focus on beam intensity measurement devices used in GSI, to briefly introduce their working principle, performance and limitations. A detailed overview of diagnostic devices for accelerator, that can't be covered in this thesis or that are described only shortly, can be found in \cite{Fork} \\
	Beam intensity is, in fact, one of the most important parameters to measure in an accelerator, because is directly connected to the amount of particles circulating in the machine. Beam intensity it's then used to monitor key machine parameters like transmission efficiency of transfer lines (by measuring the difference between extracted and injected particles), life time of particle beams in storage rings, can be used to identify beam losses and more. This measurement is also essential for experiment, in which the particle number needs to be know with the higher resolution as possible, because it is essential to know how much particle are reaching the experiment to calculate branching ratios. \\
	Usually an accelerator machine needs a wide spread of beam intensity measurement. For example a LINAC (linear accelerator) is a pulsed system with several accelerating cavities, and each beam pass through each of them only once, and each of them change beam parameters. So the diagnostic devices needs to measure the beam parameters of these short pulses that goes through the device only once and those may change sensibly between one pulse and another. So the detector needs to provide beam to beam measurement, to make it possible to analyze each of them separately. In case of a storage ring, instead, beams passes several times through the same cavities and the same detectors, that are usually placed in location of the ring in which the beam reaches a stable condition. This means that it's possible to achieve higher resolution thanks to an average measurement, and the beam current is usually stable or slowly changing. What's challenging in storage ring is usually to have a detector that is able to detect also really small slow changes in beam currents, to have a clear measurement of expected beam lifetime or beam losses. Furthermore storage ring can have both coasting and bunched beam, so it's essential to have detectors able to provide measurement for all the possible configurations. \\
	In most accelerator facilities the frequency range that needs to be covered for beam intensity measurement goes from DC to its RF frequency components (dependent on machine structure, accelerator method, ion species etc...) and it's generally provided as an integrated current measurement in Ampere (typical ranges from nA to mA in GSI) or expressed as particle per second (pps) or particle per spill (if we are talking about extracted beam, usually by slow extraction). \\
	FAIR is expected to have, in the transfer lines, beams with a particle number going up to $10^{12}$, so beam currents going from some $pA$ to $mA$ depending on beam species and extraction time. It's then important to have a detector that is able to measure beam intensity in this range. Detectors interactions with the beam is also a factor that need to be considered, especially for low intensity beam, and should be as low as possible to avoid particle losses and to deliver the less perturbed beam to further part of the machine or to experiments. Existing diagnostic is able to provide a non destructive (or non interceptive, this means that the detector is not interacting directly with the beam, thus providing to the minimum perturbation) measurement of beam currents higher than $\mu A$, but for current below these range measurement became challenging.  \\
	In the following part of this section we will briefly talk about the general used devices for beam intensity measurement, the beam transformers, that are in general limited to a current resolution of 1 $\mu A$, and other devices that are used for lower beam intensities measurement. 
	\subsection{Beam Current Transformers (BCT)}
	One way to detect the beam current is to measure it's magnetic field. A circulating current generates a magnetic field that it's proportional to its intensity and can be calculated thanks to the Biot-Savart law:
	\begin{equation}
		\overrightarrow{B} = \mu_0 \frac{\text{I}_{beam}}{2\pi r} \cdot \overrightarrow{e}_{\varphi}
		\label{Biot-Savart}
	\end{equation}
	that allows to calculate the azimuthal component of the magnetic field generated by a circulating current at a distance r. Only the azimuthal component (indicated by the $\overrightarrow{e}_{\varphi}$ versor) is interesting due to the cylindrical geometry outside the beam.
	This allow to measure the beam current directly by measuring the magnetic field produced by the beam itself. The idea seem simple but it is essential to consider that the magnetic field generated by a particle beam is really low (around 1 pT from a beam of 1 $\mu$A). \\
	BCT has a general structure composed of an high magnetic permeability torus, thus allowing to detect magnetic field of really low intensities, that act like a primary winding of a classical transformer with single winding number ($N_p = 1$ because the beam passes only once through the torus). The azimuthal component of the time varying magnetic field produced by the beam induces a current in the secondary winding, which is wound around the torus with $N_s$ turns and inductance $L_s$. This current is then amplified and measured, thanks to a resistor, as a voltage, providing the output signal of the transformer. This structure can be then modified depending on the desired use of the transformer. This is the basic characteristics of the FCT (fast current transformer) that is generally used to measure short beam pulses, but is possible to add a feedback circuit to improve the low frequency response of the transformer, that is otherwise limited by the inductance of the secondary winding and the load impedance, to have a transformer that can be able to measure longer beam pulses (up to 1 s) called ACCT (Alternating Current Current Transformer). It's also possible to measure coasting beams using a more complex beam transformer called DCCT (Direct Current Current Transformer), that uses multiple torus and additional winding to measure dc beams. Further information on the BCTs can be found in \cite{Fork}. \\
	Independently on the type of BCTs used it's essential to have a magnetic shielding to shield the torus from external fields (the standing magnetic field of the earth is $\approx 50 \mu T$, several order of magnitude higher than the magnetic fields that needs to be detected) and to have a ceramic gap, in fact if the metallic beam pipe is not interrupted the mirror current flowing through it would cancel the resulting field of the beam. \\
	Furthermore the sensitivity of BCT is not really high, with a minimum detection threshold around 1 $\mu$A due to thermal noise (mainly due to the load resistor) called Barkhausen noise \cite{Bk_noise}. Furthermore, the performance of BCTs are dependent on temperature fluctuations, electrical noise (amplifier and other active components), by the magnetic core design and by external magnetic interference, thus the BCTs are optimal diagnostic tools only for beam current over $\mu$A.
	\subsection{Faraday cups}
	One of the most sensitive detector used for beam intensity measurement is the Faraday cup. This detector consist of a cup made of an electrical conductor which is inserted into the beam line, thus being a completely destructive measurement device for the beam. The signal produced by the particle stopped in the cup is amplified and read, providing a direct measurement of the particle number. With these device is possible to achieve very precise measurement in a broad range of intensity (from 10 pA to 10 mA or higher, depending on the device characteristics). It is usually equipped with an electric potential at the entrance of the cup and a standing magnetic field to keep all the secondary electrons produced by the interaction of the particle in the cup material. If some secondary electrons are lost the measurement of charge collected with the Faraday cup will be wrong. \\
	Faraday cup are usually used for beam of low energy, around 100 MeV/u maximum. Higher energy beams will have particles that doesn't stop in the cup or can even destroy the cup itself, due to the heat generated by the energy that the particle release while going through the cup material. Often an active cooling system is implemented to avoid thermal caused breakdown also in the usual range of operation\\
	Faraday cups are then able to provide an high resolution measurement of low intensity beams, but they can be used only for low energy beams and are completely destructive, thus it is possible to use them as diagnostic only when the beam can be interrupted and destroyed (during machine setup or as beam dump after experiments).\\
	\subsection{Lower intensity beam measurement}
	The general energy of beams is higher than the limit of Faraday cups, and having a completely destructive measurement of beam currents it's almost never desired in standard operation where the beam needs to be provided to experiments. The beam that are in generally provided to experiments are usually slow extracted from a synchrotron, and slow extracted beams has a quite broad range of intensities, in GSI are generally between 10$^4$ and $10^{12}$ particle per seconds, that generally corresponds to a beam current range from pA to hundreds of $\mu A$, depending on extraction time. Thus BCTs can't be used, because this current intensities are in general below their detection threshold. In the following part of this section it's possible to find a brief description of standard devices used for intensity measurement, an overview of their application ranges is shown in Fig. \ref{Ranges}
	\begin{figure} [H]
		\centering
		\includegraphics[width=0.8\linewidth]{Plot/Chapter_0/Diagnostic_range.png}
		\caption{\small{Application ranges of detector used in GSI synchrotron for beam intensity measurement. Also the Space Charge Limit (SCL) of the synchrotron at injection energy is marked. Courtesy of P.Forck \cite{Fork}}}
		\label{Ranges}
	\end{figure}
	The standard diagnostic for this intensity ranges uses the energy lost in matter by charged particles to measure the beam intensity, thus are all considered intercepting devices. Together with the standard diagnostic will be shortly introduced the Cryogenic Current Comparator (CCC), an alternative solution to standard diagnostic and the main topic of this thesis.
	\subsubsection{Scintillators}
	Scintillators are detector based on the phenomenon of scintillation, typical of particular materials, in which the energy released by impacting charged particle is absorbed and then released as ultraviolet or visible light in a certain timescale. Scintillating material can have different proprieties, and in general the material used for the detector is chosen depending on the wavelength of the light emitted and on the timescale of the emission. It's also essential that the scintillator is transparent to the emitted light. \\
	Scintillators are detectors composed of a plate of scintillating material (with a dimension higher than the maximum beam cross section) that got crossed by the beam. The charged particles release energy while going through the material, which amount is proportional to the number of particles composing the beam. The photon are then collected through a wave guide by a photo-multiplier tube or SIPMs (SIlicon PhotoMultipliers), that converts them into electrical pulses that are counted by a scaler as a measure of the number of particles in the beam.\\
	The correct choice of scintillating material is essential to achieve an high resolution, together with the correct type of photo-multiplier or SIPM depending on the wavelength of the emitted photons \cite{Scintillators}. Plastic scintillators are widely used, they are cheap and easy to produce in any mechanical shape, and they also have relatively short decay time of few nanosecond. A short decay time is essential to be sure that all the photons produced by the previous passing of the beam is gone before the new beam needs to be measured. \\
	This is an interceptive measurement method (the scintillator is crossed by the beam) but scintillators are able to measure up to $10^6$ particle per second. The measurement provided by a scintillator is absolute without the need for an external calibration, and they are often used for direct current measurement of for the calibration of devices which are suitable for higher intensities. In GSI the main issue with scintillator is also the quite short lifetime, plastic scintillators can withstand a certain amount of radiation before starting to suffer radiation damage, asking then for correction of the signal to keep in count the deterioration or to replacement of the scintillating material. Different materials with higher radiation hardness can be used, like inorganic crystal, but they are more expensive and much difficult to produce with sufficient dimension to be used for beam detection.
	\subsubsection{Ionization Chambers}
	The Ionization Chamber (IC) is a detector that use the ionization of gas in a volume to measure the beam current. When the beam crosses the detector chamber full of gas the energy released by the particle beam causes partial ionization of the gas molecules and generates electron-ion couples. The active volume is usually confined within two electrodes (metalized plastic foils, with a thickness around 100 $\mu m$) that are kept at an high potential difference (around 1 KV/cm), in this way the charges produced in the volume are pushed to the electrodes and can't recombine. In one of the cathode is then produced a secondary electron current that can be amplified and measured, this current is then proportional to the amount of couples produced by the beam going through the chamber, then proportional to the amount of charged particle in the beam.
	Even if the thickness is really small it's still an interceptive measurement, because the beam needs to cross material.\\
	The secondary electron current produced by a particle beam at an energy E within the active length l of the chamber is:
	\begin{equation}
		I = \frac{l \cdot \text{I}_{beam}}{W} \cdot \frac{dE}{dx}
	\end{equation}
	where W is the average energy produced by each electron-ion pair (depends on the gas used). The factor $\frac{dE}{dx}$ is an empirically calibrated value, to obtain it a calibration provided by a detector that can provide absolute measurement (like scintillators) is needed.
	The lower detection limit for IC is usually around 1 pA. The upper threshold of the measurement is influenced by the presence of gasses with high electron affinity, such as $O_2$ and $H_2O$, that can capture electrons and then generate errors in the measurement. Furthermore the secondary electron current is usually kept below an upper limit of $\approx 1$ $\mu A$ to avoid saturation effects that can damage the IC.
	\subsubsection{Secondary Electron Monitors}
	To cover an higher current range is possible to use the emission of secondary electron from metallic surfaces. Secondary Electron Monitors (SEMs) is based on this principle. SEMs are generally composed of three metal foils (Aluminum is generally used for his mechanical properties, or Titanium) of $\approx 100 \mu m$ thickness, with the first and third foil biased at around 100 V (to sweep out free electrons) and the second is connected to a sensitive current amplifier. The secondary electrons produced by the beam that has crossed the metallic foil is drifted away by the bias electric field producing a secondary electron current that is amplified and measured. The secondary electron current is given by:
	\begin{equation}
		I= Y \cdot \frac{dE}{\rho dx} \cdot \text{I}_{beam}
	\end{equation}
	where $\rho$ is the density of the foil material and $\frac{dE}{dx}$ describes the energy loss of the beam per unit thickness of the foil. Y is the yield factor which describes the number of secondary electron emitted per unit of energy lost at the surface of the foil, and it's an empirical factor, that needs to be extrapolated by calibration. It's more difficult to calibrate then the IC, because it can change depending on radiation damage or aging effects on the surface of the foil, so in general the resolution is lower. SEM are then an interceptive device that are able to provide measurement of beam current with an higher upper limit than IC at the cost of resolution.
	\subsubsection{Cryogenic Current Comparator}
	The idea of the CCC it's to try to have a detector that is able to provide an absolute intensity measurement in a non-interceptive way, while reaching a resolution in the order of the nA, to extend the range of the beam current transformers and to became an alternative to standard diagnostic for low intensity measurement (while being able to provide a better calibration). The CCC detection method is purely based on the measurement of the azimuthal magnetic field of the ion beam. The standard CCC version is composed of a pickup coil, with an high magnetic permeability torus, equipped with a superconductive shield to eliminate all the magnetic fields that are not the azimuthal one produced by the beam. The magnetic field collected by the torus is then measured using a really sensitive magnetometer, a DC Superconductive QUantum Interference Device (SQUID). With an absolute calibration easily performed with a know current the CCC is able to provide an intensity measurement in a non interceptive way independently of beam position, energy and ion species, both for coasting and bunched beams. \\
	The CCC was first used by I.H. Harvey in 1972 to compare electric current with high resolution \cite{Harvey_CCC}. CCC are actually used with a very high current resolution (1 $fA/ \sqrt{Hz}$) in metrological institutes to confront current with the international standard \cite{Handbook_squid}.  The studies on CCC has then be focused on solving the main issue for the detector, the need for a reliable way to distinguish the magnetic field produced by the beam from all the external electromagnetic perturbation. The solution was found in the addition of a superconductive shield \cite{GrohmannCCC}. The work on CCC kept on, in particular in GSI, with some promising results with the CCC used as a beam diagnostic device in beam lines and storage ring \cite{PetersCCC}. However, the high manufacturing and operating costs correlated to the cryogenic device, together with the high sensitivity to external perturbation, like electromagnetic interference, pressure and temperature fluctuations and mechanical vibration, prevented a wide spread of the CCC as diagnostic device.\\
	A running version of the CCC exists at CERN, at the Antiproton Decelerator (AD), from 2016 \cite{CERNCCC_1}\cite{CERNCCC_2}. The use at the AD in CERN is very rigid, thus allowing to design a small sized cryogenics support system. The smaller and more controlled system achieved a reduced amount of perturbation effects and the CCC is now normally used during AD operation to monitor beam current with improved accuracy in respect to existing diagnostic. \\
	\section{Thesis topic and structure}
	The topic of this thesis is then the development of a CCC detector for FAIR, where the really broad range of beam parameters makes the absolute beam intensity measurement of low-intensity beams even more difficult. An existing prototype of a CCC has been developed and tested for GSI storage ring CRYRING@ESR \cite{DavidThesis}, showing the validity of the CCC has a detector for low intensity measurement in storage ring. This test has, however, shown several limitations of the detector, like non sufficient magnetic screening, low bandwidth and slew rate, that are discussed in the following part of this thesis and needs to be solved or improved. Furthermore the interest of GSI/FAIR for the detector has shifted to transfer lines for the new FAIR facility. The possibility of using the CCC detector as an online device to provide intensity measurement and spill quality analysis is essential for the success of FAIR. To achieve the goal of developing the best possible version of the detector for FAIR, different solution for a CCC detector has been developed and tested. \\
	In \textbf{Chapter 2} it's possible to find a basic introduction to superconductivity and an introduction to the concepts of SQUID, which is the detector on which the functionality of the CCC is based. On \textbf{Chapter 3} a description of the standard CCC detector (on which the prototype used in CRYRING@ESR is based on) it's provided, together with a description of alternative CCC version, based on different superconductive shield \cite{Marsic} geometry or on different core choice \cite{CorelessCCC}, to provide an overview on the different detector analyzed in this thesis work. \textbf{Chapter 4} provides a description of the GSI prototype, in particular on his cryogenic support system. \textbf{Chapter 5} will provide a detailed study of performance of the classical CCC, together with the first test of the detector on transfer lines to measure slow extracted beams and spill quality. \textbf{Chapter 6} provide a description of a coreless version of the CCC with a different shield geometry, together with a description of it's construction process and the results of the device tests. \textbf{Chapter 7} will cover the last version of the CCC used in this thesis work, the Dual Core CCC, a device with two high magnetic permeability core working in parallel and equipped with an axial magnetic shielding. The device test in the laboratory, with a study of noise and performance, can be found in this section, together with the test of the DCCC on the transfer lines and its performances as a detector for spill analysis. Finally, in \textbf{Chapter 8} it will be possible to find a comparison between the classical CCC and the DCCC as device for spill analysis, together with an analysis of which version is the possible best solution for installation as a standard detector in several points of the new FAIR facility, together with a discussion of the next steps needed to finalize and install CCC detectors in FAIR. \\
	 
	\chapter{Superconductivity and SQUID measurement}
	It's possible to measure beam intensities by detecting the azimuthal magnetic field generated by the particle beam. As discussed in the previous chapter, this can be done with high precision. However, for beam current transformers working on this principle, the sensitivity is significantly limited when dealing with low-intensity beams ( $<1 \mu A$) \cite{Fork}. To overcome the BCTs limitations, the Cryogenic Current Comparator can be used. The CCC detection method is also based on the measurement of the field created by the particle beam, that is compared to the field generated by a known calibration current. For this purpose the signal, collected by a cryogenic pick-up unit, is read using an extremely sensitive current detector, the Superconducting QUantum Interference Device (SQUID). This chapter provide a basic introduction to superconductivity and the basic concept of SQUID detectors.\\
	It's important to consider that the magnetic field that the CCC is supposed to measure  have really low intensities. Using the Biot-Savart law (eq.\ref{Biot-Savart}) it's possible to estimate the average beam magnetic field that needs to be detected by the CCC. Assuming a radius of the detector $r \approx 13 $ cm, the magnetic field of a beam around 10 nA is $B \approx 150 \times 10^{-16}\text{ T} = 15\text{ fT}$, that is much lower of external magnetic fields (like the earth one, $\approx$ 50 $\mu$T). An extremely sensitive magnetometer like a SQUID is needed to measure this low intensity magnetic field. The challenge is than to build an effective magnetic shielding able to block all fields that are not the beam azimuthal one, making then possible to take advantage of the full measurement performance of the SQUID\footnote{The superconductive magnetic shield necessary to solve this challenge that has been adopted for the CCC detector will be analyzed in details in the next chapter}.
	\section{Superconductivity}
	To be able to describe the SQUID sensor a basic knowledge of superconductivity is required. This phenomena has been observed first by Heike Kamerlingh Onnes in 1911. He ascertained that the dc resistivity of mercury falls to 0 when it was cooled down to 4.15 K. Later on, other materials that usually had quite bad conductivity at room temperature were find out to have this characteristics of 0 resistivity once cooled down below certain temperature, such as Lead (T$_C$ = 7.2 K), Tin (T$_C$ = 3.7 K) and Niobium (T$_C$ = 9.2 K).\\
	Furthermore, these materials shared other characteristics that have been observed in further experiments. Onnes ascertained that current density in superconductors has a temperature dependent threshold value above which they lose the property of zero electrical resistivity. This threshold value is called Critical Current I$_C$ \cite{Critical_current}.\\
	Further on, Meissner and Ochsenfeld discovered another characteristic of superconductors in 1933: perfect diamagnetism. This means that superconducting surfaces do not allow any magnetic field to penetrate and also expel any field trapped in the surface once cooled down below T$_C$. This observation was better explained by the London brothers (Fritz and Heinz London), who proposed an explanation for the behavior of the charge carriers in superconductors, summarized by the two London equations \cite{London_SC}.\\
	The phenomenological theory developed by the Londons was based on the assumption that electrons in the superconductors were moving as free charged particles under the influence of a uniform external electric field, because the resistance inside superconductors is 0 by definition. Thus, according to the Lorentz force law:
	\begin{equation}
		\overrightarrow{F} = m \frac{d\overrightarrow{v}_s}{dt} = -e\overrightarrow{E} + e \overrightarrow{v}_s \times \overrightarrow{B}
		\label{Lorentz_law}
	\end{equation}
    Assuming, as experiments has shown, that the magnetic field can't penetrate the superconductors, only the electric field component is not 0. Using the definition of current density: $\overrightarrow{j}_s = -\rho_s e \overrightarrow{v}_s$ (where $\rho_s$ is the superconducting charge density) it's possible to write from \ref{Lorentz_law}:
    \begin{equation}
      \frac{d\overrightarrow{j}_s}{dt} = -\rho_s e \frac{d\overrightarrow{v}_s}{dt} = \frac{\rho_s e^2}{m} \overrightarrow{E}
      \label{first_London_eq}
    \end{equation}
    The eq.\ref{first_London_eq} it's known as the First London Equation, that describes the perfect conductivity of a superconductor with supercurrent $\overrightarrow{j}_s$.
    From this equation is possible to extract the second London equation by making the curl of eq.\ref{first_London_eq} and then applying Faraday's law:
    \begin{equation}
    	\frac{d}{dt} \big{(} \bigtriangledown \times \overrightarrow{j}_s + \frac{\rho_s e^2}{m}\overrightarrow{B} \big{)} = 0
    	\label{London_eq_2}
    \end{equation}
    Eq.\ref{London_eq_2} has two solutions: a constant solution and an exponential one. Still considering the observation of Meissner the only possible solution is the exponential one, allowing to formulate than the second (or general) London Equation (using a gauge transformation fixed with the choice of the "London gauge"\cite{Superconductivity}):
    \begin{equation}
    	\overrightarrow{j}_s = -\frac{\rho_s e^2}{m}\overrightarrow{A}_s
    	\label{London_eq_general}
    \end{equation}
    where $\overrightarrow{A}_s$ it's the magnetic vector potential.
    It's then possible to insert eq.\ref{London_eq_general} in the Ampere law to obtain the London penetration depth $\lambda_L$ that provides a description of the Meisner effect. When a magnetic field penetrates the surface of a superconducting material, it is suppressed exponentially depending on the distance z from the surface:
    \begin{equation}
    	|\overrightarrow{B}(z)| = B_0 e^{-\frac{z}{\lambda_l}}
    \end{equation}
    with $\lambda_L$:
    \begin{equation}
    	\lambda_L=\sqrt{\frac{m}{\mu_0 \rho_s e^2}}
    \end{equation}
    The typical value of the London penetration depht is around tens of nm and depends on the properties of each material. To describe the effect it's possible to define currents (called screening currents) that are confined to the surface layer ($\lambda_L$) of the superconductor and that annihilate the magnetic field inside the superconductor as in Fig.(\ref{SC_rings}).\\
    For example, the superconducting material used in the making of the CCC detectors are Lead (Pb $\lambda_L$ = 39 nm) and Niobium (Nb $\lambda_L$ = 37 nm).\\
    	\begin{figure} [H]
    	\centering
    	\includegraphics[width=1\linewidth]{Plot/Chapter_2/Superconductive_rings.png}
    	\caption{\small{Magnetic field lines interaction with a superconductive ring in function of the temperature T in respect to the critical temperature T$_c$. Figure (c) shows the screening currents.}}
    	\label{SC_rings}
    \end{figure}
    However, even though the London theory was able to predict parameters of superconducting materials, it could not explain certain aspects, such as why superconductivity can be broken by excessive magnetic fields. In 1950, Ginzburg and Landau (GL) developed another phenomenological theory that provided a more advanced description of superconductivity \cite{GL_theory}. The basic idea behind the theory is that every second-order transition is usually described by an order parameter that is high in the ordered state and low in the disordered state (for example, the density for evaporation). This parameter for superconductivity is $\phi$, a complex parameter that is correlated to the magnetic field and has a squared value ($|\phi|^2$) that is proportional to the density of super electrons, the defined carrier of charge in superconductors. Using the parameter $\phi$, it is then possible to define the free energy of the system, which allows the basic properties of the material to be described.\\
    To have a microscopic theory that defines superconductivity it was necessary to wait until 1957, when J.Bardeen, L.Cooper and J.R. Schrieffer developed the BCS theory \cite{BCS_theory}, that is able to derive the most important theoretical prediction and can be used to derive both the GL theory and the London equations. Is based on the assumption that in the superconductive state electrons bound in Cooper pairs, a bound state of charge 2e and a common center of mass, with opposite spin and momenta.\\
    Those two theories also predicts the observed flux quantization in superconductors. Through a superconductive loop the magnetic flux can only be varied by discrete values corresponding to integer multiples of the flux quantum $\Phi_0$:
    \begin{equation}
    	\Phi_0 = \frac{h}{2e} = 2.068 \times 10^{-15} Wb\text{  (}1\frac{Wb}{m^2} \equiv 1 \text{T)}
    \end{equation}
    
    \section{Josephson Junction}
    As a consequence of the macroscopic quantum nature of superconductivity is the Josephson effects \cite{Squid_handbook_1}.A Josephson junction is composed of two superconductive electrodes separated by a layer of normal conducting or insulating material. The Cooper pairs of the two superconductors are able to tunnel through the junction thanks to the coupling of the macroscopic wave functions of the two superconductors. This originates a flow of current through the junction that is proportional to the phase difference $\delta$ between the phases of the superconductors wave functions and that is always 2$\pi$ periodic. In case of a SIS junction (Sc-insulation-Sc), the one usually used in the DC-SQUID, the Josephson current is:
    \begin{equation}
    	I_C= I_C\sin\delta
    	\label{First_JJ}
    \end{equation}
    Eq.\ref{First_JJ} is know as the first Josephson equation, where $I_C$ is the critical current of the superconductor. If the current flowing through the junction is higher than $I_C$ will be normally governed by Ohm's law. SQUID junctions are usually small and the current is flowing through the junction homogeneously, with the super current density $j_0= \frac{I_s}{A_j}$ with $A_j$ area of the junction.\\
    If the phase difference $\delta$ is a function of time, then a voltage U is developed across the function:
    \begin{equation}
    	\frac{d\delta}{dt}= \frac{2e}{\hbar}\cdot U = \frac{2\pi}{\Phi_0} U
    	\label{2nd_JJ}
    \end{equation}
    Eq.\ref{2nd_JJ} is known as second Josephson equation. This equation also states that if a voltage U $\neq$ 0 is applied through the junction than $I_S$ oscillates with the Josephson frequency $W_J = \frac{2\pi V}{\Phi_0}$.\\
    High quality junctions, however, presents an hysteretic current-voltage (IV curve) characteristic. Thus, if the current flowing through the junction is higher than $I_C$ an immediate non null voltage appear over the junction, but it doesn't go back to zero when $I_S$ drop below $I_C$. To use the junctions in SQUIDs is essential to remove the hysteresis, and this is generally performed with the addiction of an external shunt resistance. We can describe the behavior of shunted junctions using the RCSJ model, where J is the junction with its critical current $I_C$ and is connected in parallel with it's self capacitance C and the resistance R (that adds a noise source $I_N$). RCSJ junctions can be described with the circuit in Fig.\ref{RCSJ}
    \begin{figure} [H]
    	\centering
    	\includegraphics[width=0.6\linewidth]{Plot/Chapter_2/RCSJ_model}
    	\caption{\small{RCSJ model circuit. U is the potential applied to the junction, $I_s$ the total current through the junction, j is the junction, C the capacity, and R the shunt resistance.}}
    	\label{RCSJ}
    \end{figure}
    It's possible to apply the Kirchhoff's law to the circuit in Fig.\ref{RCSJ} to obtain this equation:
    \begin{equation}
    	\text{C}\dot{U} + \frac{U}{R} + I_C \sin\delta = I + I_N(t)
    	\label{RCSJ_KL}
    \end{equation}
    To simplify the discussion on the results of eq.\ref{RCSJ_KL} the noise component is usually set to 0, and with this assumption is possible to use eq.\ref{2nd_JJ} to obtain a differential equation linking the potential U to the phase difference $\delta$ \cite{Squid_handbook_1}:
    \begin{equation}
    	\frac{\Phi_0}{2\pi}\text{C}\ddot{\delta} + \frac{\Phi_0}{2\pi} \frac{1}{R} \dot{\delta} =I - I\sin\delta = -\frac{2\pi}{\Phi_0}\frac{dU_J}{d\delta}
    	\label{RCSJ_time_law}
    \end{equation}
    where we define the "Tilted Washboard Potential" $U_J$ of the Josephson junction as:
    \begin{equation}
    	U_J \equiv \frac{\Phi_0}{2\pi}\{ I_C(1-\cos\delta) - I\delta \} = E_j\{1-cos\delta-i\delta\}
    \end{equation}
    where $E_J= I_C  \frac{\Phi_0}{2\pi}$ is the energy of the Josephson junction (material dependent) and i is the normalized bias current ($i\equiv\frac{I}{I_C}$) that is a clear indicator of the junction status.
    The evolution of $\delta$ it's the same of the mechanical motion of a particle m subjected to a friction $\xi$ following a potential $U_J$ (Fig.\ref{washboard}). This means that if i<1 the system is trapped in one of the potential minima where it oscillates back and forth, and the time average of the voltage U across the junction is zero. If $i>1$ the local potential minima vanishes, thus the phase difference evolves in time, this dynamic case is hence associated with a finite dc voltage V across the junction which increases with increasing bias current.
    \begin{figure} [H]
    	\centering
    	\includegraphics[width=0.6\linewidth]{Plot/Chapter_2/Washboard.png}
    	\caption{\small{Graphical depiction of $U_J$. Three examples of $U_j$ for different values of i are showed. }}
    	\label{washboard}
    \end{figure}
    One last point to take in consideration for the description of SQUID junctions is the hysteresis of the potential curve, i.e. how much time is needed to the system to reach again a stable point when i changes. Usually SQUIDs uses strongly damped junctions (C $\simeq$ 0), where the inertial term of eq.\ref{RCSJ_time_law} is almost null. Thus, the system is immediately trapped in one of the potential minima when i<1, resulting in a  non-hysteretic I–V characteristics. This kind of junction are usually called RSJ junctions, and they have a normalized time-averaged voltage of:
    \begin{eqnarray}
    	\langle U_J(t)\rangle_t = 0, \qquad \qquad \qquad   \qquad \qquad \qquad \qquad \qquad\forall  I \leq I_c \\
    	\langle U_J(t)\rangle_t = \langle I_CR\frac{(i)^2-1}{i + \cos wt}\rangle_t = I_CR\sqrt{(i)^2 -1}, \qquad \forall  I < I_c
    \end{eqnarray} 
    where $w = \frac{2\pi}{\Phi_0} I_C R \sqrt{i^2-1} $.\\
    In the real world, however, the noise component of the resistance is not 0. The thermal noise, for example, smears out the transition phase and for large noise values the relationship between current and voltage approaches the proportional behavior even for currents below the critical current, and in case of strongly damped junctions this may provoke random voltage spikes in the junction. Other effects can modify the response of junctions from the noise-free theory \cite{Squid_handbook_1}, but those effects can be neglected in the general practical application of Josephson junctions in dc-SQUIDs.
    \section{Theory of DC Squid}
    A Josephson junction can be used as a magnetic flux sensor able to measure field as an integer multiple of $\Phi_0$, as explained above. However, for practical application the sensitivity has to be increased.
    Two Josephson junctions can be arranged in parallel to form a superconducting loop, called dc SQUID (Fig.\ref{dc_SQUID}).
    \begin{figure} [H]
    	\centering
    	\includegraphics[width=1\linewidth]{Plot/Chapter_2/DC_SQUID.png}
    	\caption{\small{Right: Schematic of a simple dc SQUID with two identical Josephson junctions ($\delta_1, \delta_2$) with the same critical current $I_C$. Left: Circuit schematic of the dc SQUID using the RCSJ model. }}
    	\label{dc_SQUID}
    \end{figure}
    A magnetic field applied to the SQUID generates a flux in the superconductive ring: $\Phi_a= \text{A}_{eff} \cdot \text{B}_a$\footnote{To note that the $\text{A}_{eff}$ is the effective area of the SQUID and it's generally quite different from the geometric area}. The flux needs to be quantized and can only be an integer multiple of $\Phi_0$, so an additional screening flux $\Phi_S=J\text{L}_T$ will be generated in the SQUID by the circulating current $J$ going through the ring inductance $\text{L}_T=\text{L}_1+\text{L}_2$. It's then possible to define the total flux through the effective area of the SQUID $\Phi_T = \Phi_S + \Phi_a$.
    The current flowing through each junction can be defined as half of the bias current applied to the SQUID plus (or minus) the circulating one J, and using the RCSJ model (eq.\ref{RCSJ_time_law}) it's possible to write the equation for the current circulating in each of the junction as function of the respective $\delta$:
    \begin{equation}
    	I_k=\frac{I_B}{2} \pm J = I_{0,k}\sin\delta_k + \frac{\Phi_0}{2\pi R_k}\dot{\delta}_k + \frac{\Phi_0}{2\pi} \text{C}_k\ddot{\delta}_k + I_{N,k}, \qquad k=1,2
    	 \label{Squid_system}
    \end{equation}
    where $I_{0,k}$ are the critical current of the two junctions and $ I_{N,k}$ are the two independent Nyquist noise currents.\\ For the first part of the discussion only dc SQUIDs with identical junction and no thermal noise  $ I_{N,k} = 0$ are considered. In this case it's possible to express the $\delta$ of the two junctions as:
    \begin{equation}
    	\delta_2-\delta_1= \frac{2\pi}{\Phi_0}\Phi_T=\frac{2\pi}{\Phi_0}(\text{A}_{eff}\text{B}_a + L_T I_{\Phi}) = \frac{2\pi}{\Phi_0} ( \Phi_a + \frac{\Phi_0}{2 I_C}\beta_LI_{\Phi})
    	\label{delta_SQUID}
    \end{equation}
    where $\beta_L = 2 L_T\frac{I_0}{\Phi_0}$ is the screening parameter and it's an indicator for the influence of the SQUID inductance $L_T$. \\
    If we assume that the SQUID inductance is negligible ($\beta_L \ll 1 $) than eq.\ref{delta_SQUID} can be reduced to $\delta_2 - \delta_1 = 2\pi\Phi_a$. Inserting this equation into the system of eq.\ref{Squid_system} we can obtain a relation for the critical current of the SQUID in function of the critical currents of the single junctions $I_0$:
    \begin{equation}
    	I_C=2I_0\cdot \arrowvert \cos \big( \pi \frac{\Phi_a}{\Phi_0}\big) \arrowvert
    	\label{CH2_eq_critical_current}
    \end{equation}
    In case of identical junction and really low SQUID inductance the critical current of the dc SQUID modulates between 2$I_0$ and 0. Fig.(\ref{SQUID_modulation}) shows the modulation of $I_C$ as a function of the applied flux. If $\beta_L$ is not negligible, the critical current $I_C$ is reduced and for high value of $\beta_L$ $I_C\propto \frac{1}{\beta_L}$ \cite{Squid_handbook_1}.
    \begin{figure} [H]
    	\centering
    	\includegraphics[width=0.6\linewidth]{Plot/Chapter_2/Squid_IC_modulation.png}
    	\caption{\small{ Modulation of SQUID critical current in function of $I_C$. Higher is $\beta_L$, lower is the amplitude of the modulation, for example with $\beta_L$ = 1, the amplitude of the modulation halves.}}
    	\label{SQUID_modulation}
    \end{figure}
    To perform magnetic field measurements with the dc SQUID, it is theoretically sufficient to measure the critical current directly, in particular by increasing the bias current until a dc voltage is generated across the junction. Thus, by knowing the bias voltage, it is possible to determine the critical current. However, this method is not easy and is generally not used. A much simpler mode can be used for overdamped dc SQUID with nonhysteretic current voltage characteristics: by applying a bias current $I_B$ higher than the maximum critical current $I_C$ it's possible to measure the finite voltage drop over the junctions that is directly correlated to the applied magnetic flux $\Phi_a$. In this case the system of eq.\ref{Squid_system} it's identical to the equation of a single junction in the RCSJ model (eq. \ref{RCSJ_time_law}) if a parallel resistance $\frac{R}{2}$ of the two junctions, a capacitance 2C and a critical current $2I_0\cos\pi\Phi_a$ are taken. Thus, the current voltage characteristic if $I > I_C$ is given by:
    \begin{equation}
    	V = \frac{R}{2}\sqrt{I^2 - I_C^2}
    	\label{I_V_char}
    \end{equation}
    Fig. \ref{SQUID_V_modulation} shows on the left the modulation of the voltage as a function of the applied flux, and on the right the I-V characteristics obtained by eq. \ref{I_V_char}. Small changed in the applied flux provoke notable changes in the voltage drop over the junction, making possible to use the SQUID as a magnetometer.
    \begin{figure}[H]
    	\centering
    	\begin{subfigure}[b]{0.45\textwidth}
    		\centering
    		\includegraphics[width=\textwidth]{Plot/Chapter_2/2_SQUID_V_phi.pdf}
    	\end{subfigure}
    	\hfill
    	\begin{subfigure}[b]{0.45\textwidth}
    		\centering
    		\includegraphics[width=\textwidth]{Plot/Chapter_2/2_SQUID_I_V_curve.pdf}
    	\end{subfigure}
    	\caption{Left: Modulation of the normalized voltage of the junction as a function of the applied flux, for different value of the bias current. An orange line shows the point in which the transfer function value is maximum. Right: I-V characteristics of a dc SQUID for two different applied flux.}
    	\label{SQUID_V_modulation}
    \end{figure}
    The sensitivity of the voltage with respect to the changes of the magnetic flux is called the SQUID transfer function $V_{\Phi} = |\frac{dV_{sq}}{d\Phi_a}|$ and depends on which path along the $V-\Phi$ modulation (Fig. \ref{SQUID_V_modulation}) the measurement is performed. Numerical calculation can be performed to found the optimal theoretical working point for every SQUID, but in general for ds SQUID the maximum of the transfer function can be found for $\beta_L \approx 1$ and $\Phi_a \approx 0.25 \Phi_0$. The bias current $I_B$ is usually optimized before the measurement by searching the transfer function maximum. Optimal value of $I_B$ is in theory fixed by the characteristic of the SQUID circuit, in reality it can change depending on the noise floor. With numerical calculation \Cite{Squid_handbook_1} it's possible to show that, in general, in the configuration at which the transfer function is maximized, the low frequency flux noise is at its minimum.
    \subsection{Effects of noise on SQUID performance}
    Only a SQUID model at 0 temperature and with no noise has been described above, in reality every device is characterized by its noise sources, for SQUIDs it is the flux noise, that can be directly linked to a voltage noise. The flux noise is generally dependent on the device bandwidth, and so is in general expressed in term of spectral density, which is the root mean square of the noise magnitude per unit frequency interval. In the case of the SQUID, the spectral density is usually given in $\frac{\Phi_0}{\sqrt{\text{Hz}}}$, while for devices based on SQUIDs, it is provided as a function of the measured value. For example, in the case of the CCC, it is provided in $\frac{\text{nA}}{\sqrt{\text{Hz}}}$, as the CCC is a current measurement device.
    To predict the noise behavior of a SQUID complicate numerical simulations are needed \cite{Squid_handbook_1}, but for the scope of this thesis is enough to discuss some of the main sources.
    \subsubsection{Thermal noise}
    Thermal noise is the dominating one for the SQUID. According to the Nyquist theorem, the junction resistance of the SQUID produce a noise voltage equal to:
    \begin{equation}
    	S_v(f)=4k_BTR
    \end{equation}
    with $k_B$ the Boltzmann constant ($\approx 1.38\times10^{-23}JK^{-1}$), T is the temperature in kelvin and R the resistance value. These elements produces than a voltage noise across the SQUID junctions and a current noise around the SQUID loop, that appear as a white noise (frequency independent). Two factors can be extracted to establish the effects of the thermal noise of the SQUID: the fluctuation threshold inductance $L_f=\frac{(\Phi_0/2\pi)^2}{k_BT}$ and the noise parameter $\varGamma = \frac{k_BT}{E_J}$ \cite{Squid_handbook_1}. If the SQUID inductance $L<L_f$ and $\varGamma \ll 1$ we are then in a situation of small thermal fluctuations, the usual one for dc SQUIDs. $L<L_f$ means that the macroscopic quantum interference are suppressed, and $\varGamma \ll1$ means that the Josephson coupling energy of the junction is greater than the thermal one $k_BT$. At 4.2 K, so for a SQUID operating in a liquid helium bath (usual operation condition for a dc SQUID), $L_f = 1.87 nH$. If $L<L_f$ and $\beta_L\approx1$, generally true for dc SQUID, this means also $\varGamma \ll1$, so the SQUIDs are operated in low thermal fluctuations regime. In this condition the average flux noise of dc SQUID at 4.2 K is $\approx 1 \frac{\mu \Phi_0}{\sqrt{Hz}}$, if the condition of low fluctuations is not fulfilled than this value can blow up to a point in which the SQUID can't work anymore.
    \subsubsection{Effect of asymmetry}
    The theoretical description of the SQUID was done for symmetric system, even if asymmetries can occur, and sometimes this characteristic is even used to improve the performance of the system. In general asymmetries provoke a difference between the characteristics of the critical current modulations for the two junctions. Those effects can be determined with numerical analysis \cite{Squid_handbook_1} of the critical current modulation as a function of the applied flux, in general leading to a reduction of the modulation depth and to asymmetric distortion of the $V-\Phi$ curve itself, an example is the different results for the maximum transfer function on the positive and negative slope of the $V-\Phi$ characteristics.
    \subsubsection{1/f noise}
    Last of the noise sources that can modify the SQUID behavior, is also the less understood one, even for systems at thermal equilibrium \cite{1/f_noise} . Also called "Flicker Noise", it's in general caused by fluctuations of the SQUID parameters, like the ones of the critical current $I_0$, that can be caused by imperfection in the junction material. One electron that is tunneling through the junction can get trapped in a defect, modifying locally the critical current. One additional source of this noise can be trapped flux in the SQUID body, which can originate, with its motion, a current flowing through the junction.
    In general, however, for low temperature Tc SQUID operated at 4.2 K the contribute of 1/f noise is lower than the effect of thermal fluctuations. It can became a predominant effect in device used for measurement of really low frequency signal.
    \section{SQUID readout electronics}
    The dc SQUID is a flux-to-voltage converter with a non linear, periodic, $V-\Phi_a$ characteristic (Fig.\ref{SQUID_readout}) where V is the voltage across the SQUID and $\Phi_a$ is the applied flux. Peak to peak voltage swing are usually in the order of tens of $\mu V$ with a period equal to a flux quantum $\Phi_0$. To measure this values with the highest resolution, and to maximize the SQUID performance a readout scheme needs to be used (Fig.\ref{SQUID_readout}). In this section it's then described the electronics needed to read out the information of a SQUID used as a flux sensor, so to quantitatively determine the amount of applied flux signal through the SQUID loop.
    \begin{figure}[h]
    	\centering
    	\begin{subfigure}[b]{0.49\textwidth}
    		\centering
    		\includegraphics[width=\textwidth]{Plot/Chapter_2/Characteristics.png}
    	\end{subfigure}
    	\hfill
    	\begin{subfigure}[b]{0.5\textwidth}
    		\centering
    		\includegraphics[width=\textwidth]{Plot/Chapter_2/2_SQUID_FLL_schematic.pdf}
    	\end{subfigure}
    	\caption{Left: $V_\Phi$ Characteristics of a dc SQUID with working point W. Right: Schematic of a typical dc SQUID (circular elements with the two crosses marking the junctions) direct-coupled FLL readout circuit. The FLL feedback scheme linearizes the output of the SQUID relative to the applied flux.}
    	\label{SQUID_readout}
    \end{figure}
    \subsection{Flux Locked Loop (FLL)}
    In principle, the SQUID can be operated in a small signal mode around the working point W (typically located in the steepest part of the $V_\Phi$ curve), where a small change in $\delta\Phi_a$ produces a proportional change in $\delta V = V_{\Phi} \delta\Phi_a$, where $V_{\Phi} = \frac{dV}{d\Phi_a}$ is the transfer coefficient at W. However, the proportionality between flux and voltage is maintained only for small values of $\delta\Phi_a$. To keep the SQUID as linear as possible, it's essential to keep the signal flux range $\delta_{pp}$ smaller than $\Phi_{lin} = \frac{V_{pp}}{|V_{\Phi}|} \lesssim \frac{\Phi_0}{\pi}$ \cite{Squid_handbook_1}, but this value is smaller than typical noise amplitudes, so the small signal mode is never used.
    
    To maintain the linear behavior of the SQUID, a flux-locked loop (FLL) (Fig. \ref{SQUID_readout}) can be used. In FLL mode, the SQUID is biased at the working point W, similar to the small signal readout mode. The deviation of the SQUID voltage \(V\) from the working point voltage \(V_B\) is amplified, integrated, and fed back to the SQUID through a feedback resistor \(R_F\) and a feedback coil, which is magnetically coupled to the SQUID via a mutual inductance \(M_f\). The local magnetic flux induced by the feedback current $I_{fb}$ cancels out any deviation from the operating flux $\Phi_a$, thus maintaining the SQUID at its optimal working point. It's then possible to convert the feedback current $I_{fb}$ into a measurable voltage $V_{FLL}$ using the feedback resistor $R_f$ to obtain a measurement that is directly proportional to the external flux. Furthermore, in FLL mode, the transfer coefficient is not dependent on the SQUID working point but only on constant circuit parameters (feedback resistance $R_f$ and feedback coil mutual inductance $M_f$). The dynamic range of the SQUID can easily be increased by reducing $R_f$ and increasing $M_f$, and in general, it's limited by the ADC used. Extremely precise ADCs should be used to effectively observe the noise flux of a SQUID in FLL mode and the remaining effects of non-linearity in the SQUID, so in general, it is the ADC that limits the SQUID performance. It is possible to reduce the demands and increase the range of the SQUID beyond the single flux quanta $\Phi_0$ by exploiting the periodicity of the $V-\Phi_a$ characteristic, by setting the maximum input range of the ADC to $\pm \Phi_0$. It is then possible to reset the integrator each time the flux exceeds its range and count every reset. With this method, it is possible to obtain high ranges (greater than $10^4$ $\Phi_0$ \cite{SQUID_dinamic_range}), however, it is essential to keep the integrator reset time as low as possible to allow a high system slew rate, thus allowing the measurement range of the SQUID to be defined almost only by the value of the feedback resistor $R_f$ and the mutual inductance $M_f$. \\
    In the FLL mode the overall flux density noise can then be wrote as:
    \begin{equation}
    	S_{\Phi,FLL} = S_{\Phi} + \frac{S_{V,AMP}}{V_\phi^2} + S_{I,AMP}R^2_{dyn}
    	\label{CH2_SQUID_noise}
    \end{equation}
    where $S_{\Phi}$ is the intrinsic flux noise density of the SQUID, $S_{V,AMP}$ is the pre-amplifier voltage noise density and $S_{I,AMP}$ the current noise of it. 
    $S_{\Phi}$ components were described in the previous section, however it's important to say that in the goal of using the SQUID as a beam diagnostic device these components are in general negligible in respect to external perturbation.
    The current noise term $S_{I,AMP}$ is also generally negligible due to the small dynamic resistance around the SQUID working point. It can became important at low frequencies for semiconductors based amplifiers. \\
    It's possible to estimate the dynamic performance of a SQUID in the FLL mode using the simplified model of Fig.\ref{SQUID_readout}. In general the bandwidth of the feedback loop is typically adjusted by changing the combined gain bandwidth product $f_{GBP}$ of the amplifier and of the integrator, that is directly linked to the unity-gain frequency $f_1$ of the system. In real SQUID systems, though, there are additional factors that set limits on the maximum usable bandwidth \cite{SQUID_book_2}.
    At start it's essential to add a delay element that sums up all the possible source of delays in the system, dominated by the delay time $t_d$, which is the signal delay between the SQUID and the readout electronics. If $f_1t_d \ll 0.08$ then the signal delay it's negligible and the SQUID behaves as a first order low pass with a 3 dB bandwidth $f_{3dB} = f_1$. If $f_1t_d$ increases then phase lag appears, together with a resonance peak in the transfer function close to the cut-off frequency that strongly reduces the stability of the system. The maximum 3 dB bandwidth $f_{FLL,max}$ of the SQUID system is achievable for $f_{1,max}t_d = \frac{1}{4\pi} = 0.08$ \cite{Squid_handbook_1} and it's equal to:
    \begin{equation}
    	f_{FLL,max} = 2.25 f_1 = \frac{0.18}{t_d}
    	\label{CH2_maxf}
    \end{equation}
    It's essential to keep the distance between the FLL electronics and the SQUID as short as possible to reduce \(t_d\) and increase the maximum bandwidth of the system, particularly the slew rate. The slew rate is defined as \(\dot{\Phi_f} = \left| d\Phi_f/dt \right|_{\text{max}}\) and represents the maximum signal slope that the SQUID can measure. It is generally determined by applying a sinusoidal wave with increasing slope until the system becomes unstable.\footnote{Further explanation on the measurement of slew rate and why it is essential for the CCC will be provided in the next sections. Optimization of slew rate is a core part of the work performed to use the CCC as a detector for transfer lines.} At high signal frequencies, the slew rate can be estimated as \cite{Squid_handbook_1}:
    \begin{equation}
    	 \dot{\Phi_f} = 2\pi f_1 \delta\Phi \lesssim \Phi_0 f_1 
    	 \label{Theoretical_slewrate}
    \end{equation}
    Since it is crucial for a system to function correctly, the slew rate needs to be as high as possible, so \(t_d\) needs to be minimized.
    Typical values of the loop delay are around $t_d\sim$ 10 ns \cite{Drung_article}, corresponding to a general distance of around 1 m between the FLL electronics and the SQUID\footnote{SQUID electronics it's usually operated at room temperature, so it can't stay close to the SQUID that is a cryogenic device. This is why it needs to be located outside of the cryostat}. In the SQUID system used in GSI the cable length is longer (1.9 meter), corresponding to a loop delay time $t_d\sim$ 20 ns and maximum system bandwidth of around 9 MHz (general maximum bandwidth is $\sim 20$ MHz). It's also important to consider the finite bandwidth of the amplifier which is part of the FLL readout that introduces an additional signal delay $t_{d,AMP}$ that leads to a further reduction of the maximum system bandwidth by a factor around 0.8 \cite{SQUID_book_2}. \\
    Noise components also reduces the effective system bandwidth. The broadband noise $S^b_{\Phi,FLL}$ consists of the voltage noise of the SQUID $S_{V,SQ}$ and of the one of the amplifier $S_{V,AMP}$ that are amplified, integrated and feed back to the SQUID.
    Excessive broadband noise will reduce the transfer function $V_\Phi$, the maximum slew rate and affect the stability of the feedback. The maximum 3 dB bandwidth of the system, without deforming the output signal, can be found with  \cite{Squid_handbook_1} \cite{SQUID_book_2}:
    \begin{equation}
    	f_{FLL,max} \simeq \frac{0.0044 (2\delta V)^2}{S_{V,AMP} + S_{V,SQ} }
    	\label{f_FLLmax}
    \end{equation}
    Where $\delta V$ it's called the usable voltage range and it's the difference between the amplitude V of the $V-\Phi$ modulation of the SQUID and the bias voltage $V_b$ at the working point (Fig.\ref{SQUID_readout}). To achieve the highest bandwidth it's essential to increase the amplitude of the modulation while keeping the components noise as low as possible. In the case it's realistic to estimate that the bandwidth of the SQUID system is ultimately limited only by the loop delay.
    For the SQUID setup in GSI a usable voltage range around $2\delta V \geq 1 mV$, meaning that the maximum bandwidth of the system is primarily limited by the loop delay \footnote{Calculation shows that $\delta V = 30 \mu V$ is enough to consider the effects of the broadband noise negligible for the estimation of the bandwidth \cite{Squid_handbook_1}}.
    However the broadband noise is a not negligible detail of a SQUID system, in fact it is also connected to the maximum achievable slew rate. Approximating with a linear function the $V-\Phi$ modulation of the SQUID we can obtain a relation for the slew rate \cite{SQUID_book_2}:
    \begin{equation}
    	\dot{\Phi}_{f,max} = 2\pi f_{FLL} (\delta\Phi - S^{peak}_{\Phi, FLL}(f_{FLL})) 
    \end{equation}
    Where it's possible to write $ S^{peak}_{\Phi, FLL}(f_{FLL})$ as:
    \begin{equation}
    	 S^{peak}_{\Phi, FLL}(f_{FLL}) = 4  S^{rms}_{\Phi, FLL}(f_{FLL}) = \frac{4}{V_{\Phi}}\sqrt{\frac{\pi}{2} f_{FLL} (S_{V,AMP} + S_{V,SQ})}
    \end{equation}
    where $f_{FLL}$ is the 3 dB bandwidth of the system after considering the effect of the broadband noise and with the linear flux range $\delta\Phi$. It's possible to note that a reduction of the transfer function $V_\Phi$ and of the usable voltage range $\delta V$ (eq.\ref{f_FLLmax}) leads to an immediate reduction of the slew rate of the system. With a loop delay $t_d$ the maximum achievable slew rate can be then found using eq.\ref{Theoretical_slewrate}:
    \begin{equation}
	    \dot{\Phi}_{f,max}=\frac{\delta\Phi}{2t_d} \lesssim \frac{\Phi_0}{4\pi t_d}
    \end{equation}
    A system with loop delay around 10 ns should then have a maximum slew rate of 8 $\Phi_0/\mu$s\footnote{For the system used in GSI, with a loop delay time around 20 ns, the maximum slew rate of the system is halved, so 4 $\Phi_0/\mu$s}, independent from the transfer function. In reality maximum slew rate achievable with direct read-out schemes is lower, typically between 1 and 10 $\Phi_0/\mu$s \cite{Squid_handbook_1}.
    \\
    \section{SQUID radiation hardness}
    Typical uses of SQUID detectors are far away from the possible extreme conditions that can be found in accelerator facilities. With the goal of understanding effect of moderate radiation on SQUID detectors, test at the CERN High Energy Accelerator mixed-field (CHARM) irradiation facility \cite{CHARM} has been performed. Two manufacturers of dc SQUIDs (Magnicon GmbH and Supracon AG) provided the sensors for this test (free of charge). \\
    The amount of radiation that the SQUID sensor suffers is entirely dependent on the installation location. While for storage rings (such as at the antiproton decelerator at CERN and in CRYRING@GSI), the amount of radiation the detector experiences is negligible, this is not the case for the transfer lines of FAIR. The expected yearly dose in the transfer line after SIS 18\footnote{First installation location of CCC detector in the future FAIR facility, both spatially (the closer one to the older GSI facility) and temporally (First to be installed in the official FAIR timetable)}, before the injection in SIS 100, is estimated to be:
    \begin{equation}
    	D_{sq} \leq 10 Gy/year
    \end{equation}
    This radiation dose is moderately low, the expected losses in transfer lines are theoretically almost negligible. In best cases a SQUID lifetimes can be estimated to around 10 years, if continually operated \footnote{A detailed description of effects o aging of the SQUID sensors has never been performed. Next part of this thesis will provide an estimation of aging effect of a single SQUID detector used in GSI radial prototype for around 5/6 years, but never in continuous operation.}, adding up to a maximum of some hundreds of Gray. An irradiation target of 1 kGy was set for the test in the CHARM facility to simulate the total exposure of the detector to radiation and stress test the SQUIDs.
    The sensors used where both using Niobium and aluminum oxide Josephson junctions\footnote{Magnicon Sensors: type C6XL1 and C6XL1W. Supracon sensors: High sensitivity CN2 SQUIDs} and installed on a fiberglass carrier with outer dimension of $17\times7.2\times3$ mm. The sample were installed at the CHARM facilities and irradiated for a total of 3 weeks. In this time period the samples were installed behind a copper target which was hit by a proton beam with an energy of 24 GeV. The accumulated radiation dose can be found in Tab.\ref{Dose_SQUID}:
    \begin{table}[H]
    	\centering
    	\begin{tabular}{c|c}
    		\hline
    		High Energy Hadrons fluence ($> 20$ MeV) [cm$^{-2}$] & $6.30\times 10^{12}$ \\ \hline
    		1 MeV n equivalence [cm$^{-2}$] & $1.02\times 10^{13}$ \\ \hline
    		Total numbers of particles on target & $3.81\times 10^{16}$ \\ \hline
    		Accumulated Dose [Gy] & 1368 \\ \hline
    	\end{tabular}
    	\caption{Accumulated radiation dose of the SQUID sensors during the 3 weeks of the irradiation test at the CHARM facility. All values have an uncertainty of around 35\% \cite{Cern_report} }
    	\label{Dose_SQUID}
    \end{table}
    The flux of different particles with their particular energies can be found in Fig.\ref{CHARM_flux}.\\
    After the irradiation test the SQUID sensors were sent back to the providing companies to be tested with the goal of searching any deterioration in their performance:
    \begin{itemize}
    	\item \textbf{Magnicon}: for both SQUID no significance performance reduction could be observed.
    	\item \textbf{Supracon}: One of the two sensor showed a significant reduction of the amplitude of the $V-\Phi_a$ modulation ($42 \%$) which negatively effects the SQUID transfer function $V_\Phi$. However a larger bias current was needed to reach the stable working point (maximum voltage modulation), thus indicating an electrostatic damage rather than a radiation effect. The other SQUID shows no performance degradation.
    \end{itemize}
     \begin{figure} [H]
    	\centering
    	\includegraphics[width=0.9\linewidth]{Plot/Chapter_2/CHARM_spectra.png}
    	\caption{\small{Spectrum of particle flux at different energies at the location of the SQUID sensor during the irradiation test at the CHARM test facility \cite{Cern_report}.}}
    	\label{CHARM_flux}
    \end{figure}
   While the irradiation of Supracon SQUIDs should be repeated to get a clearer picture, the results achieved, especially for the SQUIDs provided by Magnicon GmbH, are sufficient to estimate that SQUIDs are largely unaffected by moderate doses of ionizing radiation. Furthermore, the results of the test at CHARM are consistent with similar results for Josephson junctions made of different materials \cite{Irradiation_test_squid}. \\
   However, this test is not enough to ensure a reliable operation of a SQUID based detector in an environment with high dose of radiation. During this test the SQUID was not in its superconductive state and there were no applied voltage or flux. During the system operation different effects, like single-events effects, could lead to damage to the detector or to anomalies in the measured signal. Furthermore the standard FLL electronics is composed of electrical parts (like MOSFET transistor) that has typically low radiation hardness. The radiation test will then need to be repeated to collect further information on the effect on a running system and on its electronics, that will probably need to be shielded or positioned away from the area were an high dose of radiation is expected\footnote{Considering the strong effect of the delay time introduced by the distance between SQUID and FLL electronics, a estimable possible maximum for the distance is 5 m (achievable maximum bandwidth of 1 MHZ)}.\\
   In case of the planned installation at FAIR there is a spot in which the amount of expected radiation is particularly high, at 14 m from the extraction of SIS 100. An estimation of the radiation protection department of GSI gives an upper limit that is based on the maximum expected beam losses (usually the measured during operations are order of magnitudes lower). With the assumed beam losses of $3 \times 10^{10} U^{28+}$ per second during slow extraction with an energy of 2.7 GeV/u an accumulated doses of 300 KGy/y is expected for a device in operation for 250 days \cite{GSI_radiation}. At the proposed location of the CCC the expected dose is then:
   \begin{equation}
   	D^{FAIR}_sq \leq \frac{1}{d^2} D_{extraction} \approx 1.5 kGy/y
   \end{equation}
    Even though the SQUID detectors used in the CCC have not yet been tested under high levels of radiation, literature indicates that Josephson junctions can endure such conditions \cite{Irradiation_test_squid}. However, given the elevated radiation levels, it is essential to implement additional protective measures for the sensitive electronic components critical to the SQUID's operation.
    \chapter{The Cryogenic Current Comparator}
    The Cryogenic Current Comparator (CCC) it's the core part of this PhD work. This section will provide a detailed analysis of CCC detectors and it will work as an introduction for further discussion in the next chapters. Starting with the description of the working principle of CCC detector, the three versions of CCC tested during this thesis will be accurately described.
    \section{CCC working principle}
    The CCC working principle rests on Ampere's law and on the perfect diamagnetism of the superconductor in the Meissner state (also called Meissner effect). Two currents ($I_1,I_2$) flows through a superconducting tube with wall thickness several times higher than the London penetration depth. Each current generates a magnetic field (Biot-Savart law eq.\ref{Biot-Savart}) and both induces a superconducting screening current on the surface of the tube that nullify the magnetic flux density inside the tube. By applying Ampere's law over the closed contour in Fig.\ref{CH3_Meissner_effect} it's possible to obtain:
    \begin{equation}
    	\oint_a Bdl = 0 = \mu_0 (I_1-I_2-I)
    \end{equation}
    which yields to the equality of currents I = $I_1-I_2$.
     \begin{figure} [H]
    	\centering
    	\includegraphics[width=0.7\linewidth]{Plot/Chapter_3/Superconducting_tube.png}
    	\caption{\small{Schematic depiction of the Meissner effect. Two currents going through the superconductive tube generates a screening current I = $I_1-I_2$.}}
    	\label{CH3_Meissner_effect}
    \end{figure}
    This equality is valid independently of the position of the currents going through the tube, even in the case of different paths of the wires inside the tube. The current density distribution of \( I \) is always homogeneous, allowing the CCC to achieve very high resolution \cite{Handbook_squid} \cite{GROHMANN_CCC}. The magnetic field \( \Phi \) generated by the screening current, which is proportional to the difference between the two currents, can then be collected and measured using a SQUID.
    \subsection{CCC for beam current measurements}
    As described before, the CCC is able to detect particle beams by comparing the current generate by the moving charges of the ion beam $I_{beam}$ to a well defined calibration current $I_{cal}$. The measurement principle of the CCC for beam intensity measurement is shown in Fig.\ref{CH3_Schematic_CCC}.
    \begin{figure} [H]
    	\centering
    	\includegraphics[width=0.7\linewidth]{Plot/Chapter_3/Schematic_CCC_general.png}
    	\caption{\small{The schematic of the classical CCC (radial superconducting shield geometry) used as a beam diagnostic tool illustrates the device's design. The superconducting shield's rotational axial symmetry significantly attenuates non-symmetric magnetic field components (e.g., \( B_{\phi} \)). The magnetic flux generated by the ion beam \( I_a \) inside the pick-up inductance \( L_p \) is detected by the SQUID sensor and compared to a reference calibration signal \( I_{\text{cal}} \).}}
    	\label{CH3_Schematic_CCC}
    \end{figure}
    $I_{\text{cal}}$ is generally used only to extract the calibration factor $k_{\text{cal}}$ [nA/mV] that links the current passing through the CCC to the measurement provided by the SQUID. In this way, it is not necessary to match $I_{\text{cal}}$ to $I_{\text{beam}}$, allowing it to be kept constant or even omitted during the measurement. $I_{\text{cal}}$ is typically provided through a normal conducting wire that runs parallel to the ion beam, thanks to a voltage generator connected to the wire\footnote{More details on the actual calibration system used for the GSI CCC will be provided in Chapter 4}. The spatial position of the wire is a negligible factor due to the properties of the superconducting tube.\\
    A toroidal superconducting pick-up (shown in yellow in Fig.\ref{CH3_Schematic_CCC}) with inductance $L_p$ is used to couple the magnetic field to a DC SQUID using the input inductance $L_i$.\\
    The CCC can also be used to detect DC currents, as all the components used to carry the signal from the pick-up to the SQUID are superconductive, thus having no resistive losses that limit a normally conductive current transformer.\\
    It is essential to provide a way to shield the superconductive pick-up coil from external perturbations. Thanks to the geometry of the pick-up, the highest value of the inductance is for azimuthal field components \( B_{\phi} \hat{e}_{\phi} \), which already acts as a first filter against external perturbations. However, it is actually impossible to build a perfectly selective pick-up, and it is essential to consider the huge amount of external perturbations present in particle accelerators, which have a wide spectrum of intensities, up to several Tesla, many orders of magnitude higher than the magnetic field of the beam. A superconducting shield is then necessary. The superconducting tube that encloses the pick-up coil is then extended to form a passive shield against external perturbation\footnote{The superconducting shield can be built out with two different geometries, further details can be found in the following section of this chapter.}. \\
     To shield the SQUID a similar method is used, even if in a smaller scale. The SQUID and eventual additional part of it's cryogenic circuit are enclosed in a superconducting shielding cartridge that is mounted outside of the superconducting shield. The SQUID is then connected to the pick-up through really thin superconducting wires that go trough the shield thanks to miniature holes in the outer surface. Inside the cartridge the SQUID can again be enclosed in an additional superconducting cylinder Fig.\ref{CH3_SQUID_cartridge}.
    \begin{figure} [H]
    	\centering
    	\includegraphics[width=0.9\linewidth]{Plot/Chapter_3/SQUID_cartridge.png}
    	\caption{\small{SQUID cartridge of the FAIR-Nb-CCC-xD holding the SQUID and it's flux transformer. The SQUID itself is allocated inside an additional shielding cylinder on the left. On the right it's possible to see the connection point of of the pick-up coil with the cartridge itself.}}
    	\label{CH3_SQUID_cartridge}
    \end{figure}
    	There is only one small opening for external magnetic fields, which is used for the connection to the readout electronics of the SQUID. Additionally, the input inductance of the SQUID, \(L_{\text{sq}}\), is typically designed as a gradiometer. This design effectively filters out homogeneous magnetic field components, reducing the influence of distant perturbation sources. These combined measures greatly reduce the likelihood of direct magnetic field coupling to the SQUID. Consequently, any signal reaching the SQUID is introduced via the pick-up circuit.\\
        The CCC is composed of these three main components. The pick-up collects the magnetic field of the beam, while external perturbations are strongly attenuated by the superconducting shield, feeding the signal to the SQUID, which provides an output proportional to the beam current. The CCC is installed to enclose the beam tube and is cooled by a cryogenic support system that must be adapted to the specific requirements of the installation site\footnote{The system used at GSI it's described in Chapter 4}. The SQUID operates in the FLL mode described in section 2, measuring the applied magnetic flux \( \Phi_{\text{beam}} \) of the beam. The FLL mode requires a reference flux for absolute current measurement, and for accelerator operation, the natural reference is the zero-beam flux, which represents the background flux when no beam is circulating in the beamline. In any case, the CCC can monitor changes in beam current independently of the absolute current present when the SQUID enters FLL mode.
        \section{Superconducting shield topology}
        To use a CCC detector to measure beam current, it is essential to shield the pick-up coil and the SQUID from external perturbations. Significant efforts have been made to design the best possible shielding, capable of attenuating most magnetic field components that do not carry information about the beam current. Thanks to the properties of superconductors, particularly perfect diamagnetism, which prevents any field above the critical field from penetrating the superconductor beyond the London penetration depth, superconducting housing serves as excellent magnetic shielding. In 1976, Grohmann et al \cite{GROHMANN_CCC} described the field attenuation of a shield design that is based of an axial stack of multiple ring cavities as in Fig.\ref{CH3_shield_topology}. The stack of these two cavities are effectively forming two coaxial tubes that are connected on one end were they enclose the pick-up winding ($L_p$) of the detector. The overall screening factor of the shield is defined by the effective length ($l_e$) of the path within the coaxial tubes between the entrance and the position of the pick-up windings and by their radial dimension. The length $l_e$ is generally increased by additional ring disks that intrude the space between the tubes. The disks are then connected to the outer tube and to the inner tube alternately creating a meander path. Between each couple of disks is also placed an electrical insulation (generally glass reinforced plastic (commonly known as fiberglass or GRP)) to avoid shortcuts that can strongly reduce the effective length.
        \begin{figure}[H]
        	\centering
        	\begin{subfigure}[b]{0.45\textwidth}
        		\centering
        		\includegraphics[width=\textwidth]{Plot/Chapter_3/3_CCC_radial_topology_image.png}
        	\end{subfigure}
        	\hfill
        	\begin{subfigure}[b]{0.45\textwidth}
        		\centering
        		\includegraphics[width=\textwidth]{Plot/Chapter_3/3_CCC_axial_topology_shieldpickup2.pdf}
        	\end{subfigure}
        	\caption{ Cross section of two different typologies of superconducting shield. \\ \textbf{Left}: Radial superconducting shield: Radial meanders are used to shield the pick-up coil (used in the radial CCC FAIR-Nb-CCC-xD). \\ \textbf{Right}: Axial superconducting shield: Co-axial meanders are used to shield the pick-up coil (used in the coreless CCC and in the double core CCC (DCCC) for FAIR).}
        	\label{CH3_shield_topology}
        \end{figure}
	    It's possible to use an analytical model to estimate the properties of a superconducting shield. By solving the Laplace equation for a magnetic scalar potential ($\bigtriangledown^2V$ = 0) for this geometry by assuming ideal diamagnetism of the superconducting material\footnote{Mathematically expressed by fixing null the magnetic field perpendicular to the surface of the superconductor $B_{\bot,sc} = 0$} and introducing a current flowing through the CCC, it is possible to show that the azimuthal field component of the beam current $\overrightarrow{B}_\varphi(r)= \mu_0I/2\pi r\hat{e}_\varphi$ can reach the pick-up volume unaltered \cite{GROHMANN_CCC} \cite{GROHMANN_shield}. It's possible to obtain the same result also through geometrical simulations, that also allows to get more information on effects of particular components of the CCC on the shielding properties, for further details see \cite{DEGERSEM_shield}.
        Furthermore the simulations shows that any component of $B_\Phi$ that is dependent on the radial position of the current inside the CCC is strongly attenuated, which strengthen the assumption that the CCC measurement is independent from the position of the beam. If the shielding factor is sufficiently high it's possible to assume that the CCC measurement is also independent from external field components, with the intensity of attenuation directly dependent on the effective length of the shield $l_e$. Geometrical simulations can be performed to estimate $l_e$, that is mainly dependent by geometrical factors of the shield like the number of meanders, the height and a bigger difference between inner and outer diameter of the detector \cite{DEGERSEM_shield} \cite{Marsic_shield}. The CCC shield is designed to provide the best possible \( l_e \) within the limits of the required dimensions and constructive feasibility.\\
        The external magnetic field that is less attenuated by the superconductive shield is the dipole, so in general the attenuation factor is provided for the dipole field. From the analytic analysis and simulation in \cite{GROHMANN_shield} \cite{DEGERSEM_shield}, it's possible to express the dipole attenuation factor A, which is the ratio between the magnetic flux densities at the entrance and at the exit of a single ring cavity ($A= |B_{out}|/|B_{in}|$), as:
        \begin{equation}
        	A = \left(\frac{R_{out}}{R_{in}}\right)^2 = \left(\frac{h_{gap}}{R_{in}} + 1\right) ^2 = \left(\frac{l_e}{2R_{in}} + 1\right)^2 
        	\label{CH3_dipole_shielding_factor}
        \end{equation}
        with $R_{in},R_{out}$ the inner and outer diameter of the volume enclosed by the shield. The radial length of the shielding volume is $h_{gap}= R_{out}-R_{in}$. The radial shield topology has been used for the first prototype tested during this thesis work, the FAIR-Nb-CCC-xD.\\
        Further analysis on the topic of superconductive shield has shown that another shield topology is possible. The coaxial shield structure allows to achieve a much higher magnetic screening factor in respect to radial topology if the detector has a length higher than the radial thickness $h_{gap}$ \cite{Marsic}\cite{CorelessCCC}. In the axial design, instead of using stacks of disks to increase the effective length, the meanders are connected on alternating ends so they can be stacked radially. Once more the layers are interloped with electrical insulation to avoid shortcut that can reduce the effective length. The requested dimension for FAIR CCC detectors are fulfilling this condition, thanks to the request of a large inner diameter to accommodate the beamline and a small radial length to reduce the overall size of the setup.\\
        The attenuation factor for magnetic fields that are perpendicular to the axis of the CCC for one coaxial layer with radius $r_i$ and length l can be estimated as \cite{Marsic} \cite{GROHMANN_shield}:
        \begin{equation}
        	A_{axial} = e^{(\frac{l}{r_i})}
        \end{equation}
        For multiple layers the total screening factor can be calculated as:
        \begin{equation}
        	A_{axial} = \sum_i  e^{(\frac{l}{r_i})} = e^{(\sum_i\frac{l}{r_i})} \approx e^{\frac{l_e}{r_{avg}}}
        \end{equation}
        For the axial topology, it is possible to differentiate the shield by placing the meander structure on the outside surface of the pick-up coil or on the inner surface. Simulations show that a higher magnetic screening factor is achievable when the meander structure is placed on the outside surface of the pick-up coil. However, the axial CCCs used in this thesis work are both built with the meander structure on the inner surface because the manufacturing is much easier and the difference in the screening factor is, in the end, rather low.
        
        \section{Radial CCC for FAIR} 
        In this section the FAIR-Nb-CCC-xD, the prototype of CCC built for FAIR using a radial shield, will be described, in particular the estimation of the shielding factor, the construction process, the high magnetic-permeability core used and the SQUID circuit.
        \subsection{Shielding factor estimation}
        \begin{figure} [H]
        	\centering
        	\includegraphics[width=0.9\linewidth]{Plot/Chapter_3/CH3_radial_shield_simulation.png}
        	\caption{\small{Simulation of the shielding factor of the superconducting shield in the radial topology against a magnetic dipole field for different outer dimensions (outer radius \( R_{\text{out}} \), length \( l \)) and the resulting number of pairs of shielding disks. Courtesy of N. Marsic (TU Darmstadt) \cite{Marsic_shield}.
        	}}
        	\label{CH3_radial_shield_simulation}
        \end{figure}
        As described in the previous section, it is possible to perform geometrical simulations to estimate the shielding factor of a superconducting shield. TU Darmstadt performed simulations for the two different shield topologies that the CCC could use, with the goal of maximizing the shielding factor for the detector dimensions required for FAIR \cite{Marsic} \cite{DEGERSEM_shield} \cite{Marsic_shield}.\\
        In those simulations the inner diameter is fixed at $D_{in}=240 mm$, while the dipole screening factor $A$ is plotted as a function of the axial length $l$ of the outer shield and of the external radius $R_{out}$. These three dimension are enough to define the total volume of the detector. By keeping constant the area that needs to be dedicated to the pick-up ( cross section kept constant at 60 $cm^2$) it's possible to extract the number of shielding layers that can fit in the detector, as shown in Fig.\ref{CH3_radial_shield_simulation}. The chosen geometry for the prototype is similar to the one at the bottom right corner of Fig.\ref{CH3_radial_shield_simulation}, with an outer radius $R_{out} = 175 mm$ and an axial length $l=210 mm$. However the requested dimension for FAIR changed from the one used in the simulation, so the detector has an inner diameter $D_{in}=250 mm$, larger that the one used in the simulation, resulting in an effective shielding slightly lower than the estimated one. To achieve a shielding factor comparable to the one of older CCC versions ($\approxeq 75$ dB)  a stacking of several disks is needed ($\geq 8$).
        \subsection{Design and construction}
        The FAIR-nB-CCC-xD has been built in Jena in 2017 to serve as a prototype for CCC to be used in storage ring at GSI \cite{DavidThesis}, and it's then be adapted to be used as a spill analysis detector for transfer lines with modification that will be discussed in next chapters. Anyway, the main structure of the detector it's the same and will be discussed briefly in the following section. Fig.\ref{CH3_radial_CCC} shows the full view of the detector and a schematic with the definitive dimensions of this prototype.
        \begin{figure}[H]
        	\centering
        	\begin{subfigure}[b]{0.4\textwidth}
        		\centering
        		\includegraphics[width=\textwidth]{Plot/Chapter_3/CH3_FAIR-nB-CCC-xD.png}
        	\end{subfigure}
        	\hfill
        	\begin{subfigure}[b]{0.55\textwidth}
        		\centering
        		\includegraphics[width=\textwidth]{Plot/Chapter_3/CH3_FAIR-nB-CCC-xD_halfview.png}
        	\end{subfigure}
        	\caption{\small{ Left: FAIR-nB-CCC-xD before the installation in the cryogenic support system. \\ Right: Schematic cross section of the radial shield topology with the precise measurement of the prototype (not all the 12 meanders of the shield are depicted).}}
        	\label{CH3_radial_CCC}
        \end{figure}
        Each part of the detector need to be connected in a way to form one solid superconductor without holes that can deteriorate the shielding efficiency. The inner and outer Niobium disks have different diameters and are stacked consecutively on top of each other, interlaced by a sheet of fiberglass as electrical insulation. Fig.\ref{CH3_nBCCC_construction} shows two highlights of the construction process. All disks with a larger outer diameter are connected with electron beam welding along the outer circumference of the shield, while the disk with smaller inner diameter are connected to the inner circumference of the shield. After the assembly of the shielding disks, there was an electrical short between the inner and outer disks that was removed by applying a large current that locally melted the metal producing the electrical bypass \cite{DavidThesis}. With the given dimension it's possible to estimate analytically the expected dipole shielding factor A for the meander structure composed of 12 disks using eq.\ref{CH3_dipole_shielding_factor}:
        \begin{equation}
        	A_{radial} = 20 \log\left(\frac{B_{out}}{B_{in}}\right) = \prod^{12}_1 A_{radial,1} = \prod^{12}_1 \left(\frac{d_{out}}{d_{in}}\right)^2 \approx 58 \text{ dB} \footnote{The value is slightly lower than the value obtained with geometrical simulations ($\approx 70$ dB) that is much more similar to the measured value (see Chapter 5)}
        \end{equation}
        \begin{figure}[H]
        	\centering
        	\begin{subfigure}[b]{0.4\textwidth}
        		\centering
        		\includegraphics[width=\textwidth]{Plot/Chapter_3/CH3_nBCCC_construction.png}
        	\end{subfigure}
        	\hfill
        	\begin{subfigure}[b]{0.55\textwidth}
        		\centering
        		\includegraphics[width=\textwidth]{Plot/Chapter_3/CH3_nBCCC_construction_2.png}
        	\end{subfigure}
        	\caption{\small{Left: Installation of fiberglass spacers between the individual niobium shielding disks of the FAIR-nB-CCC-xD. Right: Assembled meanders disks before wielding. \textit{Courtesy of university of Jena}}}
        	\label{CH3_nBCCC_construction}
        \end{figure}
       \subsection{Pick-up and SQUID circuit} 
       While in principle the SQUID can be sufficient to measure the magnetic field of the beam, it is essential to have the highest possible effective measurement area to maximize the signal strength and the signal-to-noise ratio. However, in general, the inductance of the SQUID \(L_{\text{sq}}\), and thus its effective area, is kept low to minimize the intrinsic noise of the measured output signal \cite{Squid_handbook_1}. For this reason, it is necessary to use a superconducting pick-up coil \(L_p\) that is inductively coupled to the SQUID via an input coil \(L_i\) included on the chip carrier of the SQUID. 
       
       For an ideal transformer, the coupling of the two coils \((m,n)\), each in their separate systems, is described by the mutual inductance \(M_{(m,n)}\) and by the coupling constant \(k_{(m,n)}\) according to:
       \begin{align}
       	\Phi &= L \times I \nonumber \\
       	\Phi_A &= M_{(m,n)} I_B = k_{(m,n)} \sqrt{L_m L_n} I_B
       	\label{CH3_flux}
       \end{align}
       
       A primary current \(I_a\) creates the magnetic flux \(\Delta \Phi_T = M_{(p,a)} I_a\) in the pick-up loop, which can be seen as the secondary winding of a transformer. The change in the magnetic flux induces the current \(I_T = \Delta \Phi_T / L_T\) in the pick-up circuit with the total inductance of \(L_T = L_p + L_{\text{par}} + L_i\). Using eq.~\ref{CH3_flux}, the change in the flux in the SQUID \(\Delta \Phi_{\text{sq}}\) due to a primary current \(I_a\) can be written as \cite{CorelessCCC} \cite{Geithner_CCC}:
       \begin{equation}
       	\Delta \Phi_{\text{sq}} = M_{\text{sq,i}} I_T = M_{\text{sq,i}} \frac{\Delta \Phi_T}{L_T} = M_{\text{sq,i}} \cdot M_{\text{p,a}} \frac{I_a}{L_T} 
       	= k_{\text{sq,i}} k_{\text{p,a}} \frac{\sqrt{L_{\text{sq}} L_i L_p L_a}}{L_p + L_{\text{par}} + L_i}  I_a
       	\label{CH3_flux_difference}
       \end{equation}
       
       Eq.\ref{CH3_flux_difference} states that the maximum signal of the SQUID can be achieved when the secondary impedance \(L_i\) is equal to the primary inductance \(L_p + L_{\text{par}}\). This corresponds to a beam coupling \(M_{(p,a)} = L_p = L_a\). Using the idealized coupling factor \(k_{(m,n)} = 1\) for simplicity, eq.\ref{CH3_flux_difference} can be written as:
       \begin{equation}
       	\Delta \Phi_{\text{sq}} \lesssim \frac{1}{2} \sqrt{L_{\text{sq}}} \frac{\sqrt{L_p + L_{\text{par}}}}{1 + \frac{L_{\text{par}}}{L_p}} I_a
       	\label{CH3_flux_reduced}
       \end{equation}
       Eq.\ref{CH3_flux_reduced} shows that at a larger pick-up inductance correspond an increased flux applied to the SQUID. Furthermore, if $L_p$ is high enough, the effect of the parasitic inductance is negligible. To be able to operate the radial CCC in this condition the SQUID circuit is equipped with a superconducting flux transformer (Fig.\ref{CH3_flux_transformer})
       \begin{figure} [H]
       	\centering
       	\includegraphics[width=0.6\linewidth]{Plot/Chapter_3/CH3_flux_transformer.pdf}
       	\caption{\small{Schematic od the 2 stage SQUID circuit including the coupling transformer, the pick-up and the SQUID.}}
       	\label{CH3_flux_transformer}
       \end{figure}
       Eq.\ref{CH3_flux_difference} can be expanded to consider the effect of the transformer:
       \begin{equation}
       		\Delta \Phi_{\text{core, sq}} = M_{\text{sq,i}} \cdot M_{\text{t2,t1}} \cdot M_{\text{p,a}} \cdot \frac{1}{L_{T1} \cdot L_{T2}} I_a
       \end{equation}
       It's then possible to achieve an optimal transfer of the signal by correctly matching the primary ($L_p=L_{T1}$) and secondary inductance ($L_{T2} = L_i$) \cite{Tympel:IBIC2017-WEPCF07}. Furthermore, the flux transformer can be used to modify the frequency response of the system. The CCC can be described as an LC resonator, with C the capacity of the meander structure and L the inductance of the SQUID + pick-up coil. If the resonance peak it's too close to the operation range of the detector the SQUID will not be able to find a stable working point. The additional impedance provided by the flux transformer can then be used to dampen the LC resonance and ensure a correct SQUID operation.
       \subsection{High magnetic permeability core}
       To increase the coupling with the beam current the effective area need to be increased. Due to the limited available volume for the pick-up an high magnetic permeability core is needed. The use of a core, however, comes with several effects that needs to be considered.
       \begin{figure}[H]
       	\centering
       	\begin{subfigure}[b]{0.48\textwidth}
       		\centering
       		\includegraphics[width=\textwidth]{Plot/Chapter_3/CH3_core_inductance.png}
       	\end{subfigure}
       	\hfill
       	\begin{subfigure}[b]{0.45\textwidth}
       		\centering
       		\includegraphics[width=\textwidth]{Plot/Chapter_3/CH3_core_noise.png}
       	\end{subfigure}
       	\caption{\small{Left: Measurement of the real part of the cryogenic magnetic permeability $\mu(f)$ of the Nanoperm GSI-328plus core used in the FAIR-nB-CCC-xD. The resulting inductance of the pick-up coil is almost constant up to a frequency of 10 kHz \cite{Tympel:IBIC2016} Right: Comparative measurement of the cryogenic current noise spectral density of a bare dc SQUID (Magnicon GmbH) and of two different core materials (Vitrovac and Nanoperm) \cite{Geithner_core}}}
       	\label{CH3_core}
       \end{figure}
       First thing to consider is that the relative magnetic permeability $\mu_r(f)$ of the core is a function of the frequency of the applied field as shown in Fig.\ref{CH3_core}. Even if the core material is selected to have a very linear response through the desired measurement bandwidth, for frequencies higher than 10 kHz the magnetic permeability starts to drop significantly limiting the operating bandwidth of the detector \cite{Geithner_core}. As a consequence, the flux-to-current calibration factor \(k\) of the CCC is dependent on the frequency of the signal. While this was a negligible effect when the designed scope of use of the CCC was for storage rings, where the focus of interest was on DC or very slow changing currents, it is not negligible anymore when the detector needs to be used to analyze slow extracted spills that can have high-frequency components of interest. The introduction of the core can be seen as a natural low-pass filter which removes high-frequency perturbations. \\
       The high inductance of the core needs also to be considered. On the positive side, the inductance of the high permeability core $L_p$ is so high compared to the parasitic inductance of the pick-up circuit that the influences of $L_{par}$ over the current measurement is negligible. This means that the demand of the magnetic shield is lower for systems with cores. However, due to the huge mismatch between the impedance of the core and the one of the SQUID make almost necessary the presence of a coupling transformer like the one described above, requesting a more complex design for the pick-up circuit. \\
       Although the high permeability core leads to an increase in the effective amplitude of the signal reaching the SQUID, it also introduces additional noise to the system, as shown in Fig.\ref{CH3_core}. The magnetization of the core material is subject to thermal fluctuations and to Barkhausen noise (originating from the discrete reorientation of magnetic spin domains), resulting in low-frequency variations of the magnetic permeability and thus of the measured field. The resulting current noise density of the inductance can be estimated with the fluctuation dissipation theorem \cite{Geithner_CCC} \cite{Geithner_core}. The core is usually the dominant source of noise, especially at low frequencies, limiting the achievable current resolution. Furthermore the core it's susceptible to external perturbations. The magnetization of magnetic materials is linked to a change of their geometrical dimensions and to mechanical stress (a phenomena known as magnetostriction), opening to the possibility of converting mechanical vibrations to magnetic field affecting than the measurement. Previous studies in literature seems to connect the microphony noise observed in CCC system to the presence of the core \cite{Geithner_core}. It was also observed before a susceptibility of the CCC to temperature changes, which has origin in the change of the junction parameter of the SQUID itself \cite{Squid_handbook_1}. However, there is a contribution by the core as well, since its magnetic permeability \(\mu(T)\) is variable with temperature. The variation for the core material used in the radial CCC was measured in \cite{Tympel:IBIC2016} . Any fluctuation of the temperature will change the magnetization of the core and, in this way, will produce a signal in the SQUID.
       For radial CCC, like the FAIR-Nb-CCC-xD, the core is necessary to reach a sufficient beam coupling and an high enough signal to noise ratio, even with the low magnetic shielding available in general thanks to the radial shield topology.
       \subsection{FAIR-Nb-CCC-xD: Design Considerations and Shielding Efficiency}
       The high magnetic permeability core dominates the pick-up coil inductance \(L_p\). At 4.2 K, the inductance of the core is between 10 and 100 \(\mu\)H up to a frequency of 100 kHz, with an average inductance of 80 \(\mu\)H at 1 kHz (Fig.\ref{CH3_core}) \cite{Tympel:IBIC2016, Geithner_core}. The core inductance is high enough to fully satisfy the condition \(L_p \gg L_{\text{par}}\), thus making the effects of every parasitic inductance negligible. The signal of the pick-up coil is coupled to a single-stage dc SQUID from MAGNICON GmbH\footnote{1-stage Current Sensor C5XL1W from Magnicon GmbH, 22339 Hamburg, Germany} with an input inductance of \(L_i = 1 \mu\)H. With the direct coupling of the SQUID to the pick-up coil, it's possible to estimate the resonance frequency \(f_0\) of the pick-up circuit. It is dominated by the capacity of the shielding structure (\(C_{\text{shield}} = 44\) nF \cite{Tympel:IBIC2016}) and by the input inductance of the SQUID according to:
       \begin{equation}
       	f_0 = \frac{1}{2\pi\sqrt{L_i C_{\text{shield}}}} \approx 760 \text{ kHz}
       \end{equation}
       However, to achieve a more stable system, the SQUID is coupled to the pick-up coil using a matching transformer, optimized to maximize the inductance coupling. The use of the matching transformer moves the natural resonance frequency of the system to 170 kHz, as shown in Fig.\ref{CH3_resonance_peak}. For frequencies below 100 kHz, the flux transformer leads to an increase of the magnitude of the current by a factor of \(I/I_{\text{beam}} = 2.95\) (9.5 dB). The smothering of the resonance peak is essential to stabilize the system and ensure stable operation for the SQUID. Finally, the core \(\mu_r(f)\) is strongly damped for frequencies over 10 kHz, acting like a low-pass filter that stabilizes the system.
       
       \begin{figure}[H]
       	\centering
       	\includegraphics[width=0.7\linewidth]{Plot/Chapter_3/CH3_resonance_peak.pdf}
       	\caption{\small{The transfer function between the applied beam current \(I_{\text{beam}}\) and the current reaching the SQUID inductance \(L_i\) as a function of frequency simulated with LTspice. The transfer function is given for the direct coupling of the pick-up coil (\(L_p = 80 \mu\)H) to the SQUID \(L_i\) (solid black line) and for the coupling via a flux transformer (dashed green line).}}
       	\label{CH3_resonance_peak}
       \end{figure}
       
       Thanks to the addition of an external wire loop, it is possible to adapt the frequency response of the CCC by adding components to the SQUID circuit. In particular, the FAIR-Nb-CCC-xD is equipped with an additional 2.2\(\Omega\) resistor, connected in series to the pick-up, that enhances the low-pass behavior of the system, bringing the cut-off frequency to \(\approx 7\) kHz (Fig.\ref{CH3_2.2Ohm}). This enhanced low-pass filter is necessary not only to improve system stability but mainly to improve the slew rate of the detector. Previous tests of the prototype at CRYRING@GSI \cite{DavidThesis} have shown events of insufficient slew rate during the use of the detector on the storage ring. To improve the slew rate, which is essential for analyzing slowly extracted spills where localized peaks can have very steep slopes, the 2.2\(\Omega\) resistor has been added. Strongly dampening higher frequency components increases the stability of the system and allows achieving a maximum slew rate of 0.16 \(\mu\)A/\(\mu\)s. However, the 2.2\(\Omega\) resistor introduces additional thermal noise, increasing the noise floor of the detector as shown in Fig.\ref{CH3_2.2Ohm}, where the voltage density after the installation of the 2.2\(\Omega\) resistor (in black) is noisier than the undamped one across the full frequency spectrum. Furthermore, while the lower cut-off frequency is not an issue for the detector when used in storage rings, where the interest is in dc or slowly changing currents, it can be limiting for the analysis of spills, where operators and accelerator scientists may be interested in high-frequency components. This is particularly true for beams extracted using RF extraction methods, which are based on the excitation of the beam with high-frequency signals. The detector should be able to distinguish the effects on the spill, and this is not possible if the cut-off frequency of the CCC is too low.
       
       \begin{figure}[H]
       	\centering
       	\includegraphics[width=0.8\linewidth]{Plot/Chapter_3/CH3_2.2Ohm.pdf}
       	\caption{\small{Squid voltage density as a function of frequency for the damped CCC (2.2\(\Omega\) resistor installed) in black, and the undamped one in red. The resonance peak above 100 kHz is strongly damped, however, the noise across the full frequency spectrum is increased due to the additional thermal noise of the resistor.}}
       	\label{CH3_2.2Ohm}
       \end{figure}
       
       \section{Axial Coreless CCC for FAIR}
       
       In this section, the first of the two axial versions developed for FAIR will be described. Both of these versions are built out of lead using an axial shield topology, but they differ completely in every other aspect. This CCC, developed together with Leibniz IPHT\footnote{Quantum system department, Leibniz institute of Photonic Technology (Leibniz IPHT), 07745 Jena, Germany}, will lack a high magnetic permeability core and will be equipped with a two SQUID system to increase the slew rate.
       
       \subsection{Shielding Factor Estimation}
       
       As it was done for the radial topology, the performance of an axial shield can be estimated with geometrical simulations. The shielding factor is determined as a function of the outer radius and of the length of the detector, with a fixed internal diameter \(D_{\text{in}} = 250 \text{mm}\). The simulation was performed with the shield placed on the inside layer of the CCC, and with Nb used as material. Fig.\ref{CH3_Axial_shield_simulation} shows how the axial topology is much more effective to reach higher screening factor, 3 layers are enough to have the same shielding of a radial superconducting shield with 12 meanders. In reality, the achievable screening factors are even better than those shown in the simulations, thanks to the possibility of using Lead. Niobium walls need to have a thickness of at least 3 mm and a space between the layers of about 0.5 mm. Instead, with lead, it's possible to have wall thickness of 0.25 mm and space between the layers of around 0.5 mm, allowing fitting a much higher number of layers in the same volume. This allows for much higher screening factors while using less volume in the detector, which can be allocated to increase the volume of the pick-up coil or the shield itself. In the same area occupied by 3 layers of Nb, it is possible to fit around 15 layers of lead, for an expected screening factor higher than 200 dB. In any case, the simulations state without any doubt that the axial geometry allows for reaching much higher screening factors than the radial one while keeping constant, if not reduced, the volume of the detector.
       
       \begin{figure}[H]
       	\centering
       	\includegraphics[width=0.8\linewidth]{Plot/Chapter_3/CH3_axial_shield_simulation.png}
       	\caption{\small{Simulation of the shielding factor of an axial topology superconducting shield (shielding layers at \(R_{\text{in}}\)) for different outer dimensions (outer radius, length) and the resulting number of coaxial shielding layers. The inner diameter (\(D_{\text{in}} = 250 \text{mm}\)) and the cross-section of the pick-up are fixed. \textit{Courtesy of N. Marsic (TU Darmstadt) \cite{Marsic_BMBF}.}}}
       	\label{CH3_Axial_shield_simulation}
       \end{figure}
       
       As a last consideration, it is important to note that a Lead CCC can be manufactured without the use of any particular technique, strongly reducing the production cost. Thus, together with the lower price of Lead in respect to Nb, it is possible to build a CCC out of Lead for a 1/10 of the cost of a Nb based CCC.
       
       \subsection{Construction Process}
       
       The Coreless CCC has a length of 290 mm, an inner diameter \(D_{\text{in}} = 250\) mm, an outer diameter \(D_{\text{out}} = 320\) mm, and a shielding structure composed of 10 coaxial layers of Lead. 
       
       \begin{figure}[H]
       	\begin{subfigure}[b]{0.4\textwidth}
       		\centering
       		\includegraphics[width=\textwidth]{Plot/Chapter_3/CH3_axialCCC.png}
       	\end{subfigure}
       	\hfill
       	\begin{subfigure}[b]{0.55\textwidth}
       		\centering
       		\includegraphics[width=\textwidth]{Plot/Chapter_3/CH3_axial_dimension.png}
       	\end{subfigure}
       	\caption{\small{ Left: Full view of the axial coreless CCC for FAIR. Right: schematic of the coreless CCC with the correct dimensions. XPS is a non-magnetic foam that provides structural support for the pick-up volume.}}
       	\label{CH3_axial_CCC_halfview}
       \end{figure}
       
       This CCC is built out of Lead, and the manufacturing process has been performed in Leibniz IPHT during the summer of 2022. The process starts with the outer layer, which works as the outer layer for the shield and the inner layer of the detector, built with a sheet of "Hard" lead of 1 mm thickness. The inner layer is welded\footnote{Using Lead-Tin (60Sn-40Pb) soldering wire, it's possible to ensure that the welding surfaces are also superconductive} to the upper and lower cap, cut in ring shape from a copper sheet of 1.5 mm thickness (upper right in Fig.\ref{CH3_ax_construction}). A first fiberglass\footnote{Fiberglass reinforced plastic (GRP) provides high mechanical stability at very low weight and can be used in cryogenic environments} cylinder of 2 mm thickness is then placed after the inner layer to provide structural support.
       
       \begin{figure}[H]
       	\centering
       	\includegraphics[width=1\linewidth]{Plot/Chapter_3/CH3_construction.png}
       	\caption{\small{Top Left: Inner layer of the CCC connected to bottom and upper cap. A GRP tube is added for mechanical support. Top Right: First meander layer installed, the paper that provides insulation will then be glued to the surface using cryo-glue. Bottom Left: XPS foam is cut in split ring shapes and then glued to the inner surface of the pick-up coil. Bottom Right: Once all the foam parts are positioned, they are held in position until the glue has fixed them, then the pick-up coil is closed.}}
       	\label{CH3_ax_construction}
       \end{figure}
       
       The meander structure is instead built using 9 layers of 0.25 mm Pb-Sn (97.5 Pb - 2.5 Sn) sheet interlaced with paper layers, glued\footnote{Epoxy glue type L} to the outer surface of each meander to provide electrical insulation. The bonding of the layer provided by the glue should help, in principle, to stop relative motion of the meander layers and to reduce the sensitivity of the detector to microphonic perturbations (Top right of Fig.\ref{CH3_ax_construction} shows the first layer installed). The layers are alternately connected to the bottom or the upper cap of the CCC, while always maintaining a 5 mm space between the outer edge and the cap to which they are not connected. Particular attention is dedicated to the removal of every imperfection in the welding that can generate shortcut in the meander structure. Once the structure is completed the pick-up coil can be assembled. \\
       To increase the effective area two nested Lead coils will be used, still built out of the 0.25 mm Pb-Sn sheet used for the meanders. Again, the two coils are insulated using paper to avoid shortcut that could reduce the effective area\footnote{The use of two coil is necessary to increase the inductance as much as possible to make the effect of parasitic inductances negligible. While this is always true when a core is used, reaching high inductance without a core is difficult and ask for the maximum possible available surface}. First, it's necessary to install the inner part of the coils, composed of the inner surface and the upper and lower cap. Once this is done, it's possible to start inserting the extruded polystyrene (XPS) for mechanical support of the area (See Fig.\ref{CH3_ax_construction} bottom left and right for two phases of the assembly). XPS is chosen because it's a non magnetic foam, cryo-compatible, light-weight, with closed pores to avoid contamination of liquid helium with water or gasses that may stay trapped inside an other type of foam during the cool-down. The XPS sectors are glued to the surface and then glued together to keep them stable. Once the XPS foam is fixed the two nested pick-up coils are closed. Two small stripes of Lead are used to connect the pick-up coil to the SQUID box installed outside of the outer cap of the CCC. To allow the strip to overcome the superconducting shields two pinholes are cut in the outer surface of one cap of the CCC, and insulated again with paper to avoid shortcuts between pick-up and meanders.\\ The assembled pick-up coils extends over a diameter of 275/318.75 mm ($d_{in}/d_{out}$) and along almost the full length of the detector (287 mm). It's possible to use this dimensions to estimate the inductance of the pick-up coil with two turns as:
       \begin{equation}
       	L_p\approx 2^2 \times \frac{\mu_0\mu_rl}{2\pi}\log\left(\frac{d_{out}}{d_{in}}\right) = 4\times 12.2 \text{ nH} \approx 50 \text{ nH}
       \end{equation}
       The CCC is then enclosed in a last GRP cylinder for additional mechanical support. A last layer of Lead (1 mm of thickness) is placed, connecting bottom and upper cap, to close the detector. All the GRP supports used are necessary because Lead is a soft material and it needs an additional rigid support to avoid dramatic mechanical deformation that can create shortcuts in the meanders. \\
       \subsection{SQUID circuit}
       The SQUID circuit is installed in a lead box welded on top of the detector itself, as shown in Fig.~\ref{CH3_coreless_circuit}. Due to the much lower inductance of the pick-up coil, it is essential to consider the possible parasitic inductance present in the system. The two main parasitic inductances are provided by the coaxial screening layers of the conducting shield and the wiring of the pick-up circuit. The first parasitic inductance, $L_{par1}$, of the coaxial shield can be estimated using $R_{in}^{shield} = 125$ mm and $l_{gap} \approx 0.5$ mm to be \cite{CorelessCCC}:
       \begin{equation}
       	L_{par1} \approx \frac{\mu_0 \pi l_{gap}}{4 l} R_{in}^{shield} \approx 0.4 \text{ nH}
       \end{equation}
       The second parasitic inductance cannot be estimated theoretically and can only be measured. However, similar setups show an inductance on the order of $L_{par2} \approx 6$ nH.\\
       To optimize the signal coupling to the SQUID, its input inductance should match the input inductance of the pick-up circuit $L_p$ (including the parasitic inductance) according to $L_i = L_p + L_{par}$. To optimize the inductance matching, the Supracon SQUID CN4, a 2-stage dc SQUID\footnote{2-stage Current Sensor CN4 (4so3049/42C2), equipped with an ultra-low noise SQUID-based array amplifier (4so2966/65C3) from Supracon AG, 07751, Jena, Germany}, has been selected, with an inductance $L_i = 44$ nH. Furthermore, to achieve a higher bandwidth, the detector is also equipped with a lower inductance SQUID, CN2 from Supracon, with $L_{CN2} = 11$ nH, which works in parallel with CN4 as shown in the schematic in Fig.~\ref{CH3_coreless_circuit}.
       \begin{figure}[H]
       	\begin{subfigure}[b]{0.45\textwidth}
       		\centering
       		\includegraphics[width=\textwidth]{Plot/Chapter_3/CH3_Coreless_CCC_circuit.png}
       	\end{subfigure}
       	\hfill
       	\begin{subfigure}[b]{0.6\textwidth}
       		\centering
       		\includegraphics[width=\textwidth]{Plot/Chapter_3/CH3_coreless_SQUID_circuit.png}
       	\end{subfigure}
       	\caption{\small{Left: SQUID circuit enclosed in the lead boxes; all parts of the circuit are highlighted. Right: schematic of the SQUID circuit serving the Coreless CCC; the two SQUIDs work in parallel. The two passive components used to change the resonance frequency of the LC circuit are shown in red.}}
       	\label{CH3_coreless_circuit}
       \end{figure}
       CN4 is operated in FLL mode and is theoretically able to detect current signals from 1 nA to some $\mu$A, while CN2 is operated in direct readout mode, where it can be used as a discrete counter of flux quanta applied to the SQUID. Thus, the bandwidth of CN2 is very high, over 100 MHz, with a large range of measurable currents. The combined use of the two SQUIDs should allow detection of slowly changing currents with high resolution using CN4, and detection of fast-changing currents with lower resolution using CN2 \footnote{The idea is based on existing cascade SQUID system, able to improve the sensitivity without increasing the noise \cite{Cascade_squid}}. This system is expected to have a slew rate more than 10 times higher than that achievable with the radial CCC using only CN4, and a slew rate several orders of magnitude higher using CN2, significantly increasing the spectrum of possible applications of the CCC.
       
       The last thing to consider is the resonance frequency. By measuring the capacitance of the shield, it is possible to estimate the resonance frequency $f_0$:
       \begin{equation}
       	f_0 = \frac{1}{2\pi\sqrt{L_i C_{shield}}} \approx 1.5 \text{ MHz}
       \end{equation}
       using $C_{shield} = 240$ nF and the inductance $L_i$ of the CN4 SQUID. This frequency is too close to the desired operation range of the detector, so it is necessary to either move the frequency further away from the desired operation area or dampen it enough to be able to ignore it. The idea can be the same as that applied to the FAIR-Nb-CCC-xD, but in this case, the matching transformer is not available. The absence of the core allows matching the inductance of the SQUID to the pick-up quite easily, making the presence of the matching transformer unnecessary and allowing the removal of the noise component caused by it. Therefore, it is essential to find a different way to modify the resonance peak. In Fig.~\ref{CH3_coreless_circuit}, the two passive components that can be used are highlighted in red. By applying a resistance and a capacitance in series to the pick-up coil, it is possible to modify the capacitance of the LC circuit to shift the core frequency of the peak and decrease the intensity of the resonance, at the cost of an additional thermal noise component. In Fig.~\ref{CH3_Coreless_bandwidth}, the effect of different values of the resistance on the frequency behavior of the two SQUIDs is shown for a fixed value of the capacitance $C = 1.7 \mu$F.
       \begin{figure}[H]
       	\begin{subfigure}[b]{0.48\textwidth}
       		\centering
       		\includegraphics[width=\textwidth]{Plot/Chapter_3/CH3_CN4_damping.png}
       	\end{subfigure}
       	\hfill
       	\begin{subfigure}[b]{0.48\textwidth}
       		\centering
       		\includegraphics[width=\textwidth]{Plot/Chapter_3/CH3_CN2_damping.png}
       	\end{subfigure}
       	\caption{\small{Left: Frequency behavior of CN4 SQUID as a function of resistance value for a fixed value of the capacity of 1.7 $\mu$F. Right: Frequency behavior of CN2 SQUID as a function of resistance value for a fixed value of the capacity. Is it possible to see how the use of a different resistance lower the intensity of the resonance peak. The chosen values for the coreless CCC are the one depicted by the red curve. G represents the gain of the SQUID. \textit{Both pictures courtesy of V.Zakosarenko, Leibniz IPHT}}}
       	\label{CH3_Coreless_bandwidth}
       \end{figure}
       Fig.~\ref{CH3_Coreless_bandwidth} shows how the choice of this two passive components is able to modify the transfer function of the two SQUID. To achieve the lowest possible resonance peak intensity and the most stable transfer function over the full frequency spectra a resistance of 0.5 $\Omega$ has been chosen. It is possible to compare the expected transfer function of the SQUID used in the axial coreless CCC to the one of the FAIR-Nb-CCC-xD (Fig.~\ref{CH3_resonance_peak}) to immediately see how the bandwidth of the coreless CCC is much higher, able then to achieve a better slew rate for the detector.
       \section{Dual Core Axial CCC (DCCC)}
       In this section, the last of the CCCs tested during this thesis work will be described. The idea of the DCCC is to use the strong magnetic shielding provided by an axial shield and combine it with the optimal beam coupling provided by the use of the high magnetic permeability core. The DCCC is also built out of lead, and it was manufactured by FSU Jena during the autumn of 2023. The first prototypes of the DCCC were part of the Smart\&Small (Sm series) section of the BMBF collaboration funding this research, where the proof of principle of a double core CCC was verified \cite{SmDCCC}. For the Sm series, several studies on the noise behavior of the double core system were performed. The complete description of these studies doesn't fit within the scope of this thesis; further information can be found in \cite{Max_thesis}. Furthermore, previous experience with CCC detectors has shown that the noise behavior strongly changes when their dimensions increase, so the results of this noise analysis can't be used to predict the noise behavior of the DCCC for FAIR but have been used as general guidance during its construction.
       \subsection{Construction and shielding factor}
       All the consideration performed in section 3.4.1 for the axial shield of the coreless CCC are still valid. The construction method described in section 3.4.2 is again quite similar to the method used to build the DCCC (Fig.\ref{CH3_DCCC_schematic}), the dimension are also basically the same. However, there are some essential differences that need to be noted:
       \begin{itemize}
       	\item \textbf{Meanders}: The superconducting shield structure have only 7 meanders, for a total shield length of 140 cm. The electrical insulation used between the lead meanders is fiberglass sheet of 0.25 mm of thickness. In the end the volume occupied by a single meander layer is a bit bigger than the one occupied by the meanders in the coreless CCC, that are insulated only with paper, but the structure is now more rigid and the shortcut between meanders caused by deformation of the surfaces while cooling down are less probable. 
       	\item \textbf{Pick-up coil}: The pick-up coil is composed by two single turn coils, each of them enclosing one of the two high magnetic permeability cores, as shown in Fig.\ref{CH3_DCCC_schematic}. The pick-up coils are built to be the most symmetric as possible, with the only differences in the lengths of the cables connecting them to their respective SQUID. System redundancy allows to equip the DCCC with two SQUID measuring in parallel, to compare the noise behavior of the two pick-up coil and increase the signal to noise ratio by combining the measurement performed by the two SQUIDs.
       \end{itemize}
       \begin{figure}[H]
       	\begin{subfigure}[b]{0.5\textwidth}
       		\centering
       		\includegraphics[width=\textwidth]{Plot/Chapter_3/CH3_DCCC_full.pdf}
       	\end{subfigure}
       	\hfill
       	\begin{subfigure}[b]{0.6\textwidth}
       		\centering
       		\includegraphics[width=\textwidth]{Plot/Chapter_3/CH3_DCCC_schematic.png}
       	\end{subfigure}
       	\caption{\small{Left: Full view of the DCCC, on the top it's possible to see the connectors to the SQUIDs contained in the SQUID cartridge. Right: Schematic of the DCCC, with the two pick-up coil highlighted enclosed in the same superconducting shield. Each coil encloses a core and is connected to a SQUID. \textit{Both pictures courtesy of V.Tympel, FSU Jena}}}
       	\label{CH3_DCCC_schematic}
       \end{figure}
       The DCCC is also equipped with fiberglass cylinders to protect the inner and outer parts of the superconducting shield to avoid deformation of the outer lead surfaces of the detector. It was possible to compare, in the controlled environment of the laboratory at FSU Jena\footnote{It will be described in Chapter 4, it provides a controlled environment where the desired CCC can be installed in a bottleneck cryostat, allowing to test them with less noise input than in GSI} the magnetic screening factor of the DCCC for FAIR with that of the radial FAIR-Nb-CCC-xD and those of two Sm series CCCs, to show how much more effective the axial geometry is for magnetic shielding. Fig.~\ref{CH3_screening_factor} shows the coupling factor of the SQUID signal with an external magnetic field. It is possible to see that the FAIR DCCC (DCCC-xD) has a coupling factor three orders of magnitude lower than that of the FAIR-Nb-CCC-xD, indicating a shielding factor approximately 1000 times higher. Additionally, two different versions of the double core CCC from the Sm series are included to show how the coupling factor changes with the effective length of the shield and how the increased dimensions of the FAIR DCCC modify the screening factor.
       \begin{figure}[H]
       	\centering
       	\includegraphics[width=1\linewidth]{Plot/Chapter_3/CH3_shielding.png}
       	\caption{\small{Coupling factor of the SQUID measured as $nA$ of current produced by the SQUID for each $\mu$T of field provided to the detector. Usual with noise floor of CCC is higher than 1 nA, so everything below 1 nA is basically negligible. Both SQUIDs used in the DCCC are plotted, the same result is not achieved because the system is not exactly symmetrical (SQUID cables longer for the red one). The measurement for some Sm series CCC are shown, to highlight the effect of the increased effective length and the effect, even if small, of increased dimension on the shield effectiveness.}}
       	\label{CH3_screening_factor}
       \end{figure}
       \subsection{DCCC High magnetic permeability cores}
       The DCCC is equipped with high magnetic permeability cores to improve the detector coupling with the beam. These cores are built from a rapid-solidified, iron-based alloy strip material, which is wound into toroidal cores and optimized for the crucible temperatures in a post-treatment according to a special recipe developed and built by Magnetec \cite{Volker_IEEE}. The DCCC cores are multiple wound toroids with an inductance of 100 $\mu$H at room temperature.
       
       As discussed in Section 3.3.4, the introduction of a magnetic permeability core introduces additional noise to the system. However, the use of two cores in the DCCC helps to dampen the effects of these noise sources. In particular, it is possible to significantly reduce the effect of low-frequency noise, such as Barkhausen noise or flux jumps, by combining the signals of the two different cores through manipulation of the respective SQUID signals \cite{Max_thesis}.
       
       Thanks to technological improvements in the core material, those used in the DCCC have an inductance that is less dependent on frequency than the cores used in the FAIR-Nb-CCC-xD, making it possible to achieve a lower noise level even at higher frequencies \cite{Volker_IEEE} ($f > 1$ kHz, Fig.~\ref{CH3_DCCC_core}).
       
       Fig.~\ref{CH3_DCCC_core} shows how the DCCC has a lower noise level than the FAIR-Nb-CCC-xD at high frequencies. It also indicates potential improvements in future versions of the detector, as shown by the Sm series CCC, which presents an even lower noise floor due to its reduced dimensions and more accurately built cores. If the technology used to construct high magnetic permeability cores continues to advance, it will be possible to have higher quality cores for larger CCCs, further improving the noise behavior of the detector. The plot also shows the noise behavior of the CCC used at the Antiproton Decelerator (AD) at CERN (\cite{CERNCCC_1}\cite{CERNCCC_2}), a radial CCC built before the FAIR-Nb-CCC-xD, which already demonstrates how small improvements in the detector can enhance noise performance.
       
       \begin{figure}[H]
       	\begin{subfigure}[b]{0.54\textwidth}
       		\centering
       		\includegraphics[width=\textwidth]{Plot/Chapter_3/CH3_DCCC_inductance.png}
       	\end{subfigure}
       	\hfill
       	\begin{subfigure}[b]{0.55\textwidth}
       		\centering
       		\includegraphics[width=\textwidth]{Plot/Chapter_3/CH3_DCCC_core_noise.png}
       	\end{subfigure}
       	\caption{\small{Left: Frequency dependence of the inductance and resistance of the core used in the DCCC. Right: Noise figure of different core-equipped CCCs. It is possible to see how the FAIR DCCC prototype already shows the lowest noise of the three extended dimension CCCs, but still higher than the smaller version of the DCCC. \textit{Both pictures courtesy of V. Tympel, FSU Jena}}}
       	\label{CH3_DCCC_core}
       \end{figure}
       
       \subsection{DCCC: SQUID circuit}
       The DCCC is equipped with a 3 SQUID system, composed of two 1 $\mu$H SQUIDs\footnote{Magnicon Single Stage DC SQUID} and one 27 nH SQUID, as shown in Fig.~\ref{CH3_DCCC_SQUID_Circuit}.
       
       \begin{figure}[H]
       	\centering
       	\includegraphics[width=1\linewidth]{Plot/Chapter_3/CH3_DCCC_SQUID.png}
       	\caption{\small{Schematic of the SQUID circuit used in the DCCC. The main parts are highlighted. The third SQUID is missing; it is a 27 nH SQUID connected in series to the yellow SQUID and is omitted because its effect is negligible on the SQUIDs themselves. The two damping systems used to manage the resonance frequency are highlighted, with a low pass filter on the SQUID circuit and a 2.2 $\Omega$ resistor that can be added in the meander structure.}}
       	\label{CH3_DCCC_SQUID_Circuit}
       \end{figure}
       
       The two SQUIDs are connected in the most symmetric way possible, each to one of the two pick-ups, and are called from now on Red and Yellow SQUIDs, based on the color codes of the DCCC connector box. The third SQUID is added in series to the Yellow SQUID and is called the Green SQUID. Its purpose is to provide a strong extension of the dynamic range of the DCCC; thanks to its low sensitivity, it can be used to measure really high slew rates and high current intensity without losing the Flux Locked Loop mode. In this thesis work, where we were interested in the measurement of low-intensity spills, the Green SQUID is almost never used because it is not sensitive enough to provide useful information. In any case, it is important to mention its existence in the SQUID circuit because it allows the DCCC to also be used as a beam current monitor for high beam currents.
       
       The bandwidth of the system is much higher than that of the FAIR-Nb-CCC-xD, up to 1 MHz, though it is obviously lower than the bandwidth of the coreless CCC because a low pass filter is essential when the core is installed to move the resonance frequency of the LC circuit out of the operational area of the CCC. However, a slew rate between 3 and 10 times higher than that of the FAIR-Nb-CCC-xD is expected for the DCCC.
       
       In the circuit in Fig.~\ref{CH3_DCCC_SQUID_Circuit}, an important role is played by the 2.2 $\Omega$ resistor that can be installed in the meander structure. If this resistor is not installed, then the resonance frequency of the system is around 200 kHz. If the resistor is installed than the 200 kHz resonance peak smooths out but the thermal noise of the resistance increase the noise floor, as shown in Fig.\ref{CH3_DCCC_resonance}. The DCCC has been tested in Jena without the resistance that has been then installed before its test in GSI, to avoid resonances when used on the beamline, further information will be provided in Chapter 7, were the results achieved with the DCCC will be described.\\
       \begin{figure}[H]
       	\centering
       	\includegraphics[width=1\linewidth]{Plot/Chapter_3/CH3_resonance_peak.ng}
       	\caption{\small{LT-Spice simulation of the DCCC SQUID circuit (Fig.~\ref{CH3_DCCC_SQUID_Circuit}) to show the effects of the 2.2 $\Omega$ resistor. On the Y-axis the voltage noise density of the SQUIDs as a function of the frequency. The addition of the resistor smooths the resonance peak but increase the noise density at low frequencies.}}
       	\label{CH3_DCCC_resonance}
       \end{figure}
       One last thing to discuss is the lack of the matching transformer, like the one used in the FAIR-Nb-CCC-xD. It was decided not to install a matching transformer on the DCCC. The decision was mainly driven by the idea of having the least amount of noise sources and by the lack of empirical proof that the matching transformer effectively improved the signal of the FAIR-Nb-CCC-xD. Furthermore, the matching transformer is not as effective as the 2.2 $\Omega$ resistor at smoothing out the meander resonance peak. While for the FAIR-Nb-CCC-xD the capacitance of the meanders is small, and so is the resonance peak intensity, for the DCCC the peak is much bigger and needs to be strongly dampened. Thus, the matching transformer has been omitted, relying only on passive elements to smooth the resonance peak and trying then the keep the noise sources as low as possible.
       
       \section{Integration of CCC into the accelerator diagnostic}
       Independently of the detector types, all the CCCs share the same signal output, a -10/+10 V sweep, which was collected during this thesis work using various kinds of oscilloscopes or current converters, all operated locally. This is not feasible for operators of the accelerator machine, who need to be able to read the device signal in the general accelerator data environment. At GSI, the system used is called Front-end Software Architecture (FESA) \cite{FESA_Class}, developed at CERN (where it is generally used for the control of the LHC and other components of the accelerator complex) and then adapted to the specific requirements of GSI. FESA is a software architecture that collects the signals from the respective devices and converts them into plots or numbers that can be read by the operators to extract information on the device's measurements and apply corrections to the beam parameters. This section will briefly report the requirements of the FESA system to implement the CCC as a standard diagnostic device of FAIR. In Chapter 7, a first test of the FESA will be presented, together with the CCC data collected as usual, to verify the effectiveness of this first prototype of integration into the environment.
       
       The main steps needed to integrate the CCC into the FAIR FESA are:
       \begin{itemize}
       	\item \textbf{Convert the voltage output to a positive signal}: The first step for the FESA connection is the conversion, through a Voltage to Frequency converter, of a positive signal into its frequency components. It is necessary to convert the -10/+10 V of the SQUID to a positive 0/10 V, and this is done through a level shifter. However, there are two main problems that necessitate the correct choice of the level shifter: Firstly, the SQUID is usually sitting on top of a DC voltage offset, which is natural in the FLL operation mode; Secondly, it is essential to maintain the shape of the signal intact because it can happen, due to the offset or the internal wiring of the SQUIDs, that the signals are below the 0 level. Analysis on the best level shifter for the CCC connection to the FESA is still ongoing.
       	
       	\item \textbf{Connection to FAIR accelerator system}: The frequency signal is then transported to the electronics room where it is fed to the Large Analogue Signal and Scaling Information Environment for FAIR (LASSIE) \cite{LASSIE}. In LASSIE, it is possible, with some calibration factors depending on the detector used, to display plots of the waveform, extract the beam current, the particle number, and visualize every interesting measurement performed with the device.
       	
       	\item \textbf{On-line and post-processing of data}: Even though the CCC signal has already been read using LASSIE during the test of the prototype for the storage ring tested at CRYRING@ESR \cite{DavidThesis}, this is not sufficient to state that the CCC is integrated into the FAIR accelerator control system. The test of the CCC for spill analysis has shown, and it will be described in detail in Chapter 7, how the baseline and the low-frequency noise\footnote{Sources of this noise are several: mechanical perturbations, pressure vibrations, acoustic noise, time-averaged higher frequency noise, and electromagnetic noises. All these real signals are picked up by the SQUID and are present as noise superimposed on the beam signal.} present in the SQUID signal necessitate an on-line form of filtering for the data if the goal is to use the CCC data as feedback for other accelerator components. The data reaching LASSIE are sufficient to provide the operator with essential information on the spills but are not refined enough to be provided. for example, as feedback signals for RF spill optimization techniques\footnote{Chapter 7 will provide a more detailed description.}. It is necessary to provide digital filters that can be decided upon and changed during operation to reduce the impact of the noise on the SQUID signal. Finally, it is also necessary to store the measured data to perform post-processing and further analysis of the collected beam currents.
       	
       	\item \textbf{Slow controls}: Last but not least is the necessity of integrating into FESA all the information about the detector that is not related to the beam, for example, vacuum levels, temperatures, liquid helium levels, and many others\footnote{Further information on the various sensors that need to be managed will be provided in Chapter 4.}. These data are currently read locally on dedicated devices, but all of them needs to be collected and unified in a single slow controls environment that can be accessed at any time, together with the collected data. Obviously, for slow controls, the hardware requirements, such as resolution and sampling rate, are much less strict and generally easier to implement. In the case of the CCC, the very high number of parameters to control ($\gtrsim$ 20) makes the process of building a slow control environment non-trivial.
       	
       \end{itemize}
       The development of the FESA class is a complex work that is followed by expert programmers at GSI and it is in continuous improvements with the addition of functions, digital filters and hardware improvements. However, the test presented in the last part of this thesis is already a proof of principle that the CCC, independently of the version used, can be integrated as standard diagnostic in FAIR.
       
       \section{Summary of CCC prototypes}
       The three CCCs used in this thesis work have been described in this section. Table \ref{CH3_table} provides a summary of the essential details of the CCC detector for a quick reference. Further details about the SQUIDs and the noise will be provided in further chapter and wrapped up in the conclusion of this work.
       
       \begin{table}[ht]
       	\begin{tabular}{c|c|c|c}
       		\hline
       		Detail & FAIR-Nb-CCC-xD & Coreless CCC & DCCC \\
       		\hline
       		Material & Niobium & Lead & Lead \\
       		\hline
       		Shield geometry & Radial & Axial & Axial \\
       		\hline
       		Length[mm] & 207 & 290 & 220 \\
       		\hline
       		Inner/outer D[mm] & 250/350 & 250/352 & 250/350 \\
       		\hline
       		Gap Meanders[mm] & 1 & 0.5 & 0.5 \\
       		\hline
       		Surface thickness[mm] & 3 & 0.25/1 & 0.25/1 \\
       		\hline
       		Insulation material[mm] & GRP & Paper & GRP \\
       		\hline
       		Mass [Kg] & $\approx$ 55 & $\approx$ 25 & $\approx$ 40\\
       		\hline
       		Shielding factor & $\approx$ 70 dB & $\approx$ 200 dB & $\approx$ 180 dB \\
       		\hline
       		Pick-up inductance & 80 $\mu$H (@1 kHz) & 44 nH & 100 $\mu$H $\times 2$ (@1 kHz) \\
       		\hline
       		SQUID(s) & 1 $\mu$H SQUID & 50nH SQUID, 11nH SQUID & 1 $\mu$H SQUID$\times 2$, 27nH  \\
       		\hline
       		Bandwidth & $<$ 10 kHz & $<$ 10 MHz & $<$ 1 MHz \\
       		\hline
       		Slew rate & 0.16 $\mu$A/$\mu$S \cite{DavidThesis} & $\gtrsim$ 0.4 $\mu$A/$\mu$s  & $\gtrsim$ 1 $\mu$A/$\mu$s \\
       		\hline
       	\end{tabular}
       	\caption{Resume of differences in design between the three different CCCs prototype build and tested for FAIR}
       	\label{CH3_table}
       \end{table}
       
       \chapter{GSI/FAIR CCC beam-line cryostat}
       As reported in previous sections of this thesis, to ensure the correct functionality of the CCC, a very stable operation environment is needed, in particular pressure and temperature need to be as stable as possible. Every oscillation introduces noise that can deteriorate the measurement. It is essential, in particular, to ensure that the CCC is always in a superconductive state. To fulfill these requirements, a custom-made beamline cryostat paired with a local helium re-liquifier has been developed to provide the stable cryogenic environment needed for the operation of the CCC. This system, developed in collaboration with the Institut für Luft- und Kältetechnik gGmbH\footnote{Institute of Air Handling and Refrigeration gGmbH (ILK in German) 01309, Dresden, Germany} \cite{DavidThesis} \cite{ILK}, will be described in this chapter. Furthermore, a brief description of the Low-Temperature laboratory located at Friedrich-Schiller Universität (FSU) Jena will be provided, together with the description of the cryostat used there, to give a full description of every laboratory used during the development of the CCC prototypes for FAIR.\\
       \section{Cryostat design request}
       	Based on the properties of CCC detectors and the requirements of FAIR, the cryostat must meet precise specifications, foremost of which is the minimization of background noise. The achievable current resolution of the CCC is directly correlated with various background noises originating from the mechanical and electrical conditions of the operating environment. Thus, the demands on the cryogenic support system exceed the basic need of providing a stable cryogenic temperature. The cryostat needs to act as the first layer of defense against external perturbations.
       	
       	It is documented that SQUIDs are sensitive to variations in their operating temperature and to the pressure above the liquid helium bath \cite{Squid_handbook_1} \cite{KurianTh}. For a CCC system, it has been shown that temperature effects result in a variable error on the order of 33.5 nA/mK, which is equivalent to 73.7 nA/mbar. This strong dependency highlights the need for precise and accurate control and stabilization of pressure and temperature.
       	
       	Furthermore, every mechanical vibration that propagates to the CCC will couple directly with the SQUID, generating background noise. Even though the precise mechanism of the coupling is still unclear\footnote{The two most probable causes that have been investigated are the direct coupling through the relative motion of the SQUID inside the surrounding magnetic field or the indirect coupling through the relative motion of the magnetic material around the SQUID that changes the magnetic flux at the detector itself.}, the effects on the SQUID are generally not negligible. To minimize the transmission of vibrations to the CCC, these excitation frequencies need to be dampened as much as possible. It is also essential to ensure that the eigenfrequencies of the cryostat are not compatible with any excitation frequencies expected in the accelerator environment.
       	
       	Last but not least, the installation of the CCC as part of the standard diagnostics on the accelerator beamline imposes a set of mechanical constraints:
       	\begin{itemize}
       		\item \textbf{Minimized dimensions}: Space available on an accelerator is always limited, especially inside storage rings, so the cryostat needs to occupy the smallest possible volume.
       		\item \textbf{Thermal isolation from the beam tube}: The cryogenic volume holding the CCC needs to enclose the beam tube, and the detector needs to be as close as possible to the beam while maintaining strong thermal insulation between the cryogenic chamber and the beam tube at room temperature.
       		\item \textbf{Standalone vacuum isolation}: If the detector needs to be installed in a section of the machine where the demands for vacuum quality are high\footnote{In the transfer lines in GSI, the vacuum requirements are not so strict, usually operating on the lower edge of UHV. However, especially in CRYRING@ESR and other storage rings, the vacuum quality needs to be higher ($10^{-11}$ mbar).}, the vacuum of the beamline needs to be separated from the isolation vacuum of the cryostat.
       		\item \textbf{Bake-able}: When UHV vacuums are required ($ < 10^{-11}$ mbar), the UHV tube needs to be baked up to 200 $^\circ$C without damaging the heat-sensitive CCC detector or other sensors inside the cryostat.
       	\end{itemize}
       	
       	Table \ref{CH4_tab_cryostat} provides all the general design requirements for a CCC cryostat.
       	
       	\begin{table}[h!]
       		\centering
       		\begin{tabular}{c|c}
       			\hline
       			\textbf{Parameter} & \textbf{Requirement} \\ \hline
       			Temperature & Operating temperature below 6 K \\ \hline
       			Temperature \& Pressure stability & Variations smaller than 1 mK and 1 mbar per minute \\ \hline
       			Mechanical vibrations & Decoupling from surroundings, no resonance below 15 Hz \\ \hline
       			Materials & Non-magnetic ($\mu_r \approx 1$), preferably non-conducting (electrically) \\ \hline
       			CCC detector volume & As close as possible to beamline ($R_{\text{FAIR}} = 125$ mm) \\ \hline
       			Vacuum quality & UHV-compatible, bake-able beamline (DN100) \\ \hline
       			Operating time & $>3$ days (no active cooling) / Indefinitely (with active cooling) \\ \hline
       			Volume & As small as possible (especially for storage rings) \\ \hline
       		\end{tabular}
       		\caption{\small{List of general design aspects for a CCC beamline cryostat}}
       		\label{CH4_tab_cryostat}
       	\end{table}
       	
       	The sum of all these restrictions makes the design of a beamline cryostat for a CCC extremely challenging.
       	\section{Design study for FAIR}
       The design study of the cryostat for FAIR draws significant inspiration from previous CCC cryostats that operated at GSI or are still in operation at CERN. At GSI, the first prototype of a CCC was housed in a 20-liter helium bath cryostat constructed in 1993 \cite{Schroeder1993} and used until 2015 \cite{KurianTh}. This cryostat was designed to accommodate a CCC with an inner diameter of 130 mm for the beam transfer lines at GSI. The isolation vacuum of the cryostat was combined with that of the beamline, made feasible by the less stringent vacuum requirements of transfer lines. The thermal shield was cooled by a cryocooler, which had to be switched off during measurements due to its significant mechanical perturbations. \\
       In 2017, a helium bath cryostat with a non-removable CCC detector paired with a mechanically decoupled cryocooler helium re-liquifier was built at CERN for the Antiproton Decelerator ring \cite{CERN_cryostat} \cite{CERN_cryostat_2}, and it remains in operation to this day. The thermal shield is cooled by evaporating helium gas, which is then supplied to the re-liquifier, ensuring cryogenic operations for months.\\
       Currently, an alternative solution to replace helium bath cryostats is under development at CERN. This involves a gas-flow cryostat where the helium gas is cooled by a remote cryocooler \cite{GAS_cryostat}. This approach appears promising, as previous observations at CERN have indicated that the CCC used in the AD exhibits a lower noise floor when not submerged in liquid helium, although this observation has not been confirmed at GSI. Nonetheless, the gas-flow cryostat is still in development and may represent an interesting option for future CCC systems.
       
       Although the existing solutions for CCC cryostats are based on design choices incompatible with the requirements for FAIR (e.g., necessity for continuous operation and high vacuum level at FAIR vs. selective operation and non-standalone vacuum at GSI, non-exchangeable single CCC detector at CERN vs. multiple prototypes to test and the need for easy access at FAIR), the experience gathered from the existing solutions has been instrumental in designing the actual FAIR cryostat.
       
       Ultimately, the helium bath cryostat was selected as the desired solution for the FAIR cryostat for the following reasons:
       \begin{itemize}
       	\item The operating temperature of the CCC depends on the critical temperature of its superconducting shield and the SQUID used. For all the prototypes considered for FAIR, the materials used are Lead ($T_C = 7.2$ K) and Niobium ($T_C=9.25$ K \cite{Squid_handbook_1}). The SQUIDs used in the CCCs are also low-temperature Niobium-based, so they have similar temperature restrictions. Thus, 4.2 K (the boil-off temperature of helium) ensures the superconducting condition.
       	\item The expected critical magnetic field of Lead ($B_C = 80$ mT at 0 K \cite{Pb_Tc}) and Niobium ($B_C = 180$ mT at 0 K \cite{Nb_Tc})\footnote{Niobium is a type II superconductor, so it actually has two critical field strengths. The second one is $B_{C2} = 410$ mT. If the applied magnetic field is between $B_C$ and $B_{C2}$, the magnetic field can penetrate the superconductor while maintaining zero resistance.}. Typical stray magnetic fields in the accelerator environment are around 1 mT, so quite far from $B_C$ if the temperature is at 4.2 K, ensuring no breakdown of superconductivity.
       	\item Other methods to ensure a stable temperature similar to 4.2 K were considered, including a platform cooled by a cryocooler, a gas-flow, or liquid-flow cryostat. The use of cryocoolers as the primary cooling source was excluded because even low-vibration pulse tube type cryocoolers introduce mechanical movements to the system that significantly affect the SQUID, such that any direct mechanical connection to a cryocooler needs to be avoided. Thus, the liquid helium bath is the solution that introduces the lowest amount of perturbations.
       \end{itemize}
       
       In conclusion, the liquid helium bath cryostat has been adopted \cite{Ekin} because it best fulfills the cryostat requirements. However, the use of a liquid helium bath as a cooling solution necessitates a local helium re-liquefaction system to allow continuous operation for at least several months of beam time.
       
       \subsection{Mechanical design}
       The beamline bath cryostat for FAIR, measuring \(850 \times 850 \times 1200\) mm, is depicted in Fig. \ref{CH4_cryostat}. This cryostat integrates a UHV-compatible beamline with an inner diameter of 127.2 mm. The insulation vacuum vessel, which forms the outer layer of the cryostat, comprises a support frame made of stainless steel (AISI 304) and is sealed with eight removable aluminum\footnote{The removable windows are constructed from aluminum to reduce the overall weight of the cryostat.} windows (two on each side) for easy access to the inner components. The isolation vacuum vessel features two DN250 ports at the bottom for connecting vacuum turbo pumps and an overpressure valve for safety purposes.
       
       Inside the vacuum vessel, a thermal shield made of oxygen-free copper is positioned. This shield is equipped with a gas cooling line that collects helium exhaust from the He vessel and directs it to the helium exhaust of the vacuum vessel. The initial prototype of this cooling line is a 15-meter line with a diameter of 12 mm and a thickness of 1 mm, as illustrated in Fig. \ref{CH4_Shield_old}.
       
       \begin{figure}[H]
       	\begin{subfigure}[b]{0.65\textwidth}
       		\centering
       		\includegraphics[width=\textwidth]{Plot/Chapter_4/CH4_cryostat_3D.png}
       	\end{subfigure}
       	\hfill
       	\begin{subfigure}[b]{0.35\textwidth}
       		\centering
       		\includegraphics[width=\textwidth]{Plot/Chapter_4/CH4_cryostat_close.jpg}
       	\end{subfigure}
       	\caption{\small{Left: 3D schematic of the FAIR CCC beamline cryostat. The CCC is installed inside the He vessel and encloses the beamline. The helium exhaust on top is connected to the helium re-liquifier. On both sides, the cryostat can be connected to the beamline via a DN150CF flange. Right: Image of the vacuum vessel with the helium re-liquifier installed (top-left), collected during one of the tests in the laboratory at GSI.}}
       	\label{CH4_cryostat}
       \end{figure}
       
       The thermal shield is suspended from the top lid of the vacuum vessel using four suspension rods (3 mm in diameter) made of grade 5 titanium (TiAl6V4). TiAl6V4 has a higher tensile strength than stainless steel (AISI 316) and almost half of its thermal conductivity, making it an excellent choice for cryogenic applications. Similar to the thermal shield, the enclosed helium chamber made of AISI316 stainless steel is carried by four suspensions of TiAl6V4, as depicted in Fig.\ref{CH4_Shield_old}.
       \begin{figure}[H]
       	\begin{subfigure}[b]{0.35\textwidth}
       		\centering
       		\includegraphics[width=\textwidth]{Plot/Chapter_4/CH4_he_vessel_hanging.png}
       	\end{subfigure}
       	\hfill
       	\begin{subfigure}[b]{0.55\textwidth}
       		\centering
       		\includegraphics[width=\textwidth]{Plot/Chapter_4/CH4_shield_old_nomli.png}
       	\end{subfigure}
       	\caption{\small{Left: Helium chamber hanging from the upper lid on the vacuum vessel and covered with super insulation (MLI). Right: Thermal shield during it's production process, before being covered with MLI. The hanging connection is connected to the top of the He chamber to collect the exhausted helium that than circles inside the line that is wielded to the surface.}}
       	\label{CH4_Shield_old}
       \end{figure}
       The helium chamber is also connected on the bottom of the vacuum vessel with four additional TiAl6V4 rods to stabilize its position. Both the helium chamber and the thermal shield are provided with maintenance window (Fig.\ref{CH4_He_vessel_window}) on the two sides parallel to the beam tube, that allows access for maintenance. The total capacity of the empty helium chamber is slightly above 80l. It can be equipped with a CCC, mounted over the Helium tube, with a minimum inner diameter of 250 mm, a maximum outer one of 350 mm and a maximum length of 290 mm. Inside the helium chamber the weight of the CCC is supported by two removable u-shaped platform made of fiberglass, that are fixed on the side of the chamber (Fig.\ref{CH4_He_vessel_window}). 
       \begin{figure}[H]
       	\begin{subfigure}[b]{0.5\textwidth}
       		\centering
       		\includegraphics[width=\textwidth]{Plot/Chapter_4/CH4_he_vessel_side_window.png}
       	\end{subfigure}
       	\hfill
       	\begin{subfigure}[b]{0.5\textwidth}
       		\centering
       		\includegraphics[width=\textwidth]{Plot/Chapter_4/CH4_he_vessel_bottom_view.png}
       	\end{subfigure}
       	\caption{\small{Left: Helium chamber with both maintenance window opened and CCC installed inside. The windows allows to access the inner part of the chamber to replace the detector or for maintenance Right: Under View of the chamber with the CCC installed. The two fiberglass holder support the weight of the detector and mitigate its movements.}}
       	\label{CH4_He_vessel_window}
       \end{figure}
       \subsubsection{Mechanical vibration damping}
       Various measures have been applied to this cryostat to minimize the coupling of external vibrations with the helium chamber:
       \begin{itemize}
       	\item \textbf{Decoupling of liquefier vibration}: The liquefier cold-head is connected with rigid metal tubes to its compressor. Thus, the vibration of it directly links with the lifter inserted in the helium chamber. To minimize the propagation, a diaphragm bellow stabilized by rubber feet is placed at the connection point, mechanically decoupling the lifter from the chamber.
       	\item \textbf{Decoupling of beamline vibration}: The beamline can propagate vibration from different sources installed along the accelerator. To decouple the CCC from those vibrations, two bellows can be installed between the cryostat and the accelerator beamline.
       	\item \textbf{Decoupling from vacuum pumps}: Two vacuum pumps are always running to maintain the isolation vacuum, a pre-pump (77 Hz) and a turbo-pump (820 Hz). To decouple the vibration, the turbo-pump is equipped with a meander bellow and is connected to the ground\footnote{With a concrete block when installed in the laboratory and with a steel support filled with sand when installed on the beamline} to dampen possible vibrations. The pre-pump is connected in series to it and also benefits from these damping solutions.
       	\item \textbf{Decoupling from environmental vibration}: Every installation place is subjected to environmental vibration, which can originate from an infinite number of sources such as people walking, operating cranes, air conditioning pumps, water running, operating vehicles, and more. To dampen the propagation to the helium chamber, the cryostat feet have connection points made of rubber, and the cryostat itself is never placed directly on the ground. In the laboratory, it is placed on stacked blocks of concrete, while on the beamline, it is placed on a carefully designed support\footnote{FAIR and GSI beamline are installed at +2 meters from the floor level, so the CCC needs to be placed on top of a dedicated support} that can be equipped with a 25 mm thick dampening mat\footnote{Sylomer SR28 dampening mat (900 $\times$ 800 mm with a central hole of 640 mm in diameter), Getzner Spring Solutions GmbH, Germany} placed between the support and the cryostat. The mat provides insulation from frequencies above 25 Hz with a dampening factor around 30 dB for frequencies above 125 Hz\cite{DavidThesis}.
       \end{itemize}       	
       	However, despite all these measures, some vibrations still couple with the cryostat. Figure \ref{CH4_furier_noise} shows a noise density spectrum of the DCCC\footnote{Every CCC responds slightly differently to external perturbations. The coupling depends on a variety of factors, such as weight, relative direction, material, installation position, and more, which are impossible to predict and decouple in advance with complete precision.} installed in the FAIR cryostat, highlighting the main sources of perturbations.
       	\begin{figure}[H]
       		\centering
       		\includegraphics[width=1\linewidth]{Plot/Chapter_4/CH4_furier_noise_oldshield.png}
       		\caption{\small{Current noise density spectrum from 1 to 100 Hz calculated from a background measurement of the DCCC installed in the cryostat at the GSI Laboratory. Notable perturbations are labeled with their respective sources.}}
       		\label{CH4_furier_noise}
       	\end{figure}
       	The implemented dampening solutions are efficient, and the remaining sources of perturbation are well understood:
       	\begin{itemize}
       		\item \textbf{1.4 Hz}: This is the primary frequency, corresponding to the operating frequency of the re-liquifier compressor. Several harmonics of this frequency can be observed in the noise spectra. The liquifier is essential for the operation of the cryostat; therefore, efforts are being made to minimize this noise, with a summary to be presented in a subsequent section of this chapter.
       		\item \textbf{50 Hz}: This is the frequency of the main power line to which all electrical devices are connected. The source of this noise is both mechanical and electromagnetic. One of the main power transformers at GSI is located near the test laboratory, causing the ground to propagate vibrations from the power station\footnote{Accelerometer measurements confirm that this noise source is also mechanical}. The perturbations at 50 Hz are also electrical, the power lines and the earth lines oscillates at 50 Hz, and this propagates to the CCC. While reducing the number of earth loops and connecting some devices to a separate earth helps to decrease the intensity of the perturbation, it cannot be completely eliminated.
       		\item \textbf{14.7 Hz}: This frequency is identified as the eigenfrequency of the vacuum vessel, as determined using an accelerometer\footnote{Accelerometer KS813B, Metra Meß- und Frequenztechnik e.K., 01435 Redebeul, Germany} connected to the surface of the vacuum vessel after a mechanical excitation pulse \cite{Ucar2020}.
       		\item \textbf{$\approx 30$ Hz}: This peak is attributed to the eigenfrequency of the helium vessel. Using a simplified geometry of the cryostat, the mechanical eigenmodes of the different vessels were simulated at TU Darmstadt \cite{Marsic2019cryostat}. The results indicated three eigenmodes with frequencies in the range of 30 to 32 Hz. Considering errors due to deviations in material properties and the inexact model of the cryostat, the measured peak can be associated with these eigenfrequencies.
       		\item \textbf{77 Hz}: This frequency corresponds to the operating frequency of the vacuum prepump.
       	\end{itemize}
       	To conclude, despite the numerous measures taken to reduce the coupling of mechanical vibrations with the detector, the CCC noise is still affected by these vibrations. It is important to note, however, that perturbations can couple with the detector not only mechanically, as in the cases of the 1.4 Hz and 50 Hz frequencies, so some level of perturbation will always be present.
       	
       	\subsection{Insulator gap and beam tubes}
		As discussed in Section 1.3.1, any beam transformer must incorporate a non-conductive insulator to prevent the flow of screening current through the beam tube. Without this insulator, a screening current could circulate through the beam tube, effectively shielding the detector from the beam's magnetic field. To mitigate the risk of an electrical conductive bridge, each of the three tubes in the cryostat (namely the beam tube, thermal shield, and helium tube) are fitted with a non-conductive insulator gap.
		\begin{figure}[H]
			\begin{subfigure}[b]{0.65\textwidth}
				\centering
				\includegraphics[width=\textwidth]{Plot/Chapter_4/CH4_UHV_tube.jpg}
			\end{subfigure}
			\hfill
			\begin{subfigure}[b]{0.3\textwidth}
				\centering
				\includegraphics[width=\textwidth]{Plot/Chapter_4/CH4_he_tube.png}
			\end{subfigure}
			\caption{\small{Left: UHV tube (inner diameter 127.2 mm) enclosed by the cryostat with its ceramic insulator gap and heating pads for the vacuum bake out. Right: Cryogenic helium beam tube (inner diameter 200mm) during the helium leak test after cryogenic testing. The ceramic insulator and the bellow are visible.}}
			\label{CH4_tubes}
		\end{figure}
		The standard approach, which is also employed for the CCC cryostat, involves using a ceramic ring (Al$_2$O$_3$) with metallized end caps as the vacuum-compatible, cryogenic insulator. This ceramic ring is welded to an intermediate flange made from an alloy with minimal thermal expansion, such as Invar or Kovar. This intermediate flange is then attached to the main tube, which may be constructed from materials like 316LN stainless steel for the UHV beam tube. 
		
		During temperature changes, such as those occurring during cool-down (for the He tube) or vacuum baking (for the UHV tube), the uniform dimensional change of both the ceramic gap and the alloy due to thermal expansion minimizes mechanical stresses at the interface. This is a critical aspect, as discrepancies in thermal expansion coefficients or design flaws at the interface can lead to dynamic forces that might fracture the ceramic or compromise the welds. To counteract longitudinal forces and allow for slight axial compression during the installation of the tube within the cryostat, a corrugated bellow is incorporated into the tube. Figure \ref{CH4_tubes} illustrates both the UHV beam tube and the He tube.\\
		
		\subsection{Cryogenic Design}
		The primary goal of the cryogenic design is to reduce the evaporation rate of the helium bath to a level manageable by the available helium reliquifier, which has a liquefaction rate of 15 liters per day. Estimates of the evaporation rate have been conducted by GSI and ILK, resulting in an estimated rate of 15.2 liters per day of helium\footnote{For the scope of this thesis, we will focus on empirical observations and measurements conducted in the lab that served as the basis for improvements to the cryosystem. Further details on simulations and calculations can be found in \cite{DavidThesis}}. This low evaporation rate is achieved through the meticulous design of the thermal shield, which is cooled by helium gas exiting the helium chamber at 20 K. The thermal shield maintains a stable temperature of approximately 115 K during operation, even without the reliquifier. It effectively shields against external heat radiation (the vacuum vessel remains at room temperature, approximately 300 K), thereby reducing the heat input to the helium vessel.
		
		Additionally, the helium vessel and thermal shield are equipped with multiple layers of aluminized polyester foils, known as Multilayer Insulation\footnote{RUAG Coolcat 2 NW by RUAG Space GmbH, 1120 Vienna, Austria} (MLI). The thermal shield has 20 layers, while the helium chamber has 30 layers. MLI reduces the heat input according to the relation $\dot{q}_{rad} \propto 1/(N+1)$, where \(N\) is the number of layers \cite{MLI_theory}. However, the effectiveness of MLI insulation is influenced by various factors, such as placement, applied pressure, surface imperfections, and more, necessitating experimental measurement to determine the actual reduction in heat input.
		
		Using the basic instrumentation provided with the cryostat, it is possible to measure an evaporation rate of 15.4 liters per day without the liquefier installed, and a stable shield temperature of 115 K. However, with the liquefier installed and the CCC and calibration line operational, the evaporation rate can exceed 20 liters per day, with a shield temperature of 100 K, making standalone operation unattainable under these conditions. The cryogenic operation of the cryostat is a highly complex matter, complicated by the interplay of various factors that cannot be decoupled during analysis, thus requiring further optimization efforts.
		
		\section{Performance of Cryogenic Operation}
        Stabilizing the cryogenic operation to achieve an unlimited standby time has been one of the main efforts in the development of a CCC detector for FAIR. The available diagnostics consist of 10 temperature sensors, including 4 Carbon Ceramic Sensors\footnote{SCB-CCS04 Carbon Ceramic Sensor from Cryoandmore GbR, 41472, Neuss, Germany} monitoring the helium (He) chamber, and six PT1000 sensors monitoring the thermal shield and return line. The helium chamber is also equipped with a resistive helium level sensor\footnote{Helium level sensor (635 mm active length, 4.488 $\Omega$/cm) from Cryoandmore GbR} that provides a voltage measurement corresponding to the helium level inside the chamber. 
        
        Outside of the cryostat, it is possible to install a gas flow-meter (measuring in g/H) at the exhaust to monitor the amount of helium leaving the system. Additionally, two pressure transducers\footnote{P51UL (relative) pressure sensor, SSI Technologies LLC, Wisconsin 53546, USA}\footnote{Keller PR-23SX/0.4bar/23-2316-161, pressure sensor with a sampling rate up to 1 kHz, KELLER Pressure, St. Gallerstrasse 119, 8404 Winterthur, Switzerland} can be installed to document the pressure in the helium vessel and in the helium return line. To better understand the helium cycle in the cryostat, a schematic is provided in Fig.~\ref{CH4_cryocycle}:
        
        \begin{figure}[H]
        	\centering
        	\includegraphics[width=1\linewidth]{Plot/Chapter_4/CH4_helium_cycle.png}
        	\caption{\small{Schematic of the closed helium cycle of the cryogenic support system of the CCC. The evaporating gas flows through the cooling line to the helium reliquefier, absorbing heat from the thermal shield.}}
        	\label{CH4_cryocycle}
        \end{figure}
        
        As shown in Fig.~\ref{CH4_cryocycle}, the helium gas that originates from evaporation leaves the helium chamber through an opening at the top. The return line of the thermal shield is connected to this opening via a copper tube that runs to the bottom plate of the shield. Here, it connects to the cooling line, which is in thermal contact with the side walls thanks to a silver-based soft soldering agent. The cooling line then spirals around the shield and exits the cryostat through an exhaust at the top of the vacuum vessel. From here, the gas is guided through a Cryomech helium reliquifier\footnote{Cryomech HeRL15, Cryomech Inc., Syracuse, NY 13211, USA}, which has a specified liquefaction power from room temperature helium gas of approximately 18 l/day (0.85 W @ 4.2 K). 
        
        After liquefaction, the liquid helium is reinserted into the chamber through a double-walled, vacuum-insulated helium transfer channel known as the helium-lifter. The helium cycle is consistently operated at a slight overpressure (approximately 70 mbar, which is also the pressure at which the reliquefier is most efficient) to prevent contamination from other gases. Any contaminants tend to accumulate at the cryocooler, thereby reducing the capacity of the reliquefier.
        
        While theoretically, the reliquefier should compensate for the helium evaporation, in practice, its liquefaction power is insufficient to maintain a closed system. Beyond the heat input added by the helium lifter—a metal rod that is at room temperature at the connection with the reliquefier and then in contact, or nearly in contact, with the liquid helium in the chamber—there is additional heat input due to the over-circulation of gas. 
        
        During the liquefaction process, it has been observed that the reliquefier cools down the helium gas, increasing its density. Consequently, some gas drops down the lifter into the helium chamber before being fully liquefied, adding further heat to the chamber. \\ Another issue observed is the small pressure drop between the reliquefier and the helium chamber, $\approx$ 1-2 mbar. Any additional resistive component to control the gas flow may result in the gas being drawn from the lifter instead of flowing through the helium line. If this occurs, the temperature of the thermal shield increases, introducing more heat load to the helium vessel and thereby increasing the evaporation rate.\\
        To achieve a fully closed system, the goal is to nullify helium losses through the exhaust. If the liquefaction power significantly exceeds the evaporation rate, the pressure in the chamber could drop. In such cases, a heater in the cold head of the reliquefier increases its temperature, halting liquefaction until the pressure stabilizes.
        
        In the CCC installation at CERN, a closed system with unlimited standing time is generally maintained. The system includes a gas valve that introduces gaseous helium to compensate for losses when the helium level drops below a certain threshold. The GSI cryostat is also designed to implement this solution, although it has not been used thus far.\\
        To solve the standing time issue a campaign of improvements has been performed on the cryostat, focusing also on the minimization of the noise background (in particular high frequency noise and the 1.4 Hz reliquifier perturbation)
        \section{Improvements to the cryostat}
        \subsection{Regulation of gas flow}
        Previous experience with the CCC cryostat \cite{DavidThesis} has shown that the pressure drop of the return line is comparable to that of the helium lifter. This indicates that the helium gas has two possible pathways to reach the liquefier. Consequently, the cryostat can operate in two distinct modes: the \textit{Standard Mode}, where the gas flows through the return line, maintaining the shield cold and the pressure and temperature stable, and the \textit{Oscillation Mode}, where the gas is not only flowing through the shield but also being drawn from the helium lifter. In the latter mode, the system undergoes saw-tooth oscillations over a timescale of several minutes, with rapid pressure variations up to 32 mbar, followed by a rapid discharge\footnote{Coupled to the pressure oscillations also the temperatures in the He chamber oscillates.}. During these oscillations, the liquefier temperature also fluctuates, indicating periodic discharges of liquid helium  \cite{DavidThesis}. Although the exact mechanics of this oscillation are not fully understood, a solution was found to eliminate them and consistently maintain the cryostat in the desired \textit{Standard Mode}.
        
        A polyvinyl chloride (Trovidur\textregistered) nozzle, installed at the end of the helium lifter, that reduces its diameter to 1 mm over a length of 30 mm, effectively resolves the oscillation issue. This nozzle introduces a flow resistance of approximately 1 mbar for a flow of 95 g/h of helium. An additional strong benefit of the nozzle is that it enables the installation of a flow control valve in front of the liquefier, allowing for better management of the gas flow and minimizing the effects of over-circulation.
        
        This method has also slightly improved the standing time of the system. However, the additional pressure drop is still insufficient to allow the installation of a flowmeter in front of the liquefier. Consequently, it is currently only possible to measure the flow of gas leaving the system through the exhaust line, which indicates the amount of gas that has not been reliquefied.
        
        It is important to recall the flow measurements of the system: 15.4 l/day without the reliquifier installed, and slightly higher than 20 l/day when the reliquifier is in operation (with measurements fluctuating between 20 and 22 l/day depending on the helium level in the chamber and other factors).
        
        With the reliquifier installed and operating, the typical flow rate of He gas leaving the system is around 50 g/H, although it can decrease slightly as the liquid helium level in the chamber drops. This translates to an excessive evaporation rate of approximately 8 l/day. Using the flow control valve in front of the liquefier, which has a full range of 5 turns (with 0 being fully open and 5 being fully closed), the following observations were made:
        
        \begin{itemize}
        	\item \textbf{0/5 turns}: The gas flow rate exiting the system is the same as when the valve is not installed, i.e., 50 g/h.
        	\item \textbf{2/5 turns}: The flow rate decreases significantly, stabilizing around 30 g/h. This remains consistent until the helium level in the chamber drops below 10 l, beyond which the valve setting has minimal effect.\footnote{The flow resistance provided by the valve is small, with these settings the difference in pressure between the chamber and the return line change from $\approx 5$ mbar to $\approx 3$ mbar.}
        	\item \textbf{3/5 turns}: The flow rate begins to increase significantly, and the temperature of the reliquefier's cold head drops rapidly due to insufficient gas reaching the liquefier.
        	\item \textbf{$>$ 4/5 turns}: The cryostat enters a \textit{Reverse Mode}, where almost no gas flows through the shield, leading to a sharp increase in the shield temperature and an evaporation rate rising to up to 30 l/day. This mode, however, is useful for studying the coupling of the 1.4 Hz liquefier noise with the detector, as it allows the decoupling of the return line and thermal shield from the detector due to their inactivity.
        \end{itemize}
        \begin{figure}[H]
        	\centering
        	\includegraphics[width=1\linewidth]{Plot/Chapter_4/CH4_evaporation.png}
        	\caption{\small{Results of the flow control test. By closing the exhaust, the increase in pressure within the system can be correlated to the increase in the amount of gaseous helium in the chamber, allowing for an accurate measurement of the evaporation rate. The fitted results are consistent with the flowmeter measurements.}}
        	\label{CH4_flow_control}
        \end{figure}
        The results of the flow control test are shown in Fig.\ref{CH4_flow_control}. The results indicate an optimal working point at 2/5 turns of the valve, which significantly reduces the over-circulation of gas in the system, thereby decreasing the heat input and helium losses.
        \subsection{1.4 Hz noise studies}
        The 1.4 Hz perturbation is the one with the highest magnitude for the radial CCC tested at GSI, with peaks of amplitude up to 60 nA during measurement in CRYRING \cite{DavidThesis}. While it is possible to filter it out and remove it completely, this is not desired in the analysis of slow extracted spills, as the time scale of this perturbation may be compatible with the time scale of the "DC" component of the spill. In such cases, an inaccurate filtering method could modify the data and create artifacts during analysis. The next chapter will show that the periodicity of this noise is highly accurate, making it possible to use a narrow notch filter at 1.44 Hz that can significantly dampen the perturbations without creating artifacts in the collected data. However, before the beam-line test, this was a critical concern for the idea of moving the CCC away from the storage ring and into the transfer lines. Therefore, efforts have been made to better understand this noise source and how to dampen it before post-processing the data.
        
        To better understand, an example of 1.4 Hz noise collected with the FAIR-Nb-CCC-xD is shown in Fig.\ref{CH4_1.4Hz_noise}:
        \begin{figure}[H]
        	\centering
        	\includegraphics[width=1\linewidth]{Plot/Chapter_4/CH4_liquifier_noise.png}
        	\caption{\small{Measurement of liquifier noise performed with the FAIR-Nb-CCC-xD, the CCC most sensitive to this particular source of noise. The noise is extremely periodic, with an intensity that generally depends on the pressure in the chamber (higher pressure, higher intensities) but is quite stable over a time scale of days if the pressure is stable. This is the dominant source of noise for this CCC. It can be removed with a notch filter during post-processing of the data, but finding a way to reduce its intensity can be a core part of the campaign of background suppression.}}
        	\label{CH4_1.4Hz_noise}
        \end{figure}
        
        Three possible sources for this noise have been investigated:
        \begin{itemize}
        	\item \textbf{Acoustic}: The compressor frequency is clearly audible in the laboratory. The CCC, regardless of the version, is very sensitive to all external perturbations, including acoustic ones. To verify this source, a microphone was installed in the helium return line on top of the cryostat to see if the 1.4 Hz noise could be detected. While the noise can be measured outside, it is not detectable when the microphone is installed inside the line. Thus, the source is not acoustic.
        	\item \textbf{Mechanical}: As shown in previous experiments \cite{DavidThesis}, measurements performed using an accelerometer connected to the cryostat reveal a very small signal at 1.4 Hz, much smaller than other contributions. This result suggests that the noise is not primarily mechanical.
        	\item \textbf{Pressure Vibration}: This was the most probable source. However, the installation of a faster and more precise pressure transducer has shown that the 1.4 Hz oscillations are not visible in the pressure measurements, as shown in Fig.\ref{CH4_pressure_oscillations}. Even though the technical limitations of the pressure transducers limit the information we can obtain about the pressure behavior inside the chamber, oscillations of this intensity would require a pressure variation greater than 1 mbar, which is clearly not observed, excluding pressure as the main source of this noise.
        	\begin{figure}[H]
        		\centering
        		\includegraphics[width=1\linewidth]{Plot/Chapter_4/CH4_pressure_oscillations.png}
        		\caption{\small{Measurement of the pressure inside the He chamber performed using the Keller pressure transducers. Although more points per second would be better to analyze the pressure behavior inside the chamber, this measurement is sufficient to exclude pressure as the main source of the 1.4 Hz noise. If this were the case, the measurement should show at least an hint of oscillation with a minimum peak-to-peak amplitude of 2 mbar, which is clearly not observed.}}
        		\label{CH4_pressure_oscillations}
        	\end{figure}
        \end{itemize}
        The turning point in understanding the 1.4 Hz noise was the possibility of decoupling the return line of the shield from the helium cycle thanks to the flow control valve that allows achieving the \textit{Reverse Mode}. In this operating mode, the reliquifier functions normally, but the gas is not pushed through the return line; instead, it is sucked in from the helium lifter. In this mode, the 1.4 Hz noise vanishes completely, as shown in Fig.\ref{CH4_noise_reverse_mode}. This confirms that the source is not one of the three reported above, as these perturbations remain unchanged between the normal and the \textit{Reverse} operating mode of the cryostat.
        \begin{figure}[H]
        	\centering
        	\includegraphics[width=0.8\linewidth]{Plot/Chapter_4/CH4_noise_reverse_mode.png}
        	\caption{\small{Measurement of noise signal with the CCC when the cryostat is operating in \textit{Reverse Mode}. The 1.4 Hz noise vanishes completely.}}
        	\label{CH4_noise_reverse_mode}
        \end{figure}
        
        This indicates that the 1.4 Hz couples with the helium chamber through the vibration of the return line of the thermal shield. Further confirmation of this thesis can be found by using a bypass of the return line, essentially having the helium gas pass through the sensor cable channel that is directly connected to the chamber. Even though the resistance of this line is high due to the insulation placed to avoid thermal shock to the cable, it is still preferred over the usual return line. The gas passing through the bypass is confirmed by an increase in the shield temperature. This solution also reduces the intensity of the 1.4 Hz noise.
        
        It is then possible to assume that the length of the return line and the amount of gas flowing through it are directly connected to the intensity of the 1.4 Hz noise. However, the use of the bypass or the \textit{Reverse Mode} are not sustainable ways of operating the cryostat, which needs the cooling provided by the return line to keep evaporation as low as possible. If it is possible to reduce the length of the return line, it should be possible to reduce the intensity of the 1.4 Hz noise.
        \subsection{Improvements of thermal shield}
        Based on previous observations of the cryostat behavior, a second version of the thermal shield has been designed with two main objectives: to reduce the pressure drop of the line, thereby attempting to completely eliminate any reverse flow of gas through the liquefier, and to decrease the intensity of the 1.4 Hz perturbation. Additionally, it aims to reduce the temperature of the He gas reaching the liquefier to increase its reliquefaction power.
        
        The shield now has approximately 9 meters of line length, with 18 mm of inner diameter. The line is carved inside rectangular pieces of copper, which are then shaped to remove the maximum amount of unused material without compromising their mechanical stability. Unlike the previous version, the line is not welded to the surface but is screwed to it using accurately designed guides and reinforcement plates made of aluminum. These plates help the shield support the increased weight of the line while keeping the line in the desired position with minimal deviation. The separate sections of the lines are then connected through bent tubes with an inner diameter of 16 mm and an outer diameter of 18 mm, which are welded to the two sections. Fig.\ref{CH4_new_shield} shows the new shield during installation. Once installed, it has been covered with two MLI layers, similarly to the old version.
        \begin{figure}[H]
        	\centering
        	\includegraphics[width=1\linewidth]{Plot/Chapter_4/CH4_new_shield.png}
        	\caption{\small{New shield version during the installation. It is possible to see the He chamber lowered into the new shield and the empty vacuum vessel. The new shield return line is visible, along with its support and the welded connection point.}}
        	\label{CH4_new_shield}
        \end{figure}
        The tests of the new shield, however, were not as successful as expected, and can be summarized with three main observations\footnote{Everything is exactly the same except for the new shield: same liquefier, same flow control valve, same CCC detector, to perform the best comparison between the two shield types}:
        \begin{itemize}
        	\item \textbf{Shield Temperature}: The shield temperature of approximately 100 K was, on average, 10 K lower than the temperature of the old shield during the full cool-down time span. This highlights that the general flow of gas through the shield is higher than it was for the old shield in the optimal setting. It is, however, compatible with the temperature of the old shield before the installation of the flow control valve, where it was not operated in the optimal setting. As expected, the gas reaching the liquefier is colder, leaving the shield at around 240 K instead of 270 K (previous shield).
        	\item \textbf{Flow rate of excess gas}: Using the flowmeter and fitting the data achieved with the pressure transducer, it was possible to estimate the flow rate and also test the effectiveness of the flow control valve. With the valve completely open, the evaporation rate of the new shield is 45 g/H, compared to the 50 g/H of the previous shield, showing a small improvement. However, while it was possible to find an optimal working point for the previous shield where the evaporation rate reached 30 g/H, this was not possible for the new shield. Unexpectedly, the flow control valve made no visible difference in the operation of the new shield. Only when the valve is completely closed the return line is cut out, starting the \textit{Reverse Mode} for the cryostat.
        	\item \textbf{1.4 Hz noise}: The 1.4 Hz noise was evaluated using the CCC signal. However, the expected reduction of the noise was not present. The new shield, probably due to its much increased mass (the shield box weighs 60 kg, the old line $<$ 10 kg, while the new one weighs approximately 30 kg) or the increased flow, provides a higher coupling with the detector. The 1.4 Hz noise is around 5 times higher with the new shield. Due to the increased mass, a small increase is present also in every perturbation correlated to vibrations, such as the eigenmodes of the vessels. Fig.\ref{CH4_fourier_new_shield} shows the spectral noise density of the CCC detector with the previous shield (in blue) and with the new shield (in red).
        	\begin{figure}[H]
        		\centering
        		\includegraphics[width=1\linewidth]{Plot/Chapter_4/CH4_fourier_new_shield.png}
        		\caption{\small{Spectral noise density with the previous (blue) and the new shield (red). The 1.4 Hz noise is a factor of approximately 5 times higher with the new shield. Every other frequency correlated to mechanical perturbations is also slightly higher. See Fig.\ref{CH4_furier_noise} for more information on the sources.}}
        		\label{CH4_fourier_new_shield}
        	\end{figure}
        \end{itemize}
        Summarizing, the new shield was an interesting approach to solving the standing time issue and increasing the signal-to-noise ratio. However, its results were not as predicted, so even though the new shield is still good enough to be used for FAIR, the previous version (with a lighter, smaller and longer return line) shows slightly better performance.
        
       \subsection{Liquifier Improvements}
       The final improvement of the cryogenic system is a new, more powerful, reliquifier. Technological advancements have enabled the development of a liquifier with a capacity of 25 l/day for room temperature gas, with a noise level low enough to be used in a CCC detector.
       \\
       \\
       MISSING PART ON THE LIQUIFIER TEST; TO ADD ONCE PERFORMED
       \\
       \\
       \section{Calibration Line}
       Up to now, we have described the cryostat's cryogenic functionality and the passive solutions implemented to minimize the background noise. The final component to describe is the calibration line solution integrated into the system.
       
       A feedthrough in the vacuum vessel allows connection to the calibration line, which consists of an insulated thin manganin-copper wire running through the CCC detector. The wire is as thin as possible to reduce the heat input inside the vessel. The wire has a total length of around 5 meters, resulting in a resistance of approximately $100 \Omega$ at room temperature (and $54 \Omega$ at 4.2K\footnote{The resistance of the voltage to current generator is much higher than the wire resistance, it is then possible to assume that the effect of variation in the resistance value of the wire are negligible in respect to the uncertainty of the function generator.}). It is then possible to connect a multi-function generator\footnote{MFG-2230M, GW Instek, New Taipei City 236, Taiwan} to the wire through a custom-built voltage-to-current converter, which is directly connected to the cryostat. This converter includes a low-pass filter with a cut-off frequency of around 1 MHz\footnote{This filter is necessary because the multi-function generator produces digital spikes at very high frequencies ($>$ GHz) that may cause instability in the system} and attenuates the signal by approximately 5 orders of magnitude:
       \begin{equation}
       	1 V_{pp} \Rightarrow 1.7 \mu A \pm 0.5 \% \quad \text{for } f<1 \text{ kHz}
       \end{equation}
       where the uncertainty is dominated by the accuracy of the multi-function generator. However, the calibration is usually performed using an input of 0.1-1 Vpp, where the overall uncertainty is lower than the average measurement background. The calibration factors for each CCC detector will be provided in the following chapters.
       
       \section{Slow Control Requirements}
       As described in section 3.6, an essential part of the work on the CCC detector is its integration into the standard accelerator diagnostic system. Additionally, the integration of all sensors used to monitor the CCC detector status is necessary. Table \ref{CH4_table} provides an exhaustive list of the sensors operating during the stand-alone operation of the detector:
       
       In the current setup, all these devices are read locally or through an Ethernet connection with a local laptop. The slow control system needs to provide:
       \begin{itemize}
       	\item \textbf{Generalized Readout}: A single GUI, accessible by the operators, that provides the essential values to monitor correct operation, particularly the isolation vacuum level, the turbo pump status, the helium level, and the chamber overpressure.
       	\item \textbf{Remote Control}: It needs to be possible to operate the active devices remotely. This mainly includes the control of the vacuum pumps and the liquifier heater.
       	\item \textbf{Data Acquisition}: The background of the CCC measurement depends on some of the operating values like pressure and temperature. It is necessary to be able to collect this data and combine it with the CCC measurement for background optimization during the post-processing of the data.
       	\item \textbf{Acquisition Rate}: The slow control needs to have a minimum acquisition rate to notice changes in the cryostat behavior. Previous experience has shown that significant modifications in the operating conditions can occur on the time scale of minutes, so a minimum rate of 1/10 Hz is necessary.
       \end{itemize}
		\begin{table}[H]
			\centering
			\begin{tabular}{c|c}
				\hline
				Measured Value & Number of Sensors (measurement type)  \\
				\hline
				Isolation Vacuum & 2 $\times$ Pfeiffer Pirani gauge (RS 232)\\
				Turbo Pump Operation & Pfeiffer HiPacce DCU400 Turbo pump control (25-wire cable)\\
				Pre-pump Operation & Remote control (RS 232)\\
				Overpressure of He Chamber & Keller pressure transducer (4-wire measurement) \\
				Overpressure of Return Line & P51UL pressure sensor (4-wire measurement)\\
				Overpressure of Reliquifier & P51UL pressure sensor (4-wire measurement)\\
				Reliquifier Cold-head Temperature & Cryomech controller (5-wire measurement)\\
				Temperatures & 10 $\times$ temperature sensor (4-wire measurement)\\
				He Level & Resistive helium level sensor (4-wire measurement)\\
				Calibration current & Remote control of function generator (RS 232)\\
				\hline
			\end{tabular}
			\caption{List of all operating sensors to control the CCC status with the readout used. All the multiple wire measurements generally bring a voltage between 0-5 Volts that can be used to read the measurements.}
			\label{CH4_table}
		\end{table}
       \section{FSU Jena laboratory}
       All the CCC prototypes were first tested in the controlled environment provided at the cryogenic laboratory located at FSU Jena, shown in Fig.\ref{CH4_jena_cryostat}.
     
       The setup consists of a wide-neck helium bath cryostat in which the CCC rests on a non-conductive platform made of fiberglass. The distance between the SQUID and the FLL electronics is around 1m, approximately half of the distance in the GSI cryostat (1.9m), so the measured slew rate will be higher than the effective one at GSI. The FSU cryostat is also equipped with a calibration line, connected to a multi-function generator, that can be used to apply various signals to define the performance of the installed detector. Furthermore, a Helmholtz coil is placed around the cryostat and is used to apply magnetic fields $\overrightarrow{B}_\bot$ orthogonal to the CCC axis. A second coil is wound around the cryostat to create a magnetic field $\overrightarrow{B}_\|$ that is parallel to the CCC axis. With these fields, the magnetic shielding factor A of the superconducting shield can be determined.
       
       After filling the cryostat, it is generally necessary to wait some time (in the order of a day) to reach thermal equilibrium. Before thermal equilibrium is reached, discrete jumps in the magnetization of the high-permeability core (Barkhausen noise) cause the SQUID to spontaneously lose its working point, making long measurements impossible. After thermal equilibrium is reached, the probability of these jumps decreases enough to allow for minutes or hours of continuous measurement.
       
       In order to provide reproducible conditions, the cryostat can also be placed inside an anechoic chamber to minimize external magnetic perturbations. To minimize acoustic and mechanical perturbations, measurements are collected during the night, when activity in the university is at a minimum. For these reasons, the noise background collected at FSU is very low, allowing for precise comparison between detectors and determination of the intrinsic detection limits defined by the spectral current noise of the detector.
       \begin{figure}[H]
       	\begin{subfigure}[b]{0.6\textwidth}
       		\centering
       		\includegraphics[width=\textwidth]{Plot/Chapter_4/CH4_jena_scheme.png}
       	\end{subfigure}
       	\hfill
       	\begin{subfigure}[b]{0.35\textwidth}
       		\centering
       		\includegraphics[width=\textwidth]{Plot/Chapter_4/CH4_Jena_Cryostat.png}
       	\end{subfigure}
       	\caption{\small{Test bench at FSU Jena for characterization of CCC prototypes. The beam current can be simulated with the current $I_{cal}$ running through the calibration loop.}}
       	\label{CH4_jena_cryostat}
       \end{figure}
       However, in the cryostat at GSI, both in the laboratory and on the accelerator, the noise figures are much higher, increasing the noise floor of the detector by approximately 2 orders of magnitude. In the end, it is the susceptibility of the detector to these external perturbations that determines the achievable resolution.
       
       
       
       
        \chapter{Spill analysis with radial CCC prototype for FAIR}
        \chapter{Axial Coreless CCC prototype for FAIR}
        The axial coreless CCC is the first prototype for FAIR with a radial shield design tested in the GSI cryostat. As explained in Chapter 3, the axial design provides much stronger magnetic shielding, and the absence of a high magnetic permeability core should make the detector less sensitive to external perturbations and significantly reduce the low-frequency noise caused by magnetic flux variations in the core material.
        
        The first part of this chapter will describe the tests performed in Jena, where the controlled environment allows for the measurement of the detector's maximum capability. The second part of this chapter will provide a description of the performance in the GSI cryostat.
        \section{SQUIDs parameters}
        A critical aspect of the coreless CCC prototype is the characterization of the SQUIDs employed within the system. As outlined in Chapter 3, the coreless CCC is equipped with two distinct SQUIDs, each serving a specific role:
        
        \begin{itemize}
        	\item \textbf{SQUID CN4}: The primary and more sensitive SQUID, which is amplified by an ultra-low noise SQUID-based array, has an inductance of 44 nH and operates in Flux-Locked Loop (FLL) mode.
        	\item \textbf{SQUID CN2}: A less sensitive SQUID, with an inductance of 11 nH, capable of operating solely in direct mode.
        \end{itemize}
        
        SQUID CN4 serves as the primary sensor due to its higher sensitivity, while SQUID CN2 is only utilized when the system's slew rate exceeds the maximum handling capacity of CN4. Consequently, a comprehensive characterization of CN4 has been performed.
        
        The system is controlled via a single-channel Supracon SQUID electronics system\footnote{JESSY SQUID system, developed by Supracon AG}, which includes a flux-locked loop (FLL) module positioned as close to the SQUID as possible. This module integrates both a pre-amplifier and an integrator, with a separate control unit managing the overall operation. While the electronics are designed to drive both SQUIDs, the bias settings are primarily optimized for SQUID CN4. The bias settings provided to CN4 are also applied to CN2, which limits the flexibility of the system's configuration. 
        
        Additionally, a double-stage SQUID electronics system, also provided by Supracon, can be employed. This system allows for the independent adjustment of the bias values for both CN2 and the amplifier array, enabling more precise tuning of CN4's operational point. 
        
        When the system is driven by single-channel electronics, the settings that yield the highest transfer function for the sensitive SQUID (CN4) may not correspond to the settings that minimize the noise of the amplifier or of CN2. Consequently, it is often necessary to search for an optimal balance between the noise level and the transfer function amplitude. However, with the double-stage SQUID electronics, it is possible to independently adjust each bias, allowing for the identification of the point that simultaneously achieves the best transfer function and the lowest noise. Therefore, the SQUID parameters are measured using the double-stage electronics to ensure optimal performance.
        
        
        The parameters of SQUID CN4 were determined by applying a test current, $I_\Phi^{fb}$, through the feedback coil of the FLL loop or by using a current, $I_{cal}$, passing through the calibration coil installed in the FSU cryostat. The first task was to identify the optimal working point, which minimizes the noise and maximizes the system's slew rate. This optimal point is defined by the combination of bias current ($I_b$) and applied magnetic flux ($\Phi_a$) that maximizes the SQUID's transfer function $|\partial V_{sq}/\partial\Phi_a|$ (see Section 2.3).
        
        The characteristics of SQUID CN4 are depicted in Fig.~\ref{CH6_modulation}. It is evident that, unlike the SQUIDs discussed in Chapter 5, the use of double-stage SQUID electronics alters the shape of the SQUID's characteristic curve, deviating from the typical sinusoidal waveform.
        
        \begin{figure}[H]
        	\centering
        	\includegraphics[width=1\linewidth]{Plot/Chapter_6/CH6_modulation.png}
        	\caption{\small{Periodic modulation of the output voltage $V_{SQ}$ of CN4 in response to an applied magnetic flux generated by the current $I_{\Phi}$ circulating through the feedback coil.}}
        	\label{CH6_modulation}
        \end{figure}
        
        Despite this deviation, the SQUID voltage oscillates with a period corresponding to a single magnetic flux quantum, $\Phi_0$, as described by the critical current equations (eq.~\ref{CH2_eq_critical_current} and eq.~\ref{I_V_char}). This periodicity enables the extraction of the mutual inductance of the feedback coil:
        
        \begin{equation}
        	\frac{1}{M_f} = 18.6 \quad \mu A/\Phi_0
        \end{equation}
        
        The optimal working point is located at the steepest slope of the modulation curve, where the SQUID transfer function, $V_\Phi = |\partial V_{sq}/\partial\Phi_a|$, reaches its maximum value:
        
        \begin{equation}
        	V_\Phi = |\partial V_{sq}/\partial\Phi_a| = 135.5 \mu V/\mu A = 2497 \mu V/\Phi_0
        \end{equation}
        
        This value already accounts for the amplification provided by the pre-amplifier. When compared to the Magnicon SQUID used in the FAIR-Nb-CCC-xD, and as will be further discussed in Chapter 7 in relation to the Magnicon SQUIDs employed in the DCCC, it is evident that the transfer function of the coreless CCC's SQUID is significantly larger. This enhancement leads to a substantial reduction in the voltage noise produced by the amplifier ($S_{V,AMP}$). Moreover, the gain $G_{SQUID}$ of the SQUID, when operating in FLL mode, can be expressed as:
        
        \begin{equation}
        	G_{SQUID} = \frac{V_\Phi M_f}{R_f}  = 6.7 \times 10^{-4} \qquad \text{with} \qquad R_f= 200\, \text{k}\Omega	
        \end{equation}
        
        In the JESSY FLL readout system, the gain-bandwidth product (GBP) is fixed at 6 GHz, and the static feedback resistor $R_f$ is selected prior to the measurement campaign to achieve the desired system bandwidth and measurement range. Given the loop delay $t_d = 20$ ns of the FAIR cryostat, the maximum system bandwidth of $f_{3\,\text{dB, max}} = 9$ MHz (see Eq.~\ref{CH2_maxf}) can be obtained with a feedback resistor $R_f = 200$ k$\Omega$, according to:
        
        \begin{equation}
        	f_{3\,\text{dB, max}} = \frac{1}{2\pi t_d} = 2.25 \, G_s \times \text{GBP} = 9 \, \text{MHz} \qquad \text{with} \qquad \text{GBP} = 6 \, \text{GHz}
        \end{equation}
        
        Given a unit gain frequency $f_1(R_f,\text{GBP}) = 4 \, \text{MHz}$ (where $f_1 t_d = 1/(4\pi)$), the maximum slew rate of the system, according to Eq.~\ref{Theoretical_slewrate}, is given by:
        
        \begin{equation}
        	\dot{\Phi}_{f,\text{max}} = \frac{\delta \Phi}{2 t_d} \lesssim \Phi_0 f_1 = 4 \frac{\Phi_0}{\mu s} \approx 17.1 \, \mu \text{A}/ \mu \text{s}
        \end{equation}
        
        This calculation uses the mutual beam inductance $M_a$, which can be determined by applying a current through the calibration line and observing the resulting voltage modulation of the SQUID. The measured mutual inductance is:
        
        \begin{equation}
        	\frac{1}{M_a} = 4.3 \, \mu \text{A}/\Phi_0
        \end{equation}
        
        Here, the mutual beam inductance $M_a$ quantifies the coupling between the SQUID and the beam current.
        
        All the values provided above were obtained during the testing of the axial coreless CCC at FSU Jena, prior to the installation of the detector in the GSI cryostat. Once installed, it is possible to modify these values by adjusting the feedback resistor $R_f$ and the gain-bandwidth product (GBP) of the FLL circuit. Notably, the coreless CCC was also tested at GSI using the SQUID electronics provided by Magnicon, which allows for these adjustments during measurement. Thus, the presented values should be considered as illustrative examples of the coreless axial CCC's potential performance.
        
        For the SQUID CN2, which features a broader dynamic range but lower sensitivity and operates exclusively in direct mode, the modulation can be estimated as:
        
        \begin{equation}
        	\frac{1}{M_{f,CN2}} = 32.5 \, \mu \text{A}/\Phi_0
        \end{equation}
        
        This value is derived after accounting for the 1/20 divider. When measuring with CN2, it is essential to count the number of modulations before the integrator reset, which is proportional to the applied current. Hence, the modulation is the key parameter for CN2. The other parameters provided for CN4, which are related to its operation in FLL mode, are not applicable to CN2 as it does not operate in this mode.
        
        Finally, as with the FAIR-Nb-CCC-xD, the slew rate of the detector was measured by applying a 200 kHz sine wave and gradually increasing its amplitude until the CN4 SQUID could no longer maintain operation in FLL mode. With the optimal working point settings, the coreless axial CCC achieved a maximum slew rate of:
        \begin{equation}
        	\text{Slew Rate}_{\text{max}} = 3.43\,\mu\text{A}/\mu\text{s} \qquad \text{(@ 200 kHz)}
        \end{equation}
        This value is more than 20 times higher than the slew rate of the FAIR-Nb-CCC-xD, indicating a substantial improvement in the detector's ability to track rapidly changing currents. This enhancement is particularly crucial for the intended application of the detector in monitoring the transfer lines.
        
        
        
        \section{Screening Factor}
        The shielding factor of the CCC against an external magnetic field was experimentally determined. A uniform magnetic field $B_a$ was generated at the location of the CCC using a Helmholtz coil that enclosed the cryostat (see Fig.~\ref{CH4_jena_cryostat}). The magnetic field was applied with its primary direction either parallel or orthogonal to the axis of the CCC, with field strengths of up to 225~$\mu$T in the orthogonal direction and up to 1~mT in the axial direction. 
        
        The CCC was subjected to both a static magnetic field and a time-varying magnetic field, modulated according to $B_a(t) = B_0\sin(2\pi f t)$, with fixed frequencies ranging from 1~Hz to 10~Hz. These excitation frequencies were specifically chosen to assess the performance of the superconducting shield, as the cryostat is nearly transparent to such low frequencies. At higher frequencies, eddy currents can be induced in the cryostat, which act as a shield and consequently reduce the intensity of the magnetic field reaching the detector.
        
        Unlike the FAIR-Nb-CCC-xD, where the magnetic field excitation can be directly observed on an oscilloscope, the perturbations in the coreless CCC are only discernible using a spectrum analyzer. This distinction is crucial because it implies that when evaluating the contribution of the magnetic field perturbation to the signal, it is not possible to isolate this perturbation from the general noise background. As a result, the reported shielding factor is generally underestimated.
        
        However, with an excitation signal at 5 Hz, a peak value of 12~$mV_{pp}$ was detected, corresponding to an induced flux in the SQUID of:
        \begin{equation}
        	\frac{12\,mV_{pp}}{2\,V/\Phi_0} = 6\,m\Phi_0 \quad \text{with} \quad \text{BeamCoupling}_{\text{FLL}} = 2\,V/\Phi_0
        \end{equation}
        By dividing the generated flux by the applied magnetic field, the effective parasitic area can be determined as:
        \begin{equation}
        	A_{\text{eff}}^{(\text{par})} = \frac{\Delta\Phi_{\text{sq}}}{\Delta B_a} = 0.015\,\mu\text{m}^2
        \end{equation}
        in the Z direction (along the axis of the detector). A similar measurement in the orthogonal direction (X direction, perpendicular to the detector's axis) yields an orthogonal parasitic area of 0.2~$\mu\text{m}^2$. These values can be compared to the effective area of the pick-up coil, $A_{\text{eff}}^{(\text{pick-up})} = 3.8\,\text{cm}^2$, which is used for detecting the magnetic field from an ion beam and depends on the inductance and geometry of the detector \cite{CorelessCCC}.
        
        Using the effective area and the parasitic area, the attenuation factor for external magnetic fields can be estimated as:
        \begin{equation}
        	A_{\text{shield}} = \frac{A_{\text{eff}}^{(\text{pick-up})}}{A_{\text{eff}}^{(\text{par})}}
        \end{equation}
        This calculation results in $A_{\text{axial}} > 200$~dB and $A_{\text{orthogonal}} = 183$~dB, representing a significant improvement over the magnetic screening factor of the radial FAIR-Nb-CCC-xD. The difference between the axial and orthogonal directions is expected, as the shielding factor is inherently stronger along the meanders. For a more comprehensive comparison, the coupling factor of the external magnetic field with the SQUID, which is 0.003~$nA/\mu T$, can be used to directly compare it with the values obtained for the other two prototypes, as shown in Fig.\ref{CH3_screening_factor}.
        
        \section{Sensitivity}
       The sensitivity of the detector was also assessed using the calibration line integrated into the FSU Jena cryostat. By applying a slowly decreasing calibration current ($I_{cal}$), the minimum detectable sensitivity of the coreless CCC was determined. Figure~\ref{Ch6_wave} illustrates the response of the coreless CCC SQUID to a rectangular wave input with an amplitude of 15~nA$_{pp}$. It is evident that the CCC accurately tracks the variations in the input current. However, it should be noted that the coreless CCC exhibits higher susceptibility to high-frequency perturbations compared to the FAIR-Nb-CCC-xD, which is reflected in a higher noise floor. This increased noise level becomes significant when detecting low-intensity signals.
       
       As depicted in Figure~\ref{CH6_sensitivity}, the minimum detectable current is approximately 4~nA, which is slightly above the noise threshold. Consequently, the sensitivity of the coreless CCC is somewhat inferior to that of the FAIR-Nb-CCC-xD. Specifically, the noise floor of the coreless CCC is about two orders of magnitude higher than that of the radial version under the same conditions. A direct comparison can be performed by collecting the signals using a spectrum analyzer.
       
       \begin{figure}[H]
       	\centering
       	\includegraphics[width=1\linewidth]{Plot/Chapter_6/CH6_wave.png}
       	\caption{\small{Coreless CCC SQUID signal (in red) excited by a signal provided through the calibration line (in red), with an amplitude of 15~nA$_{pp}$. The SQUID signal accurately tracks the input signal, but with a noticeable noise floor.}}
       	\label{Ch6_wave}
       \end{figure}
       
       \begin{figure}[H]
       	\centering
       	\includegraphics[width=1\linewidth]{Plot/Chapter_6/CH6_resolution.png}
       	\caption{\small{Smallest detectable signal by the coreless CCC. For clarity, the input signal was switched to a continuous triangular wave (in blue) with an amplitude of 4~nA$_{pp}$. The SQUID signal (in red) continues to track the input, which is nearly at the noise floor level.}}
       	\label{CH6_sensitivity}
       \end{figure}
       \section{Current noise}
        \begin{figure}[H]
        	\centering
        	\includegraphics[width=1\linewidth]{Plot/Chapter_6/CH6_noise_2.png}
        	\caption{\small{Current spectral noise density (rms) of the coreless CCC (CN4), measured at FSU Jena after the system reached thermal equilibrium. Prominent noise sources include mechanical perturbations between 1 and 100 Hz. An intrinsic white noise level of approximately $20 \, \text{pA}/\sqrt{\text{Hz}}$ is observed at frequencies around 10 kHz.}}
        	\label{CH6_noise} 
        \end{figure}
        
        The current noise spectral density of the coreless CCC was measured using a spectrum analyzer, with the results presented in Fig.~\ref{CH6_noise}. These measurements were conducted in the FSU Jena cryostat after the system had reached thermal equilibrium\footnote{It is noteworthy that, unlike the FAIR-Nb-CCC-xD and DCCC, the coreless CCC does not exhibit flux jumps during the transition to equilibrium, as predicted, due to the absence of a magnetic core.}. The noise profile is a combination of intrinsic SQUID noise, thermal noise from the signal filter, and external environmental perturbations:
        
        \begin{itemize}
        	\item \textbf{SQUID:} The intrinsic flux noise density of the two-stage SQUID (Supracon CN4) using the JESSY readout system is specified as $S_\Phi < 2.5 \, \mu\Phi_0/\sqrt{\text{Hz}}$, and the white voltage noise density of the preamplifier is given as $S_{V,\text{AMP}} = 0.33 \, \text{nV}/\sqrt{\text{Hz}}$. This results in a total noise density for the SQUID of less than $2.83 \, \mu\Phi_0/\sqrt{\text{Hz}}$ (Eq.~\ref{CH2_SQUID_noise}), corresponding to a current noise of approximately $12.3 \, \text{pA}_{\text{rms}}/\sqrt{\text{Hz}}$, when considering the coupling between the SQUID and the beam current. From Fig.~\ref{CH6_noise}, the measured white noise level of the SQUID is observed to be $\lesssim 20 \, \text{pA}_{\text{rms}}/\sqrt{\text{Hz}}$ at frequencies around 10 kHz, where the cryostat acts as a Faraday cage, effectively shielding the system from external electromagnetic perturbations.
        	
        	\item \textbf{Thermal noise:} Simulations of the thermal noise density of the signal filter \cite{DavidThesis} indicate that its noise is highly localized around the resonance frequency of 1.2 MHz, with no significant contributions below 100 kHz, which aligns well with the simulated spectrum of the SQUID circuit (Fig.~\ref{CH3_Coreless_bandwidth}). In Fig.~\ref{CH6_noise}, an increase in noise above 100 kHz can be observed, confirmed by the noise acquisition shown in Fig.~\ref{CH7_noise}, where the noise of the coreless CCC (in black) is compared with that of the DCCC (in red). The noise increases drastically for frequencies above 100 kHz, indicating the detector's heightened sensitivity to high-frequency perturbations.
        	
        	While thermal noise contributes to this increase, the significant rise in noise suggests that the detector is more sensitive to high-frequency perturbations than previous versions. Earlier iterations of the detector struggled to maintain a stable working point when subjected to sources that induced coupling between high-frequency perturbations and the detector. For instance, the FSU cryostat, can be equipped with a helium level sensor along the detector axis, that can pick up high-frequency perturbations and propagate them inside the cryostat, functioning as an antenna.
        	
        	To assess the system's susceptibility to external perturbations, a custom-built emitter capable of emitting signals with 1 mW of power at frequencies ranging from 20 MHz to several GHz was employed. The latest coreless CCC version successfully maintained a stable working point under such excitation, without degradation in SQUID performance. However, the introduction of this "antenna" led to an increase in background noise, suggesting that the absence of filtering elements in the SQUID circuit—such as the low-pass filter integrated into the FAIR-Nb-CCC-xD—might have a more significant impact on the CCC's performance than initially anticipated. 
        	
        	It is worth to note that earlier versions of the detector exhibited far worse behavior under similar conditions. These issues were mitigated through improved grounding, correction of meander shortcuts, and better cabling of the SQUID circuit. Consequently, while this prototype has already demonstrated enhanced performance, particularly in its resistance to high-frequency perturbations, there remains potential for further improvements. Specifically, attention should be given to additional damping of high-frequency perturbations to optimize the detector's overall performance.
        	
        	\item \textbf{Environmental perturbations:} These are the main contributors to noise in the frequency range below 1 kHz, primarily originating from mechanical and thermal sources, including pressure-related effects. At low frequencies ($<10$ \,Hz), the noise exhibits a $1/f^\alpha$ behavior. A key factor in this low-frequency noise is temperature fluctuations, such as drifts in the operating environment, which generate corresponding signals in the SQUID. The core has traditionally been considered the primary source of low-frequency noise; however, the coreless CCC remains highly sensitive to temperature (pressure) oscillations. This suggests that the core's actual effect on the system's susceptibility to temperature oscillations is negligible, and that the oscillations impact on the SQUID and the entire CCC are the main sources of coupling. Accurately quantifying the real effectiveness of the detector components on noise is challenging due to the strong environmental link and the difficulty in predicting or decoupling these fluctuations, necessitating further experiments specifically designed for this purpose.
        	
        	Finally, the noise spectrum between 10 and 100 Hz is dominated by mechanical and electromagnetic perturbations. The sources are the same as those described in Chapter 5\footnote{An additional peak at 5 Hz, attributed to the air conditioning motor's rotation frequency at the university, is present, as these tests were conducted during summer.}. Contrary to expectations, the absence of the core and the rigid connection of the meanders, achieved by using cryoglue, did not significantly impact the detector's susceptibility to mechanical perturbations. Therefore, the relative motion of the core or between the meander layers can be excluded as the primary cause of microphony. Additionally, magnetostriction of the core can be ruled out as the primary coupling mechanism of mechanical vibration. The exact coupling mechanism of the magnetic perturbation remains unclear and requires further investigation to eliminate or mitigate this noise source. The DCCC, described in Chapter 3 and analyzed in detail in Chapter 7, theoretically offers an improvement in minimizing the effects of external perturbations. By utilizing the redundant readout of the two nearly symmetrical systems, it is theoretically possible to subtract the signals from the two SQUIDs, thereby removing external perturbations that should affect both symmetrical pick-up coils equally.
        	
        \end{itemize}
    \section{Test in GSI cryostat}
	The tests conducted in Jena demonstrated that within the controlled environment provided by the wide-neck cryostat at the FSU laboratory, the Coreless CCC exhibits sufficient sensitivity, even if lower than the FAIR-Nb-CCC-xD one. The detector shows exceptional magnetic shielding, with a significant improvement over the radial detector, and possesses a much better slew rate. However, it also has a higher noise level, particularly at high frequencies.
	
	\begin{figure}[H]
		\begin{subfigure}[b]{0.42\textwidth}
			\centering
			\includegraphics[width=\textwidth]{Plot/Chapter_6/CH6_installation.jpg}
		\end{subfigure}
		\hfill
		\begin{subfigure}[b]{0.58\textwidth}
			\centering
			\includegraphics[width=\textwidth]{Plot/Chapter_6/CH6_installation_2.jpg}
		\end{subfigure}
		\caption{\small{Coreless CCC during installation in the GSI cryostat. Left: Before installation in the helium vessel; Right: After installation, bottom view. The beam tube passing through the detector acts as an antenna for high-frequency perturbations.}}
		\label{CH6_coreless_installation}
	\end{figure}
	
	As shown in Fig.~\ref{CH6_coreless_installation}, taken during installation in the cryostat, the beam tube passing through the detector acts as an antenna that couples high-frequency perturbations with the detector. This raised concerns about the coreless CCC's increased susceptibility to such disturbances. Additionally, the average noise level in the GSI cryostat is typically one to two orders of magnitude higher than that in FSU Jena, suggesting that the higher noise of the coreless CCC might not result in performance improvements over the FAIR-Nb-CCC-xD. On the other hand, the enhanced magnetic shielding and improved slew rate indicated the potential for a more stable system.
	
 
 
    \printbibliography
    
\end{document}
